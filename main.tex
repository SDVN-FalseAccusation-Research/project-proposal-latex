\documentclass[12pt, a4paper]{report}
\usepackage{graphicx} % For including images
\usepackage{titlesec} % For customizing section titles
\usepackage{tocloft} % For customizing table of contents
\usepackage{acro} % For acronyms
\usepackage{rotating}
\usepackage{multirow}
\usepackage[table]{xcolor}
\usepackage{array}
\usepackage{tabularx}
\usepackage{longtable}
\usepackage{float}
\usepackage{setspace}


%\usepackage{hyperref} % For clickable links in the document
%% ____Bibliography____%%
\usepackage[numbers,sort&compress]{natbib}
\usepackage{chapterbib}
\usepackage[breaklinks]{hyperref}
%\hypersetup{colorlinks=true,citecolor=blue,linkcolor=blue,urlcolor=blue}
% Page margins
\usepackage[left=1in, right=1in, top=1in, bottom=1in]{geometry}

% Remove page number from the first page
\thispagestyle{empty}

% Customize table of contents, list of figures, and list of tables
\renewcommand{\cfttoctitlefont}{\hfill\Large\bfseries}
\renewcommand{\cftaftertoctitle}{\hfill}
\renewcommand{\cftloftitlefont}{\hfill\Large\bfseries}
\renewcommand{\cftafterloftitle}{\hfill}
\renewcommand{\cftlottitlefont}{\hfill\Large\bfseries}
\renewcommand{\cftafterlottitle}{\hfill}

\DeclareAcronym{SDVN}{
  short = SDVN,
  long  = Software-Defined Vehicular Networks
}
\DeclareAcronym{SDN}{
  short = SDN,
  long  = Software-Defined Networking
}
\DeclareAcronym{VANET}{
  short = VANET,
  long  = Vehicular Ad Hoc Networks
}
\DeclareAcronym{V2V}{
  short = V2V,
  long  = Vehicle-to-Vehicle
}
\DeclareAcronym{V2I}{
  short = V2I,
  long  = Vehicle-to-Infrastructure
}
\DeclareAcronym{V2P}{
  short = V2P,
  long  = Vehicle-to-Pedestrian
}
\DeclareAcronym{V2N}{
  short = V2N,
  long  = Vehicle-to-Network
}
\DeclareAcronym{RSU}{
  short = RSU,
  long  = Roadside Units
}
\DeclareAcronym{OBU}{
  short = OBU,
  long  = On-Board Units
}
\DeclareAcronym{GNN}{
  short = GNN,
  long  = Graph Neural Networks
}
\DeclareAcronym{LSTM}{
  short = LSTM,
  long  = Long Short-Term Memory
}
\DeclareAcronym{ZKP}{
  short = ZKP,
  long  = Zero-Knowledge Proofs
}
\DeclareAcronym{ZKRP}{
  short = ZKRP,
  long  = Zero-Knowledge Range Proofs
}
\DeclareAcronym{BFT}{
  short = BFT,
  long  = Byzantine Fault Tolerance
}
\DeclareAcronym{API}{
  short = API,
  long  = Application Programming Interface
}
\DeclareAcronym{DSRC}{
  short = DSRC,
  long  = Dedicated Short-Range Communications
}
\DeclareAcronym{GPS}{
  short = GPS,
  long  = Global Positioning System
}
\DeclareAcronym{NTP}{
  short = NTP,
  long  = Network Time Protocol
}
\DeclareAcronym{RSA}{
  short = RSA,
  long  = Rivest-Shamir-Adleman
}
\DeclareAcronym{ECDH}{
  short = ECDH,
  long  = Elliptic Curve Diffie-Hellman
}
\DeclareAcronym{AES}{
  short = AES,
  long  = Advanced Encryption Standard
}
\DeclareAcronym{HMAC}{
  short = HMAC,
  long  = Hash-based Message Authentication Code
}
\DeclareAcronym{SHA}{
  short = SHA,
  long  = Secure Hash Algorithm
}
\DeclareAcronym{BERT}{
  short = BERT,
  long  = Bidirectional Encoder Representations from Transformers
}
\DeclareAcronym{QoS}{
  short = QoS,
  long  = Quality of Service
}


\begin{document}


\thispagestyle{empty}

\begin{center}

\begin{center}
 \includegraphics[width=2cm,keepaspectratio=true]{uor_logo.jpg}
 % uor_logo.jpg: 236x331 pixel, 72dpi, 8.33x11.68 cm, bb=0 0 236 331
\end{center}

\vspace{1.5cm}
\begin{huge}
%%%%%%%%%%%%%%%%%%%%%%%%%%%%%%%%%%%%%%%%%%%%%%%%%%%%
% Project title
%%%%%%%%%%%%%%%%%%%%%%%%%%%%%%%%%%%%%%%%%%%%%%%%%%%%
A Blockchain, ZKP, and GNN-LLM-based Combined Defense for False Accusation Attack Mitigation in Software Defined Vehicular Networks
%%%%%%%%%%%%%%%%%%%%%%%%%%%%%%%%%%%%%%%%%%%%%%%%%%%%
\end{huge} \\
\vspace{1cm}

\begin{normalsize}
An undergraduate project proposal report submitted to the
\end{normalsize}\\
\vspace{1cm}

\begin{large}
Department of Electrical and Information Engineering\\
Faculty of Engineering\\
University of Ruhuna\\
Sri Lanka
\end{large}\\

\vspace{1cm}

\begin{normalsize}in partial fulfillment of the requirements for the \end{normalsize}\\
\vspace{1cm}

\begin{large}\textbf{Degree of the Bachelor of the Science of Engineering Honours}\end{large}\\

\vspace{1cm}
\begin{normalsize}by  \end{normalsize}
\vspace{1cm}

\begin{tabular}[h]{lll}
 %%%%%%%%%%%%%%%%%%%%%%%%%%%%%%%%%%%%%%%%%%%%%%%%%%%%%%%%%%
 % Names and Registration Numbers
 %%%%%%%%%%%%%%%%%%%%%%%%%%%%%%%%%%%%%%%%%%%%%%%%%%%%%%%%%%
 M.R.M. Ashfaq	& - & 	EG/2021/4417\\
 T. Jathurshan	& - & 	EG/2021/4568\\
 M.F.A. Munsif	& - & 	EG/2021/4684\\
 M.K.M. Shamil 	& - &	EG/2021/4810
 %%%%%%%%%%%%%%%%%%%%%%%%%%%%%%%%%%%%%%%%%%%%%%%%%%%%%%%%%%
\end{tabular}\\
\vspace{1cm}

20th January 2026\\
\vspace{1cm}

%%%%%%%%%%%%%%%%%%%%%%%%%%%%%%%%%%%%%%%%%%%%%%%%%%%%%%
% if one supervisor
%%%%%%%%%%%%%%%%%%%%%%%%%%%%%%%%%%%%%%%%%%%%%%%%%%%%%%%
%  .............................................. \\
% Prof. A.B.C. Dee\\
% (Supervisor)


%%%%%%%%%%%%%%%%%%%%%%%%%%%%%%%%%%%%%%%%%%%%%%%%%%%%
% If one supervisor
%%%%%%%%%%%%%%%%%%%%%%%%%%%%%%%%%%%%%%%%%%%%%%%%%%%%
.............................................. \\
Dr. P.A.D.S.N. Wijesekara\\
(Supervisor)


\end{center}

%%%%%%%%%%%%%%%%%%%%%%%%%%%%%%%%%%%%%%%%%%%%%%%%%%%%%%%%%%%%%%%%%%%%%%%%%%%%%%%%%%%%%%%%%%%%%%%%%%
% END OF FILE
%%%%%%%%%%%%%%%%%%%%%%%%%%%%%%%%%%%%%%%%%%%%%%%%%%%%%%%%%%%%%%%%%%%%%%%%%%%%%%%%%%%%%%%%%%%%%%%%%%


\renewcommand{\thepage}{\roman{page}} % Start page numbering in roman

\chapter*{Abstract}
False accusation attacks represent serious security threats against Software, Defined Vehicular Networks (SDVN), wherein malicious nodes maliciously level accusations against benign nodes in order to influence trust and reputation mechanisms. False accusations can dramatically influence routing decisions and significantly impede SDVN operations and stability. This research identifies four common false accusation attack variants: opportunistic innocuous Single, Accuser Fabrication, Sybil, Flooded Consensus, Timing Disruptions in Noise, rich periods, and Tampered, Logs, based Evidence Spoofing. \\ \\
Existing SDVN countermeasures mostly treat false accusations independently, lack secure privacy, preserving verification protocols, and do not sufficiently model high, complexity false accusation behaviors across attack variants, limiting functionality.\\ \\
This research proposes an innovative multi, layered solution architecture that synthesizes blockchain, Zero, Knowledge Proofs (ZKPs), and Graph Neural Networks enhanced with Large Language Models (GNN, LLMs). Blockchain facilitates replicable, distributed, and tamper, evident logging of reputation and false accusations records. ZKPs allow confidential verification of accusations. GNN, LLMs facilitate topology analysis, as well as inferences of unseen high, complexity false accusation attack variants by learning behavioral graph interactions. \\\\
Our approach is evaluated and validated across all four false accusation variants in simulations, revealing its ability to significantly defend against false accusation attacks without compromising network operation and user privacy. To the best of our knowledge, this is the first use of blockchain, ZKPs, and GNNs, LLMs in tandem to deny false accusation attacks in SDVNs.
\newpage

% Table of Contents
\tableofcontents
\newpage

% List of Figures
\listoffigures
\newpage

% List of Tables
\listoftables
\newpage

% Acronyms
\addcontentsline{toc}{chapter}{Acronyms} % Add to table of contents
\acuseall % Use all acronyms to ensure they appear in the list
\printacronyms
\newpage

\renewcommand{\thepage}{\arabic{page}} % Start page numbering in arabic 
\setcounter{page}{1} % start page numbering from 1
\setcounter{secnumdepth}{3}

% Main Content
\chapter{Introduction}
\section{Evolution of Networking Paradigms: From SDN to SDVN}
To understand the security challenges in Software-Defined Vehicular Networks, it is essential to examine the evolution from traditional Software-Defined Networking through Vehicular Ad-hoc Networks to the integrated SDVN architecture.

\subsection{Software-Defined Networking (SDN)}
Software-Defined Networking (SDN) represents a revolutionary paradigm shift by fundamentally decoupling the control plane from the data plane \cite{godanj2016simple}. In traditional networks, both intelligence for routing decisions (control plane) and packet forwarding (data plane) reside together within network devices. SDN addresses these limitations through architectural separation where the control plane is extracted and centralized into software-based SDN controllers, while the data plane remains in simplified network devices focusing solely on packet forwarding \cite{godanj2016simple}.

The SDN architecture is composed of three layers:
\begin{itemize}
	\item \textbf{Application plane:} the network applications that specify the desired network behavior and policies, interacting with the control plane through the northbound API \cite{kreutz2014software}.
	\item \textbf{Control plane:} SDN controller that maintains a global view of the network state, makes routing decisions, and programs the forwarding devices through the southbound API \cite{kreutz2014software}.
	\item \textbf{Data plane:} physical and virtual forwarding devices that forward packets according to flow tables installed by the network controller \cite{kreutz2014software}.
\end{itemize}

\begin{figure}[H]
    \centering
    \includegraphics[width=0.85\textwidth]{diagrams/1_SDN.png}
    \caption{SDN Data Plane Architecture \cite{nunez2023brief}}
    \label{fig:sdn}
\end{figure}

\subsection{Vehicular Ad,  hoc Networks (VANET)}
A specialized form of a Mobile Ad, hoc Network, Vehicular Ad,  hoc Networks (VANET), offer vehicle,  to,  vehicle communication (V2V) and vehicle,  to,  infrastructure (V2I) communication \cite{raja2010issues}. VANETs provide a means of communication for vehicles equipped with On,  Board Units to communicate with fellow vehicles and infrastructure located around the road \cite{raja2010issues}.

 Features of VANET:
\begin{itemize}
 \item \textbf{Intensely dynamic topology:} As vehicles travel at various speed, link failure occurs frequently \cite{raja2010issues}.
 \item \textbf{Frequent topology changes:} As vehicles join and leave roadways,  topology changes occur \cite{rehman2013vehicular}.
 \item \textbf{Highly variable network density:} The number of vehicles varies widely by location, time of day and events \cite{lee2021vanet}.
 \item \textbf{Distributed Decision Making:} Locally informed decisions are made by each vehicle without global network awareness \cite{raja2010issues}.
\end{itemize}

\begin{figure}[H]
 \centering
 \includegraphics[width=0.85\textwidth]{diagrams/2_VANET.png}
 \caption{Architecture of Vehicular Ad,  Hoc Networks (VANETs) \cite{rehman2013vehicular}}
 \label{fig: vanet}
\end{figure}
\subsection{Software,  Defined Vehicular Networks (SDVN)}
Software,  Defined Vehicular Networks (SDVN) hybridise SDN and VANET as a programmable, centrally controlled architecture with mobile communication ability \cite{correia2017architecture}. SDVN overcomes’ VANET‘s inherent limitations, , especially distributed routing within rapidly changing topologies,,  by centralising network intelligence whilst retaining vehicle,  to,  vehicle messaging \cite{li2016control}.

SDVN architecture:
\begin{itemize}
	\item \textbf{SDN, PCV}:  Vehicles act as SDN switches with controllers providing forwarding guidance \cite{li2016control}
	\item \textbf{SDN, enabled RSUs}: Packet forwarding and communication with controllers \cite{li2016control}
	\item \textbf{Hierarchical Control}:  Local controllers with regional routing, globally networked \cite{correia2017architecture}
	\item \textbf{Hybrid Control}: Urgent safety messages require direct V2V messaging;  other traffic travels via SDN \cite{dhawankar2017software}
\end{itemize}

\begin{figure}[H]
    \centering
    \includegraphics[width=0.85\textwidth]{diagrams/3_SDVN.png}
    \caption{Architecture of Software-Defined Vehicular Networks (SDVN) \cite{hama2025security}}
    \label{fig:sdvn}
\end{figure}

\subsection{Security Challenges in SDVN}
SDVNs inherit all security issues present in SDN and VANETs. In addition,  they can be targeted by new attack points due to centralized control plane and highly dynamic environment. SDVN networks offer flexible routing and enhanced network connectivity, however the mobile data,  plane entities and centralized logic in an SDVN system are tightly coupled and thus attack effects could be magnified if security assumptions are broken \cite{arif2020sdn}.

Achieving robust decision making under highly dynamic network environment remains a key security challenge in SDVN.  When the network topology changes rapidly,  links frequently go up and down, and the network density is unknown or variable,  it is challenging to accurately and reliably gather most up, to, date state information and report that to the control plan \cite{arif2020sdn}. Therefore, control,  plane decisions are rendered to potentially stale or inaccurate state information,  thereby diminishes traditional network security validation.

Another security concern arises from using centralized or hierarchical controllers \cite{hama2025security}. Controllers collect varying network information, including the total system state information,  and carry out network,  wide enforcing rules, so that they can be targeted by an adversary. So,  any failure or attack against the control plane can quickly affect the rest of the network through routing, quality of service, and system stability degradation \cite{hama2025security}.

Furthermore,  security in SDVN should be achieved without compromising privacy. Vehicles reveal a high rate of private and sensitive information on their state and environment to other traffic participants,  however revealing too much private information would breach privacy and legal policies. Existing security solutions have difficulties achieving data integrity, system resiliency and privacy in big vehicular networks, thus unresolved issues remain in securing SDVN operation in a trustworthy fashion,  as proposed in the sequel.

\subsection{False accusation attacks in SDVN}
False accusation attacks in SDVN refer to the malicious vehicles, RSUs or even compromised controller injecting false or misleading reports against legitimate nodes in order to influence trust and reputation system. The centralized control plane in SDVN broadcasts network, wide info based on feedback from data, plane, false accusation can work against the legitimate network entities to affect the viability of traffic routing and service allocation and consequently reduce the performance and network stability. The rapid network changing topology, ultra high mobility and demanding control decisions on timely manner can make it more vulnerable to false accusations in SDVN.

\subsection{Limitations of Existing Works}
Current security and trust management frameworks for SDVN have some common weaknesses in combating false accusing attacks. Most existing solutions consider only simplified, naive attack behaviors and ignore the advanced multi,  variant,  coordinated and dynamic aggression strategies in authentic SDVN scenarios \cite{lee2012efficient}.  Moreover,  heavily revealing behavioral evidence for validation not only breaches privacy \cite{ullah2025decentralized},  but also makes reputation store vulnerable to exploits and malicious modifications. Centralized or mutable reputation store also poses another practical barrier to real, world SDVN systems due to a single point of failure and tampering.  Besides,  the rigidity of traditional rule, statistic driven detection method is unable to build relational/temporal models of accusation dynamism and cooperation,  and leads to unscalable and unreliability in open SDVN environments \cite{cardona2020software}.

\section{Problem Statement}
Trust, security and reputation schemes in SDVN are ineffective against sophisticated false accusing attacks under realistic, open SDVN settings, due to high mobility,  self,  interests,  deception techniques and the striful privacy requirements.

\section{Objectives and Scope}
\subsection{Objectives}
This research mainly aims to build up an effective and adaptive defense system to the false accusation attacks, especially in SDVNs.

\singlespacing

The subtler objectives are:
\begin{enumerate}
	\item study the effects of false accusation attacks on trust and reputation system under high mobility context in SDVNs, and describe the realistic scenario and characterize the attack flow pattern of data,  plane (vehicle or RSU) and control,  plane (controller) attacks;
	\item develop a tamper resistant and private provenance reputation management system, including blockchain timestamped logging, Zero,  Knowledge Proofs for accusation authentication and Graph Neural Network,  based anomaly detector;
	\item evaluate this fraud detection system using simulation experiment including NS, 3 and Hyperledger Fabric, with accuracy of attack detection, false identity attacks,  system performance and QoS.
\end{enumerate}

\subsection{Scope}
In order to keep the research clear and feasible, the investigation is focused on:
\begin{itemize}
	\item false accusation attack to the trust and reputation system in SDVNs;
	\item considering the both data,  plane adversaries like malicious vehicles/RSUs and compromised RSUs, as well as the control,  plane adversaries like malicious controllers or compromised controllers;
	\item solution based on blockchain, Zero Knowledge Proof,  GNN and Large Language Model;
	\item using simulation rather than real field test to evaluate the attack detection accuracy, reputation stability, routing efficiency, QoS performance, etc.;
	\item excluding the physical,  layer attack,  traditional cryptography secure key management mechanism and non,  reputation attack which not manipulation the accusations.
\end{itemize}
This scope set up the goal obviously to focus the research on building a resilient, privacy,  preserving,  sophisticated and intelligent defense approach against false accusing attack, and relevant to the real,  world deployments.

\subsection{Methodological Contributions}
The significant methodological contributions of this study include:
\begin{itemize}
	\item \textbf{Multi,  Layered Defense Scheme:} A SDVN defense framework with combination of blockchain, ZKPs and AI,  based learning models to prevent false accusation attacks on both data plane and control plane entities.
	\item \textbf{Decentralized Reputation Management:} Blockchain,  based reputation scheme with all, recorded immutable logs and a weighted endorser trust score,  which prevents single authorities or Sybil, time amplification from manipulating the blacklist decision.
	\item \textbf{Privacy, Preservation of Accusation Proving:} ZKP scheme that the vehicle and RSUs only prove accusations’ validity without revealing sensitive information such as identities or external conditions.
	\item \textbf{Graph Neural Network,  Based Anomaly Detection:} GNN models to leverage the spatio,  temporal and relational coherence of coordinated attack behaviors in highly dynamic SDVN topologies.
	\item \textbf{LLM, Enabled Inferencing of Attacking Intents:} LLM, enabled semantic reasoning layer on the unstructured controller logs and network events to synthesize finer, grained interpretation of attack incidents and provide explainable incident reports to LMs. \item \textbf{Automated Controller Trust Evaluation:} Smart contract, based controller trust that automatically makes failover decision when SDN controllers are compromised.
\end{itemize}

\subsection{Implication of the research}
The work offers important implications for designing secure and trusted SDVN infrastructures.  It guarantees effective false, accusations mitigation while maintaining privacy and network performance by using SP within trust,  based routing, usage, based policy management, white, list marking, and convoy formation. Built on the explainable AI elements and platformed on SDVN infrastructure,  the framework supplies high operability transparency and administrative control for practical deployment. For other cyber,  physical and software,  defined systems based on distributed trust,  the proposed approach also applies,  such as in smart transportation, industrial internet of things, and intelligent edge net.

\subsection{Novelty of the Proposed Work}
The novelty of this work is in the problem modeling and attack modeling,  rather than in the individual techniques used.  This work formulates false accusation attacks in SDVNs as a multi,  variant, cross,  layer trust attack with both data plane and control plane perspectives in realistic scenarios such high mobility and privacy, preserving.  Instead of the stereotypical setting where false accusations are independent events, the proposed model accounts for their temporal, relational, and behavioraldependencies such as coordinated attack and timing,  based attack strategies. Based on the formulation,  this work then proposes a comprehensive defense framework where the accusation validation and trust evolution are simultaneously regulated by cryptographic consistency, historical trust behavior, and network,  wide views. Blockchain, Zero,  Knowledge Proof, and GNN,  LLM are only used as means to the end, i. e.,  the real problem modeled here, rather than as another piece of fragmented or assumptive defenses in the literature.

\subsection{Organization of the Paper}
The rest of the paper is structured as follows.  In Chapter 2,  a wide coverage of work related to trust management and false accusation attacks in SDVNs is presented,  showing gap areas.  In Chapter 3,  the system model, threat model, and the multi,  layered defense framework are described in detail.  The project timetable and requirements for implementation and evaluation are provided in Chapter 4. Finally, findings are summarized in Chapter 5 and the potential impact of this approach on trustworthy deployment of SDVNs is discussed.

\chapter{Literature Review}
\section{Previous Works}

\subsection{Software-Defined Vehicular Networks (SDVN)}
The Software, Defined Vehicular Networks (SDVN) were proposed by applying SDN architecture concepts such as centralized control via separation of control,  plane from data,  plane,  higher programmability and traffic engineering to vehicular network scenarios where dynamic traffic between geographically distributed nodes, such as handovers, congestion, and strict latency requirements, are of utmost importance to secure network trust, routing efficiency and network stability \cite{cardona2020software}.  The architecture of SDVN is built on several fundamental components such as On,  Board Units (OBU) to support inter,  vehicular communication,  Road Side Units (RSU) acting as gateway between vehicles and infrastructure, and Trusted Authorities (TA) in charge of establishing trust relationships,  where control plane is layered in the same way as SDN with hierarchical SDN control plane where central controllers have global command of the network that can secure traffic flow control,  coordinated security policy enforcement, and access control management \cite{adnan2021towards}.  This single point of control however also presents certain vulnerabilities that can be exploited to manipulate reputation framework and trust systems in SDVN where successfully compromised control plane or RSU can influence entire network segments, with introduction of security concerns involving control plane communications, SDN controller digital identity impersonation attacks, and RSU corruption affecting the entire network,  leading to potential selection of reputation based false accusation attack by manipulating reputation management system where malicious nodes can coerce honest nodes to be falsely accused, direct all traffic towards attack nodes while blacklisting legitimate pathways,  significantly impair QoS with high packet loss while elevating communication delays, and destabilize trust system by proliferation of unstable trust scores \cite{adnan2021towards}.  Corresponding research gaps and defenses are identified with solutions combining decentralized verification with blockchain, based delivery of tamper,  proof reputation records, Zero Knowledge Proofs to enable validation without revealing additional information, and AI, enhanced anomaly detection to counter reputation based exploits such as single attacker opportunistic false accusations,  colluding crowdsourcing election,  or sustained timing manipulation during high,  rumor phases, Evidence falsification, and RSU/Sensor hardware co, opted collusions \cite{cardona2020software}.

\subsection{False Accusation Attacks in Vehicular Networks}
False accusation attacks are a critical vulnerability in vehicular networks where malicious nodes spread false alert messages accusing normal nodes as misbehaving nodes, with simulation showing accelerated throughput decrease and more severe network degradation than selfish nodes as false accusation nodes restrict legitimate nodes from routing,  thus contributing to efficiency and survivability degradation of networks \cite{lee2012efficient}.  Its core attack source involves malicious nodes on the forwarding path of the message falsely claiming that normal nodes have failed to deliver their data packets while forwarding packets,  which causes source nodes to falsely consider all normal nodes as misbehavers upon deletion of acknowledgment messages,  hence excluding legitimate nodes from the network, by reputational smearing in distributed cooperative intrusion detection system especially when nodes colluding or fake reporting of reputational values by compromised nodes, on account of diverse attack variants such as single,  accuser opportunistic spurious reports, Sybil, simulated votes to increase sequential falsity,  timed false accusations at high,  noise intervals which are indistinguishable from real error reports, evidence falsifying with forged logs and traceroute simulations,  GPS false routes with fabricated reporting, trust, disturbance attacks, and collusions with compromised gate,  kept infrastructure \cite{gyawali2020machine}, thereby resulting in malicious path removal with bad nodes being blacklisted, relocation of traffic load to attacker‘s nodes, Quality of Service degradation,  and instability of control,  plane with fluctuating trust scores \cite{che2022trust}.  Research complications still exists in devising robust countermeasures with decentralized validation via blockchain, enabled evidence, Zero,  Knowledge proof for witnesses to protect sensitive vehicular data, and machine learning empowered analysis of network topologies by Graph neural network, led anomaly detection models to find suspect clusters of lies in conjunction with Large Language Models for situational awareness and counterword generation to effectively nullify these emerging false accusation threats in Software,  Defined Vehicular Networks.

\subsection{Blockchain Technology in Vehicular Networks}
For trust management, blockchain has been considered as a promising solution for vehicular networks. Built upon distributed ledgers,  blockchain provides an immutable support for exchanging and storing reputation scores, trust evidence and behavior traces in a public and verifiable manner by using decentralized consensus,  and thus overcoming single point of failure and performance bottleneck encountered in centralized trust management approaches[1]. A distributed reputation management over consortium blockchain is realized by adopting three novel phases for global trust computation:  message verification and local trust metric calculation by vehicles, global trust computation by miners or RSUs and consensual addition of reputation throughput blocks to the tamper,  proof blockchain and dispute arbitration mechanisms for tracking malicious vehicles[2]. Existing schemes suffer from critical limitations:  trust evaluation based only on geographical neighbor that is insufficient to distinguish false accusations from malicious vehicles in the advantageous position;  lack of verification mechanisms of whether a node‘s accusation is genuine and whether the behavior of a node is malicious without yielding the underlying contents; unawareness to detect Sybil may network attack by suffusing believable fake identities to unknowing vehicles,  or collusion attack involving with compromised infrastructure components to perform consensus attacks,  or critical lack of intelligent analysis scheme to grasp malicious accusation patterns based on network topology or utter logs and behavioral evidence contexts [3]. Missing yet are essential researches on verifiable trust computing framework that employs privacy, saving Zero,  Knowledge Proof for accusation verification without revealing sensitive vehicular contents, intelligent anomaly detection scheme based on GNN for network topology graph analysis to determine malicious collusion and anomalous collection of accusation evidences,  LLM, based contextual contents analysis framework to incorporate graphs and undeniable logs for potential confrontation against single,  accuser, Timing, based, Sybil, evidence,  spoofing and collusion attacKs in SDVNRs with real, time efficiency and effective game mitigation tool for maximized trustworthy routing,  credibility and stability in verifiable malicious accusation environment.

\subsection{Zero-Knowledge Proofs (ZKP)}
Zero, Knowledge Proof (ZKP) is also a zero, knowledge proof system used in cryptography to prove of proof of the truth of knowledge statement without providing any knowledge as to the content of the statement,  other than its truth value (or interval of truth) and justifications of such, which makes it useful for privacy and security preservation applications.  As ZKP can help ascertain the validity of information without relinquishing privacy, it also helps in privacy and security preservation applications in vehicular networks. Vehicles can adopt ZKP to demonstrate attributes such as identity, privilege, or possession of some information without exposing too much sensitive data and compromising privacy and privacy preservation applications ZKP can be used in three specific applications: 1) privacy preservation authentication, 2) privacy preservation location, or information on zero, knowledge range proofs of vehicle locations, 3) privacy preservation anonymous credential verification for accountability systems without requiring any third parties, as explored in \cite{zhou2024leveraging,  kalmykov2022using}.  If ZKP is promising, the application of ZKP and trusted third, party verifiers for fairly validating accusation proofs in reputation, based trust management systems remain problematic;  ZKP based approaches that have been proposed so far, can be categorized under authentication of information and credential verification applications, but do not present the solutions for node behavior and accusation verification in privacy preserving manner without revealing sensitive vehicular data, \cite{kalmykov2022using}.  Very few papers used ZKP for presentation and argumentation that proved the accused node forwarded packets in a way that would otherwise be visible to the network,  proved the veracity of the accusation, or proved the negative impact of the accusation, proved that the accused had all the right information without revealing the entire information, proved that behavioral expectations were satisfied while addressing privacy issues, proved that reputation was calculated accurately without disclosing the value, addressed a number of false,  accusation attack signatures in context of false, message attack, integrated with compromised infrastructure, double, acuser, elite, rumor, attack, and injected attack, but not in context of defender,  stable, trues,  learning, and specific attack (falsified,  fabricated,  aggregated, flooding,  in,  time,  accurate,  silent,  audio, evidence,  spurious, attack, by colluding, traffic, injection,  attack,  using compromised infrastructure, and double,  acuser, elite,  attack). In addition, the verifiability of reputations,  trust optimization, and privacy preserved ZKP for defense against false accusation attack, in dynamic SVVNs has been rarely explored.

\subsection{Artificial Intelligence in Vehicular Networks}
GNNs have been identified as an innovative technology for unraveling intricate network configurations within Software,  Defined Vehicular Networks by means of message passing strategies that assimilate neighboring node information for deciphering relational trends, thereby facilitating the detection of anomalous accusation clusters that deviate from standard network behavior, GNN,  based solutions trained in a self, supervised manner exhibiting heightened network intrusion detection effectiveness, aiding in network topology investigation and edge feature examination to discern cohesive malicious conduct including Sybil attacks,  collusion,  tampering,  exaggeration of reputation targets,  where several nodes forge unfounded accusations, and other deception tactics such as using bogus network traces to perpetuate harmful fabrications by LLMs, by exploiting linguistic patterns in accusing messages, authenticating hypothesis truthfulness, and deploying automated response creation \cite{zoubir2024integrating, belcastro2025enhancing}.  Prominent LLMs like BERT,  based models and transformer architectures have shown exceptional performance in cybersecurity through profound contextual comprehension and natural language processing of diverse security data such as system logs, network flows, accusation messages, and threat analysis reports, with promising applications in examining accusation message semantics, claim validity verification,  detecting linguistic markers of malicious reports, and synthesizing automated reactions \cite{belcastro2025enhancing}.  An integrated approach utilizing the detection capabilities of GNNs with the contextual analysis capabilities of LLMs holds potential for effective False Accusation attack detection in vehicular networks.  GNNs can identify anomalous accusation structures,  relational patterns, and aggressive attack signatures like sudden accusation clusters or multi, node collusions, while LLMs can analyze graph,  derived data with ancillary natural language logs for contextual threat evaluation, automated response synthesis, and reasoning about complex assaults with fabricated evidence,  manipulated communication data, high,  noise, infiltration timing, and evidence,  spoofing tactics like traceroute mimicry targeted at attacking actors \cite{chattopadhyay2024gnn}.  However, still unmet challenges include evolving integrated GNN,  LLM solutions that synergize network topology insight for detecting nuanced false accusations, such as isolated,  accuser exploitation and Sybil, enhanced consensus, spamming, with capabilities for rapid real, time operation to satisfy safety needs,  secure user privacy via fusion with Zero,  Knowledge Proofs, and establish unstoppable trust with integration with blockchain for indelible evidence record, keeping in distributed Software,  Defined Vehicular Networks emphasizing route optimization and Quality of Service assurance \cite{belcastro2025enhancing}.

\section{Gaps in Literature}
\subsection{Limited Blockchain and AI Integration for SDVN }
Existing studies either investigate blockchain,  based trust management or AI,  based intrusion detection for vehicle networks.  Blockchains enable record tamper resistance for reputations and decentralized consensus \cite{gazdar2022decentralized}, while Machine Learning, based AI enables anomaly detection to identify case, specific random behaviors \cite{chattopadhyay2024gnn}.  Unfortunately,  current frameworks of either one alone are inadequate for decanting compromised vehicles for false accusation defenses.  Limited research endeavors involve at least a Graph Neural Network (GNN) for topology analysis to discover abnormal accusation group formation trend \cite{zoubir2024integrating} and at least a Large Language Model (LLM) for semantics reasoning of attack evidence and reports onto blockchain, establishing a verifiable reputation trust basis.  Notably, solutions to incorporate weighted endorser trust logic with AI, powered analysis of vehicle behavior pattern deviations to prevent malicious framing and AI, based discriminator between genuine and malevolent group‘s false accusations \cite{che2022trust} are missing. A multi, layer architecture to combine blockchain decentralized reputation record, GNN accusation structure anomaly detection, and LLM semantic attack report analysis will form a robust false accusation defense platform against multi, lateral false accusation types such as Sybil, based consensus flooding \cite{jaballah1904software} and noise, resilient causality, based false report generation in high velocity SDVN.

\subsection{Insufficient Privacy-Preserving Verification Mechanisms for Reputation Systems}
Existing reputation,  based trust management mechanisms for SDVN do not employ any privacy, preserving techniques to convince others of an honest node’s reputation score, and instead require the collection and sharing of information such as packets forwarded,  messages sent, and routing states for other nodes to verify the truth of any expressed accusations \cite{gazdar2022decentralized}. Zero,  Knowledge Proof (ZKP) applications have been developed for use in vehicular networks for verifying identities and credentials \cite{lavin2024survey} but have yet to be utilized within reputation systems to prove the accuracy of accusations without restoring the sensitive information withheld during the proof generation \cite{kalmykov2022using}.  None of the current literature describe ZKP schemes for demonstrating that a node consistently forwarded packets correctly without revealing paths,  or that an attacker’s accomplice has evidence of an honest message in their communications logs, or that reputation scores derive from proper trust evaluations without revealing sensitive details. Such schemes would help fill the security/privacy gap experienced by current SDVN reputation mechanisms where system designers are forced to sacrifice either one for the other \cite{kalmykov2022using}.  There is a pressing need to explore the use of Zero,  Knowledge Proofs with decentralized blockchain,  based reputation systems \cite{lavin2024survey} so that SDA vehicles would be able to verify nodes are honest without exchanging location,  velocity,  and destination information that adversaries could leverage for false attacks.

\subsection{Lack of Intelligent Detection for Coordinated False Accusation Attacks}
Existing misbehavior detection schemes deployed in SDVN use techniques such as statistical anomaly detection, threshold,  based filtering, and historical pattern matching that successfully detect simple, isolated packet dropping or naked false reporting (i. e., in the absence of coordinated collusion),  but are ineffective against more sophisticated multi, faceted collaboratively coordinated attacks \cite{gyawali2020machine}.  Conventional machine learning detection schemes that utilize CNNs and RNNs that analyze feature vectors derived from individual node behaviors have no means to model the highly complex relational dependencies in dynamic vehicular network topologies where trust relationships and accusation flows occur across multiple agencies to form broad interdependent clusters \cite{chang2025blockchain}.  The threats of interest require new solutions to be able to detect Sybil, boosted consensus flooding that occurs when a fleet of fake vehicles produce a massive,  conspicuous consensus and timing,  based false indications of network impairment that occur when collaborative malicious actors time their report transmissions to coincide with periods of high network congestion in order to obfuscate malicious reporting in the noise,  as well as swarms of colluding malicious RSUs and controllers producing false accusations just using conventional, standard models of anomaly detection \cite{ahmed2022privacy}.  Few detection models or frameworks have used GNNs to model a vehicular network as a graph in order to identify topological anomalies such as star, shaped accusation graphs, suspicious community clusters, or rampant community, level attack behavior against valid vehicles. There is great potential in designing graph neural network frameworks that can identify suspicious network topology patterns and detect anomalous clusters of accusations against valid vehicles to alert on colludal attacker teams,  as well as leverage attention mechanisms between message, passing iterations to filter out malicious timing,  based allegations,  during attack events occurring with high levels of network noise.

\subsection{Insufficient Immutable Evidence Logging and Historical Verification}
Current vehicular network security systems don‘t offer reliable chains, of, custody of accusation evidence,  so malicious nodes can create and alter packet logs, routing path traces and sensor measurements without notice \cite{chang2025blockchain}.  The existing vehicular security systems don‘t keep cryptographically verifiable timestamps alleging when behavioral evidence was captured and logged,  so malicious nodes can later generate fake evidence replicating the past behaviors of honest nodes to frame them \cite{ahmed2022privacy}.  Existing systems lack blockchain,  based evidence integrity protection by storing the original packet hashes in an immutable proof, of,  capture event \cite{gazdar2022decentralized},  so malicious nodes cannot forge backward, looking false accusations.  Existing systems lack forensic audit trail recording who accused who,  who provided what evidence,  which endorsers cast votes and how their reputation scores changed with cryptographically unforgeable signatures \cite{gazdar2022decentralized},  which diminishes the possibility of further uncaught frauds. These limitations can be overcome by creating a stable blockchain evidence management system maintaining immutability and traceability of evidence,  keeping unforgeable timestamps of capture events within reasonable timing windows \cite{ahmed2022privacy},  keeping a full forensic audit trail of accusation events, and providing automatic trust weight reduction of malicious nodes.

\section{Table of Literature Review Summary}

\begingroup
\renewcommand{\arraystretch}{1.15}
\setlength{\tabcolsep}{5pt}
\begin{longtable}{|p{3cm}|p{6.5cm}|p{6.5cm}|}
\caption{Comparison of prior work limitations and proposed contributions for SDVN false-accusation mitigation.}
\label{tab:false_accusation_detection} \\
\hline
\textbf{Category} &
\textbf{Limitation} &
\textbf{Proposed solution} \\
\hline
\endfirsthead

\multicolumn{3}{c}%
{{\bfseries \tablename\ \thetable{} -- continued}} \\
\hline
\textbf{Category} &
\textbf{Limitation vs. SDVN false-accusation mitigation} &
\textbf{Proposed contribution} \\
\hline
\endhead

\endfoot

\hline
\endlastfoot

False accusation / reputation poisoning &
No tamper-proof chain-of-custody; weak against Sybil-amplified flooding, timing-noise exploitation, and forged evidence; no privacy-preserving verification (\cite{lee2012efficient}) &
Blockchain-backed immutable evidence logging with ZKP-based verification and weighted endorser trust scoring to prevent unilateral false accusations \\
\hline

ML-assisted misbehavior detection &
Can be evaded by stealthy attackers below thresholds; limited integrity assurance for logs/evidence; not SDVN control-plane aware (\cite{gyawali2020machine}) &
GNN-based relational anomaly detection plus LLM-assisted log reasoning, anchored to blockchain-verified evidence and controller-aware mitigation \\
\hline

Trust distortion / Sybil-related manipulation &
Does not provide cryptographic proof of accusation authenticity; limited detection of coordinated accusation graphs (\cite{che2022trust}) &
Sybil-resistant registration, weighted endorser consensus, and GNN detection of coordinated accusation structures \\
\hline

Blockchain trust / reputation &
Immutability alone cannot stop false accusations; limited Sybil resistance; weak intelligent detection for coordinated accusation campaigns (\cite{li2020blockchain}) &
Blockchain combined with ZKP-based accusation validation and AI-driven detection of coordinated false-accusation patterns \\
\hline

Decentralized trust aggregation (blockchain) &
Heuristic/proximity-based trust exploitable; lacks privacy-preserving evidence validation; limited resilience to coordinated flooding/collusion (\cite{gazdar2022decentralized}) &
Privacy-preserving evidence validation via ZKPs, multi-endorser consensus, and anomaly detection to resist coordinated flooding/collusion \\
\hline

Blockchain + learning-assisted trust &
Not tailored to explicit false-accusation variants (timing-based accusations, forged logs, controller fabrication); potential overhead without end-to-end verification (\cite{chang2025blockchain}) &
Explicit attack-variant modeling with end-to-end verification, combining blockchain, ZKP proofs, GNN detection, and LLM reasoning \\
\hline

GNN-based anomaly detection &
Typically detection-only unless integrated with verifiable reputation evidence; no direct accusation authenticity mechanism (\cite{zoubir2024integrating}) &
GNN detections tied to blockchain-validated evidence and smart-contract enforcement for accusation authenticity \\
\hline

GNN security analytics &
Does not ensure evidence integrity/privacy; mitigation not explicitly tied to accusation workflows and trust updates (\cite{chattopadhyay2024gnn}) &
Integrated trust workflow linking GNN outputs to ZKP-based validation, reputation updates, and enforcement actions \\
\hline

Advanced DL for security analytics &
Not accusation-specific; lacks blockchain/ZKP-style verifiability for trust decisions under false accusations (\cite{belcastro2025enhancing}) &
LLM-assisted contextual analysis of logs with blockchain-trusted state and ZKP verification for explainable decisions \\
\hline
\end{longtable}
\endgroup


\chapter{Methodology}
\section{Research Design}

\begin{figure}[H]
	\centering
	\includegraphics[width=\linewidth]{diagrams/FYP_Architecture_small.png}
	\caption{Integrated High‑Level Architecture of the Proposed SDVN Framework.}
	\label{fig:arch}
\end{figure}

Figure \ref{fig:arch} illustrates the integrated high-level architecture of the proposed SDVN framework.
As shown in Figure \ref{fig:arch}, the AI analysis layer sits at the top and acts as the intelligence and reporting component.
It consumes two main categories of inputs: (i) the current network state (e.g., node status and trust values)
and (ii) continuous logs and events generated from RSUs and controllers (e.g., warnings, repeated anomalies,
and observed misbehavior). A graph-based model processes the network as connected entities (vehicles, RSUs,
and controllers) to identify suspicious patterns from relationships and traffic behavior. Its output is then
passed to a language-based reasoning component that produces a concise, human-understandable explanation of
what is happening, why it is suspicious, and what response is recommended. The final output of this layer is
an attack detection decision plus a response report that is forwarded to the enforcement path.

The blockchain layer is created using \textbf{Hyperledger Fabric} to provide a shared, tamper-proof audit
log of security-relevant updates across all controllers. This layer contains three smart contract modules
(SC1--SC3) that handle different classes of trust decisions:
\begin{itemize}
    \item \textbf{SC1: Registration and Authentication} -- holds the status of joined/participating nodes and
    whether a node is allowed to operate in the network.
    \item \textbf{SC2: Reputation Management} -- stores scores acquired from different events and updates
    based on confirmed evidence, so no single controller can tamper with trust values.
    \item \textbf{SC3: Controller Trust} -- stores controller trust values to ensure safe operation if a
    controller becomes unreliable or compromised.
\end{itemize}
Using Fabric, controllers can submit updates (e.g., a reputation update or a controller trust update) to the
blockchain, which validates the update through endorsement and ordering before committing it to the ledger.
This ensures that trust decisions are consistent, correct, and auditable.

The control plane comprises a set of SDN controllers (Controller 1--3) that manage their respective RSU
regions. Each controller:
\begin{itemize}
    \item collects measurements and alerts from RSUs,
    \item produces structured logs/events for security evaluation,
    \item exchanges coordination messages with peer controllers through east--west communication to keep
    policies consistent,
    \item queries the Fabric-maintained state (participation, reputation, controller trust) before applying
    sensitive actions,
    \item enforces response actions (e.g., isolation, access restriction, forwarding adjustments) back toward RSUs.
\end{itemize}
In this design, controllers are both \textit{contributors} (they submit events/updates to Fabric) and
\textit{consumers} (they read the trusted state from Fabric to guide enforcement decisions).

The data plane contains vehicles and RSUs where actual wireless communication occurs.
Vehicles exchange messages directly (V2V) and also interact with nearby RSUs (V2I).
RSUs serve as the operational gateway between vehicles and controllers by:
\begin{itemize}
    \item relaying traffic and connectivity between vehicles and the control plane,
    \item collecting local observations (e.g., abnormal message rates, repeated suspicious behavior),
    \item forwarding summarized observations and alerts to the assigned controller for analysis and logging.
\end{itemize}
After the control plane decides a mitigation action (guided by the trusted Fabric state and AI reports),
the action is pushed down to RSUs, which then apply it to the local vehicular environment (e.g., restricting
a node’s access through network rules or prioritization).


\section{Threat Model}
The adversary model adopts a Zero-Trust Architecture where any network component may potentially be compromised. The threat model defines attacker capabilities, attack scenarios, and security assumptions \cite{adnan2021towards}.

\subsection{Attacker Capabilities}
\textbf{C1 - Data Plane Compromise:} Attackers can compromise up to n-f vehicular nodes and RSUs where n is the total number of nodes. Compromised nodes can participate in routing and reputation voting while executing malicious behaviors such as false accusations or packet manipulation.

\textbf{C2 - Control Plane Compromise:} Adversaries may compromise up to nc-1 SDN controllers out of nc total controllers (e.g., 3 out of 4). The blockchain ordering service tolerates up to $\lfloor(n_o-1)/3\rfloor$ Byzantine faults through BFT-SMaRt consensus, with additional resilience provided by the Controller Trust Evaluation Smart Contract \cite{yahiatene2018blockchain}.

\textbf{C3 - Wireless Communication Attacks:} Attackers can intercept, modify, replay, or flood V2V and V2I wireless transmissions over IEEE 802.11p DSRC channels.

\textbf{C4 - Computational Bounds:} Adversaries are computationally powerful but bounded by classical computing. Post-quantum cryptography (FALCON-1024, Kyber-1024) protects against future quantum threats, while frequent key rotation mitigates classical cryptanalysis \cite{bensasson2018scalable}.

\subsection{Attack Scenarios}
\textbf{Scenario 1 - Single-Accuser Fabrication:} A privileged malicious node with high reputation fabricates false evidence against honest vehicles, leveraging trusted status to cause blacklisting.

\textbf{Scenario 2 - Sybil-Amplified Flooding:} Multiple fake identities coordinate to flood the reputation system with false accusations, creating artificial consensus.

\textbf{Scenario 3 - Timing-Based Accusations:} Attackers exploit legitimate network stress events (congestion, handovers) to mask malicious accusations as genuine failures.

\textbf{Scenario 4 - Evidence-Spoofing:} Malicious vehicles fabricate report logs tied to manipulated packets, invent routes through the network, falsify sensor observations, or claim rule violations to frame honest nodes.

\section{System Assumptions}
The framework operates under the following security and infrastructure assumptions:
\begin{itemize}
    \item \textbf{A1 - Cryptographic Security:} Cryptographic primitives (FALCON-1024, Kyber-1024, RSA-2048, ECDH-256, AES-256, HMAC-SHA-256) are impervious to known attacks when correctly used \cite{bensasson2018scalable}.
    \item \textbf{A2 - Byzantine Fault Tolerance:} BFT-SMaRt consensus supports up to $\lfloor(n_o-1)/3\rfloor$ Byzantine orderers, ensuring safety and liveness \cite{castro1999practical}.
    \item \textbf{A3 - Bootstrap Trust:} A trusted controller exists during initial deployment to organize credentials and start the blockchain.
    \item \textbf{A4 - Physical Layer Availability:} Underlying communication infrastructure (IEEE 802.11p, cellular backhaul) provides sufficient availability for safety-critical applications.
    \item \textbf{A5 - Time Synchronization:} All entities maintain loosely synchronized clocks ($\pm$1 second) via GPS or NTP for timestamp-based replay detection.
    \item \textbf{A6 - Computational Resources:} Vehicles and RSUs possess sufficient processing power (multi-core CPUs, HSMs) for cryptographic operations and ML inference.
    \item \textbf{A7 - Registration Authority:} A trusted registration authority (manufacturer, operator, or government agency) issues initial credentials tied to physical vehicle identities.
    \item \textbf{A8 - Honest Majority:} The majority of network participants are honest, with high-trust endorsers predominantly following protocol specifications.
\end{itemize}


\section{Overview of the False Accusation Attack}
\subsection{Single-Accuser Opportunistic Fabrication (Malicious Vehicle/RSU)}
\begin{figure}[H]
    \centering
    \includegraphics[width=0.7\textwidth]{diagrams/Single_Accuser/Single_Accuser_DV.png}
    \caption{Single-accuser fabrication when a vehicle/RSU is malicious (data-plane).}
    \label{fig:overview_single_accuser_dv}
\end{figure}

Figure~\ref{fig:overview_single_accuser_dv} tells the data-plane story step by step. First, in
\textbf{(step 1) Position of the Attacker}, a malicious vehicle/RSU positions itself near the honest node.
Next, in \textbf{(step 2) Fabrication of the Accusation}, it forges a false accusation packet. The attacker then
uploads this claim in \textbf{(step 3) Reporting to Brain (The Upload)} via the RSU to the controller. In
\textbf{(step 4) The Decision (Processing)}, the controller processes the report and updates the trust decision.
Finally, \textbf{(step 5) The Execution of the Consequence (The Block)} is issued, and the honest node is
blacklisted and isolated.

\subsection{Single-Accuser Opportunistic Fabrication (Malicious Controller)}
\begin{figure}[H]
    \centering
    \includegraphics[width=0.7\textwidth]{diagrams/Single_Accuser/Single_Accuser_CV.png}
    \caption{Single-accuser fabrication when the controller is malicious (control-plane).}
    \label{fig:overview_single_accuser_cv}
\end{figure}

Figure~\ref{fig:overview_single_accuser_cv} narrates the control-plane case where the controller is malicious.
In \textbf{(step 1) Target Identification}, the controller selects an honest vehicle as the victim. It then performs
\textbf{(step 2) Internal Reputation Fabrication} by creating false evidence against that node. Without external checks,
the controller issues \textbf{(step 3) Blacklist Command}. Through \textbf{(step 4) Network-Wide Propagation}, the blacklist is
sent across RSUs and neighboring nodes. The sequence ends in \textbf{(step 5) Complete Isolation}, where the victim is
fully cut off from the network.

\subsection{Sybil-Amplified Consensus Flooding (Malicious Vehicle)}
\begin{figure}[H]
    \centering
    \includegraphics[width=0.7\textwidth]{diagrams/Sybil_Attack/Sybil_Attack_DV.png}
    \caption{Sybil-amplified consensus flooding with attacker-generated Sybils (data-plane).}
    \label{fig:overview_sybil_dv}
\end{figure}

Figure~\ref{fig:overview_sybil_dv} shows a data-plane Sybil attack as a staged narrative. In
\textbf{(step 1) Sybil Generation}, the attacker creates multiple fake identities. Those Sybils then begin
\textbf{(step 2) The Flooding (The Attack)} by sending coordinated false accusation packets toward the RSU.
These accusations are forwarded in \textbf{(step 3) Reporting to Brain (The Upload)} to the controller, which
perceives them as coming from many distinct nodes. The controller performs \textbf{(step 4) The Decision (Processing)},
interpreting the volume as consensus. The story ends at \textbf{(step 5) The Consequence (The Block)}, where the
victim is blacklisted via revocation commands.

\subsection{Sybil-Amplified Consensus Flooding (Malicious Controller)}
\begin{figure}[H]
    \centering
    \includegraphics[width=0.7\textwidth]{diagrams/Sybil_Attack/Sybil_Attack_CV.png}
    \caption{Sybil-amplified consensus flooding when the controller fabricates Sybils (control-plane).}
    \label{fig:overview_sybil_cv}
\end{figure}

Figure~\ref{fig:overview_sybil_cv} presents the control-plane Sybil story where the controller itself
manufactures consensus. In \textbf{(step 1) Create Phantom Sybil IDs}, the controller generates fictitious identities
(S1*, S2*, S3*). It follows with \textbf{(step 2) Insert Fabricated Reports}, injecting false accusations tied to those IDs.
To legitimize the appearance of agreement, the controller performs \textbf{(step 3) Data Retrieval for Trust Evaluation},
then applies \textbf{(step 4) Trust Evaluation (Sybil-amplified consensus)} to justify a downgrade. The sequence ends with
\textbf{(step 5) The Consequence (The Block)} as the victim is blacklisted through controller-issued revocation commands.

\subsection{Timing-Based Accusations During High-Noise Periods (Malicious Controller)}
\begin{figure}[H]
    \centering
    \includegraphics[width=0.7\textwidth]{diagrams/Timing_Based_Accusations/Timing_Based_Accusations_CV.png}
    \caption{Timing-based accusations when the controller is malicious (control-plane).}
    \label{fig:overview_timing_cv}
\end{figure}

Figure~\ref{fig:overview_timing_cv} shows a control-plane timing attack as a sequence. In
\textbf{(step 1) Continuously Monitoring the Logs}, the malicious controller watches for packet-loss events.
When \textbf{(step 2) Legitimate Packet Loss (due to high noise)} occurs, the controller seizes the moment and
performs \textbf{(step 3) Opportunistic Fabrication}, framing the noise-induced loss as misbehavior. It then
executes \textbf{(step 4) Blacklist with Proof to All Nodes and RSU}, broadcasting revocation commands that isolate
the victim.

\subsection{Timing-Based Accusations During High-Noise Periods (Malicious Vehicle, Honest Controller)}
\begin{figure}[H]
    \centering
    \includegraphics[width=0.7\textwidth]{diagrams/Timing_Based_Accusations/Timing_Based_Accusations_DV.png}
    \caption{Timing-based accusations when the attacker is in the data plane (honest controller).}
    \label{fig:overview_timing_dv}
\end{figure}

Figure~\ref{fig:overview_timing_dv} narrates the data-plane timing attack under an honest controller. The attacker
starts by \textbf{(step 1) Sending too many packets during high-noise periods} to induce congestion. This leads to
\textbf{(step 2) Legitimate Packet Loss (due to high noise)} across the links. While loss persists, the attacker keeps
\textbf{(step 3) Checking the Logs} to pinpoint the moment of maximum degradation, then issues a \textbf{(step 4) Fabricated
Accusation (timed to align with high noise)}. The honest controller proceeds with \textbf{(step 5) Controller Validates
Accusation}, and if accepted, it completes \textbf{(step 6) Sending Blacklist with Proof to all Nodes and RSU}, resulting
in the victim’s isolation.

\subsection{Evidence-Spoofing Attack (Malicious Vehicle Collaboration)}
\begin{figure}[H]
    \centering
    \includegraphics[width=0.7\textwidth]{diagrams/Malicious Node Attack - Evidence Spoofing.png}
    \caption{Evidence-spoofing when malicious vehicles collude.}
    \label{fig:overview_evidence_node}
\end{figure}

Figure~\ref{fig:overview_evidence_node} shows how colluding vehicles frame an honest node using tampered evidence.
The story begins with \textbf{(step 1) Vehicle Send Legitimate Traffic}, where normal packets flow toward the honest RSU.
Next, in \textbf{(step 2) Multiple Vehicles Coordinates and Collaboratively Fabricate the Evidence}, the attackers in the
collusion zone craft malicious packets and forward the falsified evidence toward the controller. Meanwhile, the RSU
performs \textbf{(step 3) RSU Sends the Legitimate Traffic to the Controller}, so the controller sees a mix of legitimate
and tampered records. The sequence ends at \textbf{(step 4) Controller Trusting the Seemingly Corroborated Tampered Evidence},
leading to blacklisting of the target victim.

\subsection{Evidence-Spoofing Attack (Malicious RSU)}
\begin{figure}[H]
    \centering
    \includegraphics[width=0.7\textwidth]{diagrams/Malicious RSU Attack - Evidence Spoofing.png}
    \caption{Evidence-spoofing when the RSU is malicious.}
    \label{fig:overview_evidence_rsu}
\end{figure}

Figure~\ref{fig:overview_evidence_rsu} describes the RSU-side spoofing scenario. First, in \textbf{(step 1) Nodes Send
Legitimate Traffic to RSU}, honest vehicles forward normal packets to the RSU. The RSU then executes
\textbf{(step 2) RSU Intercepts, Tampers the Packet, and Tampered Packet Sent to the Controller}, replacing the evidence
before it reaches the controller. Finally, the process concludes with \textbf{(step 3) Victim BlackListed}, as the
controller trusts the spoofed packet trail and isolates the honest node.

\subsection{Evidence-Spoofing Attack (Malicious Controller)}
\begin{figure}[H]
    \centering
    \includegraphics[width=0.7\textwidth]{diagrams/Malicious Controller Attack - Evidence Spoofing.png}
    \caption{Evidence-spoofing when the controller is malicious.}
    \label{fig:overview_evidence_controller}
\end{figure}

Figure~\ref{fig:overview_evidence_controller} shows evidence spoofing from a compromised controller. In
\textbf{(step 1) Both Vehicles Send Legitimate Traffic}, normal packets move through the network. The honest RSU
performs \textbf{(step 2) RSU Forwards Legitimate Packet to Controller}, delivering unmodified evidence. The attack
occurs at \textbf{(step 3) Malicious Controller Itself Tampers The Packet \& Targets Vehicle B Unfairly}, where the
controller alters the evidence and fabricates blame, leading to the victim’s blacklist action.


\section{Proposed mitigation framework}
\subsection{Proposed system architecture}
\begin{figure}[H]
	\centering
	\includegraphics[width=1\textwidth]{diagrams/simple_blockdiagram.png}
	\caption{Simple Block Diagram of the Proposed Mitigation Framework.}
	\label{fig:simple_block}
\end{figure}

As shown in Figure~\ref{fig:simple_block}, the system operates as a closed-loop workflow.
The \textit{Registration and Authentication Module} controls which entities are allowed to participate,
while the \textit{Location Verification and Behavioral Monitoring} component continuously observes node
activity and produces \textit{logs and events}. These logs are forwarded to the \textit{Trust Update Core},
which processes security-relevant evidence and updates three trust functions: \textit{participation status}
(SC1), \textit{reputation updates} (SC2), and \textit{controller trust} (SC3). The validated outputs are
stored at the \textit{shared trusted state/ledger}, which is the most recent recorded trusted network state.
Meanwhile, the \textit{AI Analysis and Report} block reads in the logs/events and the trusted state from the
ledger to highlight the abnormal network state and produce the suggested response. This is followed by the
\textit{Mitigation and Enforcement} block which enforces the response actions through the control/monitoring path,
and feedbacks the results to the trust update process.


\subsubsection{Blockchain-Based Decentralized Trust Management}
\label{subsec:blockchain-trust}


To fundamental our approach, the use of blockchain technology is a way to remove single point of failures in
the centralized SDN architectures \cite{sharma2017distblocknet}. A permissioned blockchain is used as a
distributed database to record authentication requests, decisions related to the security of routing and
reputation updates, in an unaltered way. This ensures that an attacker cannot remove evidence related to
illegal activity or coerce the historical state of the system. The blockchain runs over multiple SDN
controllers, creating a consortium where essential decisions are made by consensus of trusted members
\cite{yahiatene2018blockchain}. Figure \ref{fig:functional_diagram} shows how these blockchain‘s features are
utilized for node admission, registration and authentication, and the way that trust decisions are
implemented.

Three specialized smart contracts govern different security aspects. First, the Registration
and Authentication Smart Contract manages node identity verification using Zero-Knowledge
Proofs (ZKPs), allowing vehicles/RSUs to prove authenticity without revealing sensitive
identifiers \cite{bensasson2018scalable}. As shown in Figure~\ref{fig:functional_diagram}, when a node attempts
to join the network, the workflow branches depending on whether it is a \emph{first-time access}.
For first-time access, a Registration Authority performs out-of-band verification and checks
physical identity attributes (e.g., VIN/serial number). After successful verification, cryptographic
credentials are issued and the node creates its key materials and ZKP-related artifacts before
submitting an initial registration request (left branch of Figure~\ref{fig:functional_diagram}).
For returning nodes (right branch), the node generates a session request, constructs a ZKP of
identity, and sends an authentication request to the controller, which then retrieves the node
record from the blockchain (``Blockchain: Retrieve Node Record'') as illustrated in
Figure~\ref{fig:functional_diagram}. The ZKP validity decision (``ZKP Valid?'') ensures that only
registered entities can proceed without exposing private identity information \cite{bensasson2018scalable}.

Second, the Reputation Management Smart Contract uses a weighted endorser trust scoring scheme to
minimize the impact of false accusations on the blacklisting decision \cite{alshaibani2023blockchain}. In
the flowchart of Figure~\ref{fig:functional_diagram}, this maps into the trust decision step (``Trust Score
Sufficient?’’) following ZKP validation. Instead of relying on one report, reputation updates and
punishment actions are endorsed by multiple endorsers, each of whom‘s vote is weighted according to
past reputation accuracy. New nodes and previously malicious nodes have little impact on endorser trust
voting, obstructing Sybil attackers from quickly undermining reputation scores. Changes require
consensus from multiple high, trust endorsers, often requiring at least 2/3 of total trust vote to sanction
or admit a node \cite{bessani2014state}. If there is not enough trust, the path takes the branch
(``Blacklisted?’’), where blacklisted nodes are rejected and borderline nodes require additional proof
checks before receiving access, as summarized in Figure~\ref{fig:functional_diagram}. This prevents a
node from transitioning from authenticating to active participation without satisfying both crypto
verification and trust policy.

Third, the Controller Trust Evaluation Smart Contract maintains vigilance against controller compromise
by constantly auditing the controller \cite{khan2017topology}. In Figure~\ref{fig:functional_diagram}, the
controller is responsible for authenticating each new login (``Controller Validates’’) and reporting this
to the blockchain to trigger subsequent policy checks. To circumvent a malicious/compromised
controller, a randomized fraction of nodes comment on whether the controller‘s fed topology images
actual packet flows. If inconsistencies are observed, trust score is reduced, and if controller trust drops
below a set minimum, a backup node is automatically designated to preserve the network‘s health
\cite{raja2020energy}.

Finally, Figure~\ref{fig:functional_diagram} shows that once a node has been validated and trusts VOTE,
the system distributes session keys (``Issue Session Keys Encrypt with Kyber, 1024’’) and immutably logs
this event (``Blockchain: Update Authentication Timestamp’’). These serve as a verifiable record of the
join event so any audit trail records who was added, when, and with what trust status. This same
authenticated join event is communicated to the AI learning layer (``GNN: Add Node to Network Graph’),
ensuring deep learning uses only verified blockchain elements.


\begin{figure}[H]
	\centering
	\includegraphics[width=1\linewidth]{diagrams/Functional_Diagram.png}
	\caption{Functional workflow for node registration and authentication with blockchain-backed trust evaluation.}
	\label{fig:functional_diagram}
\end{figure}

\subsubsection{AI-Driven Anomaly Detection}
While the immutable public record ensures no party can alter its records and the consensus mechanism ensures no party can decide which records get added, artificial intelligence adds intelligent pattern recognition which is able to call out coordinated attacks that rule, based systems may miss. The vehicular network is considered a temporal graph in which vehicular nodes and roadside units are vertices connected by communication edges \cite{zhou2020automating}. Graph Neural Networks (GNN) reason over this time, varying topology to detect unusual patterns characteristic of a coordinated attack.

In particular, the GNN detects Sybil attack clusters by finding groups of nodes with suspiciously similar human behavioral patterns, identical registration time stamps, or star, topology accusation domains where multiple accusers target a single victim \cite{balaram2023highly}. Temporal attention methods monitor the trajectories of reputation values, calling out abnormal drops associated with allegation floods. State of the art research finds that heterogeneous graph attention networks deliver over 99\% accuracy at identifying Sybil nodes within vehicular networks \cite{chen2025sybil}.

The large language model (LLM) supplements the GNN by analyzing unstructured data sources from the controller for evidence of attack such as log files, error messages, or system alerts \cite{zhang2024large}. The LLM performs semantic analysis for causal reasoning, such as reading a controller failover about a minute before a reputation spike as indicative of a timing attack. After understanding the who, what, when, and how of an attack, the LLM writes an incident report explaining what was detected, which nodes did it, what confidence levels the model has, and what actions should be taken to counterattack. The report of the integrated system allows network administrator to decide whether to perform a controller failover.

\subsubsection{Cryptographic Defense Mechanisms}
Cryptography protocols are the basis of attack prevention. Digital signatures guarantee that we get reliable evidence,
and nobody could file accusations or packets without being identified throughout any packet or packet attack
\cite{pournaghi2020necppa}. Future quantum computers attacks will use post, quantum cryptography algorithms; they
can prevent such threats by working in long term infrastructures, like vehicular networks. Further attack prevention
are time, stamped authentication tokens, valid only within a time window, so the adversary can‘t replay legitimate
packets to confuse routing procedures \cite{benjaballah2021security}.

Location verification smart contracts check that nodes claiming particular geographic locations are located there,
so that wormhole attacks where malicious nodes far away collude to appear as neighbors are thwarted. Cross checking
with several roadside units validates a location claim, while kinematic constraints (max velocity and acceleration)
detect impossible location changes that reveal GPS spoofing.

\subsubsection{Integrated Defense Workflow}
The entire system, as shown in Figure~\ref{fig:simple_block}, flows in the form of a continous closed,
loop cycle in monitoring, validation, analysis and response. Vehicles authenticate to the
\textit{Registration and Authentication Module} on a regular basis, where legitimacy is checked without
revealing private identity data. In every cycle, nodes pass routing and status information to the RSUs
and controllers and detected malicious activity in the field is recorded by the
\textit{Location Verification and Behavioral Monitoring} block, producing the \textit{Logs and Events} flow
that can be seen in the diagram.

When an accusation is generated, it is submitted through the controller path to the trust update process.
As indicated by the \textit{Trust Update Core} and the Fabric blocks in Figure~\ref{fig:simple_block}, each
accusation is accompanied by verifiable evidence (e.g., signatures, timestamps, and location-related proofs),
which is checked before it contributes to any trust update. Validated updates are then processed by the
Fabric trust functions: participation handling (SC1), reputation updates (SC2), and controller reliability
tracking (SC3). Accepted outcomes are committed to the \textit{Shared Ledger/World State}, ensuring that all
controllers and analysis components query a consistent trusted state.

In parallel, the \textit{AI Analysis Layer} in Figure~\ref{fig:simple_block} continuously consumes both
(i) the \textit{Logs and Events} and (ii) the Fabric-maintained \textit{Shared Ledger/World State}. The graph
analysis component evaluates the evolving network graph to identify suspicious structures and behavior patterns.
When anomalies are detected, the language-based reasoning component combines these findings with system logs to
produce threat assessments and recommended actions.

If both the AI evaluation and the decentralized validation result reach conclusions that indicate the existence of
an attack, then the response is issued from the \textit{Mitigation and Feedback} block (Figure~\ref{fig:simple_block}).
The controllers execute the chosen response back in the field (e. g., isolating or limiting any
malicious hosts), while the final decision and results are stored back into Fabric so future cycles can incorporate
verified history. In parallel, SC3 monotonically observes controller trustworthiness to prevent insider attacks and
facilitate safe controller switching if needed.

This holistic design requires an adversary to defeat all four individual layers of protection, evidence, based
consensus, decentralized validation, trust ranking, and AI, based anomaly detection, thereby significantly raising
the difficulty of successful reputation manipulation and coordinated assault without compromising functionality
\cite{huo2023trustgnn}.


\subsection{Attack, Specific Mitigation Strategies}
Our approach specifically counteracts each of the false accusation attack variants as follows:

\subsection{Single, Accuser Opportunistic Fabrication}
A single malicious node attempts to fabricate convincing evidence against the victimized vehicle. The weighted endorser trust scheme in the Reputation Management Smart Contract can prevent this, by requiring multiple high, trust endorsers to sign off on an accusation \cite{alshaibani2023blockchain}. An individual malicious node can only influence the magnitude of change in the victim‘s reputation, but not induce a blacklist, since the accused victim is only upheld if $\geq$2/3 of overall weighted endorsers agree to the blacklisting. An accused malicious node is counted as an endorser of its own false accusation, although their sigature contributes to the conviction but does not alter the few agents’ initial trust weights.

Furthermore, the accusation validation cryptography prevents accusations without valid signatures, timestamps, or location proofs from being accepted. In addition, the blockchain maintains a traceable recorded log of all accusations with timestamps, so the system can audit nodes that have repeatedly made unfounded accusations and begin to diminish their reputation weights to mean, zero \cite{bessani2014state}.

\subsection{Sybil, Amplified Consensus Flooding}
Sybil attacks are defined by the creation of multiple fictitious identities, used by the attacker to designate a
flood of simultaneous accusations, thus seeking a pseudo, consensus. The Sybil shield they devise consists of 3
levels of security. The first one is identified by the Registration Smart Contract utilizing Zero, Knowledge Proofs
to verify the possession by each node of distinct cryptographic keys related to a physical vehicle
\cite{bensasson2018scalable}. Both implementation approaches of such Sybil schemes are hard in terms of time and
cost; they consist either of stealing valid cryptographic tokens or of registration by an out, of, band (OOB)
communication, both are easily identified.

Secondly, the newcomer nodes initially have very weak trust weights. Even if an attacker is successful in
registration of several Sybil IDs, they carry almost zero influence in reputation decision at the beginning, and it
takes long time for their trust to build up through long, term honest beaconing, preventing quick flooding of
consensus \cite{balaram2023highly}.

Third, the GNN characterizes the accusation topology \cite{chen2025sybil} by detecting the attack pattern of
partially coordinated attack. By watching the attack dynamics via the network topology, the multiple low, trust
nodes simultaneously shoot at one victim, the attack paper pattern of a star topology for accusation is classed as
anomaly. Having one or several lines which share some features such as same registration time, similar behavior
dynamics, and accusation time cluster, Sybil has a high confidence to announce the attack after detection, thereby
entering active investigation and blacklist the attacker group in advance.

 \subsection{Timing, Based Accusations During Elevated Noise Periods}

Advanced attackers take advantage of normal levels of network stress,, congestion, controller handover, channel
deterioration,, to surreptitiously sneak accusations into the system as normal failures. The proposed temporal
attention mechanism neural network is designed to handle this situation by considering timing of accusations during
varying network conditions \cite{zhou2020automating}. This neural network will gather long, term records of normal
packet loss rates across various network conditions (idle, moderate congestion, heavy congestion, handover
periods).

When accusations are encountered during elevated noise periods, the LLM can use semantic reasoning to differentiate
normal failures vs coordinated attack \cite{zhang2024large}. For example, if multiple vehicles near the congestion
area simultaneously report packet loss, then this can be reasoned to also be a result of network congestion.
However, if the first few nodes to send accusations are located within the congestion area but other vehicles in no
congestion subnetworks send normal reports, then this is a timing, based attack.

In addition, the weighted endorser will perform inherently well by design: in the case of true elevated noise
periods, the probability that multiple, independent highly trusted endorsers send reports confirming packet loss is
high. Thus, timing, based accusations will turn out to be outliers and head other attacks in a noisy environment.

 \subsection{Evidence, Spoofing with Tampered Logs and Collusion}

The most advanced attack is to produce falsified, cryptographically, sound evidence this include manipulated packet
logs, forged signatures, or even collution by multiple intruded nodes. The technique relies on multi, layer
cryptographic validation as the main line of defense \cite{pournaghi2020necppa}. To deceive the verifiers, the
evidence must contain: (1) authentic digital signatures filed by known nodes, (2) unforgeable timestamps that do
not deviate from the expected latency window, (3) Zero, Knowledge Proofs signed by the accused node showing
legitimate presence in the network, and (4) verifiable location proofs signed by several roadside units
\cite{quevedo2020intelligent}.

Blockchain chain, of, custody also enhances protection. Evidence must cite traceable hashes of the original packets
recorded on the blockchain at capture time. Retrospective fraud of evidence is infeasible since the blockchain
timestamp reveals the time of storage. Malicious actors cannot manipulate blockchain entries to retrospectively
align evidence with time series data \cite{sharma2017distblocknet}.

To address multiple, colluded compromised nodes, the Controller Trust Evaluation Smart Contract plays a pivotal
role \cite{raja2020energy}. If colluding attackers provide shared falsified evidence, but the verified traffic flow
contradicts controller logics, the cross, endorsement validation uncovers inconsistencies. This identifies the
attacker, coalition. Other than that, cryptography, based protections such as post, quantum secure zero, knowledge
proofs prevent collusions even in the presense of near, omnipotent adversaries \cite{bensasson2018scalable}.

In the event of a smashing compromise of the SDN controller with colluding colluders, the Controller Trust
Evaluation Smart Contract assesses the trust score of the controller by similar random cross, endorsement tests
\cite{khan2017topology}. If the controller attempts to change routing tables and reputation values in order to
favor attacker nodes, the trust score plummets. If it falls below a preset threshold, a new backup controller is
automatically deployed from a different site, eliminating end, line attack.

\section{Experiments/Results.}
The proposed framework will be tested using simulation, based evaluation in which all five variants of false
accusation attack will be simulated at different levels of attack intensity (in other words, different levels of
network penetration,) 0\%, 20\%, 40\%, 60\%, 80\%, 100\%. Performance and effectiveness will be evaluated with
standard vehicular network metrics: Packet Delivery Ratio (ratio of packets successfully delivered), Packet
Interception Ratio (ratio of packets caught), and successful attack detection measures (Matthews Correlation
Coefficient) \cite{nayak2021tbddosa}.

Comparison with other current approaches, including trust, based, cryptographic, and simple blockchain based
approaches, will prove that the approach to be implemented is the most successful in remaining unaffected,
accurately detecting, and excluding malicious nodes from the network. Results will be analyzed for statistical
significance by conducting multiple trials on different network topologies and mobility patterns
\cite{adnan2021towards}.

\chapter{Timeline and Resource Required}

\section{Timeline}
\renewcommand{\arraystretch}{1.2}
\setlength{\tabcolsep}{3pt}

\begin{sidewaystable}[htbp]
\centering
\scriptsize

\caption{Project Timeline}
\resizebox{\textheight}{!}{%
\begin{tabular}{|p{6cm}|*{40}{c|}}
\hline

% ---------------- Year row ----------------
\multirow{2}{*}{\textbf{Task}}
& \multicolumn{8}{c|}{\textbf{2025}}
& \multicolumn{32}{c|}{\textbf{2026}} \\ \cline{2-41}

% ---------------- Month row ----------------
& \multicolumn{4}{c|}{November}
& \multicolumn{4}{c|}{December}
& \multicolumn{4}{c|}{January}
& \multicolumn{4}{c|}{February}
& \multicolumn{4}{c|}{March}
& \multicolumn{4}{c|}{April}
& \multicolumn{4}{c|}{May}
& \multicolumn{4}{c|}{June}
& \multicolumn{4}{c|}{July}
& \multicolumn{4}{c|}{August} \\ \cline{2-41}

% ---------------- Week numbers ----------------
& 1 & 2 & 3 & 4
& 1 & 2 & 3 & 4
& 1 & 2 & 3 & 4
& 1 & 2 & 3 & 4
& 1 & 2 & 3 & 4
& 1 & 2 & 3 & 4
& 1 & 2 & 3 & 4
& 1 & 2 & 3 & 4
& 1 & 2 & 3 & 4
& 1 & 2 & 3 & 4 \\ \hline

% ---------------- Tasks ----------------
Select the project topic
& \cellcolor{red} & & & 
& & & &
& & & &
& & & &
& & & &
& & & &
& & & &
& & & &
& & & &
& & & & \\ \hline

Discuss with supervisors and co-supervisors
& \cellcolor{red} & & & 
& & & &
& & & &
& & & &
& & & &
& & & &
& & & &
& & & &
& & & &
& & & & \\ \hline

Study the related work and implementation
& & \cellcolor{red} & \cellcolor{red} & \cellcolor{red}
& & &  & 
& & & &
& & & &
& & & &
& & & &
& & & &
& & & &
& & & &
& & & & \\ \hline

Prepare the project proposal and presentation
& & & & 
& \cellcolor{red} & \cellcolor{red} & & 
& & & &
& & & &
& & & &
& & & &
& & & &
& & & &
& & & &
& & & & \\ \hline

Study related technologies
& & & & 
& & & \cellcolor{red} & \cellcolor{red}
& \cellcolor{red} & & &
& & & &
& & & &
& & & &
& & & &
& & & &
& & & &
& & & & \\ \hline

Model sample LV network
& & & & 
& & & &
& & \cellcolor{red} & \cellcolor{red} & 
& & & &
& & & &
& & & &
& & & &
& & & &
& & & &
& & & & \\ \hline

Impact assessment for sample network
& & & & 
& & & &
& & & & \cellcolor{red}
& \cellcolor{red} & & &
& & & &
& & & &
& & & &
& & & &
& & & &
& & & & \\ \hline

Collect required actual data from LECO
& & & & 
& & & &
& & & &
& & \cellcolor{red} & & 
& & & &
& & & &
& & & &
& & & &
& & & &
& & & & \\ \hline

Model actual LV network using OpenDSS software
& & & & 
& & & &
& & & &
& & & \cellcolor{red} & \cellcolor{red}
& & & &
& & & & 
& & & &
& & & &
& & & &
& & & & \\ \hline

Model the demand profile
& & & & 
& & & &
& & & &
& & & & 
& \cellcolor{red} & & &
& & & &
& & & &
& & & &
& & & &
& & & & \\ \hline

Impact assessment for increasing PV penetration 
& & & & 
& & & &
& & & &
& & & &
& & \cellcolor{red} & \cellcolor{red} &
& & & &
& & & &
& & & &
& & & &
& & & & \\ \hline

Impact assessment for increasing EVCS penetration
& & & & 
& & & &
& & & &
& & & &
& & & & \cellcolor{red}
& \cellcolor{red} & & &
& & & &
& & & &
& & & &
& & & & \\ \hline

Identifying the optimization techniques
& & & & 
& & & &
& & & &
& & & &
& & & &
& & \cellcolor{red} & \cellcolor{red} &
& & & &
& & & &
& & & &
& & & & \\ \hline

Developing optimum power flow algorithm for day head market
& & & & 
& & & &
& & & &
& & & &
& & & &
& & & & \cellcolor{red}
& \cellcolor{red} & \cellcolor{red} & \cellcolor{red} &
& & & &
& & & &
& & & & \\ \hline

Developing optimum power flow algorithm for intraday market
& & & & 
& & & &
& & & &
& & & &
& & & &
& & & &
& & & & \cellcolor{red}
& \cellcolor{red} & \cellcolor{red} & \cellcolor{red} &
& & & &
& & & & \\ \hline

Integrating Blockchain technology with smart contract
& & & & 
& & & &
& & & &
& & & &
& & & &
& & & &
& & & & 
& & & & \cellcolor{red}
& \cellcolor{red} & \cellcolor{red} & \cellcolor{red} &
& & & & \\ \hline

Testing and validation of developed 
& & & & 
& & & &
& & & &
& & & &
& & & &
& & & &
& & & &
& & & &
& & & & \cellcolor{red}
& \cellcolor{red} & & & \\ \hline

Prepare final report and presentation
& & & & 
& & & &
& & & &
& & & &
& & & &
& & & &
& & & &
& & & &
& & & &
& & \cellcolor{red} & \cellcolor{red} & \cellcolor{red} \cellcolor{red} \\ \hline


\end{tabular}}
\end{sidewaystable}



\section{Resource Required}
For the implementation of the proposed blockchain, ZKP, and GNN-LLM-based defense framework for mitigating false accusation attacks in SDVN, the following tools and technologies are needed:

\begin{itemize}
    \item \textbf{Network Simulator 3 (NS-3):} A discrete-event network simulator for modeling SDVN, simulating V2V and V2I communications, and evaluating the framework under various attack scenarios.
    
    \item \textbf{Hyperledger Fabric:} A permissioned blockchain platform for implementing the distributed ledger, deploying smart contracts for registration, reputation management, and controller trust evaluation with BFT-SMaRt consensus.
    
    \item \textbf{PyTorch or TensorFlow:} Deep learning frameworks for implementing Graph Neural Networks with temporal attention to detect Sybil attacks and coordinated false accusation patterns.
    
    \item \textbf{Hugging Face Transformers:} A library for integrating Large Language Models to process controller logs and system alerts, perform semantic reasoning, and generate automated incident reports.
    
    \item \textbf{libsodium or PyCryptodome:} Cryptographic libraries for implementing post-quantum algorithms (FALCON-1024, Kyber-1024), digital signatures, key exchange, and encryption for securing communications.
    
    \item \textbf{zkSNARK Libraries (libsnark or circom):} Zero-Knowledge Proof frameworks for privacy-preserving authentication, enabling identity verification without revealing sensitive vehicle information.
    
    \item \textbf{Python with NetworkX and Pandas:} Programming environment for graph construction, feature extraction, statistical analysis, and visualization of experimental results.
\end{itemize}

These tools will be employed for the design, implementation, simulation, and evaluation of the proposed defense framework against false accusation attacks in Software-Defined Vehicular Networks.

\chapter{Conclusion}
False accusation attacks pose a severe threat in SDVNs, allowing malicious nodes to directly attack the trust
management system by submitting fabricated accusations. In turn, false accusations result in route depletion,
transient functionality loss, and overall lowered QoS. This research fundamentally pushes the state of the art
forward in defending against false accusations through the first multi, layer framework aimed at combating single,
accuser fabrication, Sybil, amplified flooding, double blaming, and false verdict evidence falsification attacks.

The novel framework proposes a hybridized framework incorporating the decentralized blockchain protocol and token
incentivized registration smart contract, the negation proof enhanced Zero, Knowledge Proof scheme, and the
intelligent GNN, LLM agents, tasked with aversion handling and contextual understanding. Our framework uniquely
combines these state, of, the, art technologies within a Zero, Trust Architecture equipped with specialized smart
contracts that establish the registration and other trust protocols across the multi, user system.

Simulation will then validate using NS, 3 the long, term aggregation and interpretation of these three advanced
technologies in a complex multi, user system across different attack levels. Performance metrics will include
Packet Delivery Ratio, Packet Interception Ratio, and Matthews Correlation Coefficient with results compared to
other trust, based defense methods and simple blockchain, based solutions. To the best of our knowledge, this is
the first paper to establish an integrated framework that leverages all three emerging applied advanced
technological fields towards safeguarding vehicles in SDVNs from false accusation attacks.

% References
\renewcommand{\bibname}{References}
\bibliographystyle{ieeetr}
\addcontentsline{toc}{chapter}{References} % Add to table of contents
\bibliography{bibliography} 
%\printbibliography

\end{document}
