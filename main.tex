\documentclass[12pt, a4paper]{report}
\usepackage{graphicx} % For including images
\usepackage{titlesec} % For customizing section titles
\usepackage{tocloft} % For customizing table of contents
\usepackage{acro} % For acronyms
\usepackage{rotating}
\usepackage{multirow}
\usepackage[table]{xcolor}
\usepackage{array}
\usepackage{tabularx}
\usepackage{longtable}
\usepackage{float}
\usepackage{setspace}


%\usepackage{hyperref} % For clickable links in the document
%% ____Bibliography____%%
\usepackage[numbers,sort&compress]{natbib}
\usepackage{chapterbib}
\usepackage[breaklinks]{hyperref}
%\hypersetup{colorlinks=true,citecolor=blue,linkcolor=blue,urlcolor=blue}
% Page margins
\usepackage[left=1in, right=1in, top=1in, bottom=1in]{geometry}

% Remove page number from the first page
\thispagestyle{empty}

% Customize table of contents, list of figures, and list of tables
\renewcommand{\cfttoctitlefont}{\hfill\Large\bfseries}
\renewcommand{\cftaftertoctitle}{\hfill}
\renewcommand{\cftloftitlefont}{\hfill\Large\bfseries}
\renewcommand{\cftafterloftitle}{\hfill}
\renewcommand{\cftlottitlefont}{\hfill\Large\bfseries}
\renewcommand{\cftafterlottitle}{\hfill}

\DeclareAcronym{SDVN}{
  short = SDVN,
  long  = Software-Defined Vehicular Networks
}
\DeclareAcronym{SDN}{
  short = SDN,
  long  = Software-Defined Networking
}
\DeclareAcronym{VANET}{
  short = VANET,
  long  = Vehicular Ad Hoc Networks
}
\DeclareAcronym{V2V}{
  short = V2V,
  long  = Vehicle-to-Vehicle
}
\DeclareAcronym{V2I}{
  short = V2I,
  long  = Vehicle-to-Infrastructure
}
\DeclareAcronym{V2P}{
  short = V2P,
  long  = Vehicle-to-Pedestrian
}
\DeclareAcronym{V2N}{
  short = V2N,
  long  = Vehicle-to-Network
}
\DeclareAcronym{RSU}{
  short = RSU,
  long  = Roadside Units
}
\DeclareAcronym{OBU}{
  short = OBU,
  long  = On-Board Units
}
\DeclareAcronym{GNN}{
  short = GNN,
  long  = Graph Neural Networks
}
\DeclareAcronym{LSTM}{
  short = LSTM,
  long  = Long Short-Term Memory
}
\DeclareAcronym{ZKP}{
  short = ZKP,
  long  = Zero-Knowledge Proofs
}
\DeclareAcronym{ZKRP}{
  short = ZKRP,
  long  = Zero-Knowledge Range Proofs
}
\DeclareAcronym{BFT}{
  short = BFT,
  long  = Byzantine Fault Tolerance
}
\DeclareAcronym{API}{
  short = API,
  long  = Application Programming Interface
}
\DeclareAcronym{DSRC}{
  short = DSRC,
  long  = Dedicated Short-Range Communications
}
\DeclareAcronym{GPS}{
  short = GPS,
  long  = Global Positioning System
}
\DeclareAcronym{NTP}{
  short = NTP,
  long  = Network Time Protocol
}
\DeclareAcronym{RSA}{
  short = RSA,
  long  = Rivest-Shamir-Adleman
}
\DeclareAcronym{ECDH}{
  short = ECDH,
  long  = Elliptic Curve Diffie-Hellman
}
\DeclareAcronym{AES}{
  short = AES,
  long  = Advanced Encryption Standard
}
\DeclareAcronym{HMAC}{
  short = HMAC,
  long  = Hash-based Message Authentication Code
}
\DeclareAcronym{SHA}{
  short = SHA,
  long  = Secure Hash Algorithm
}
\DeclareAcronym{BERT}{
  short = BERT,
  long  = Bidirectional Encoder Representations from Transformers
}
\DeclareAcronym{QoS}{
  short = QoS,
  long  = Quality of Service
}


\begin{document}


\thispagestyle{empty}

\begin{center}

\begin{center}
 \includegraphics[width=2cm,keepaspectratio=true]{uor_logo.jpg}
 % uor_logo.jpg: 236x331 pixel, 72dpi, 8.33x11.68 cm, bb=0 0 236 331
\end{center}

\vspace{1.5cm}
\begin{huge}
%%%%%%%%%%%%%%%%%%%%%%%%%%%%%%%%%%%%%%%%%%%%%%%%%%%%
% Project title
%%%%%%%%%%%%%%%%%%%%%%%%%%%%%%%%%%%%%%%%%%%%%%%%%%%%
A Blockchain, ZKP, and GNN-LLM-based Combined Defense for False Accusation Attack Mitigation in Software Defined Vehicular Networks
%%%%%%%%%%%%%%%%%%%%%%%%%%%%%%%%%%%%%%%%%%%%%%%%%%%%
\end{huge} \\
\vspace{1cm}

\begin{normalsize}
An undergraduate project proposal report submitted to the
\end{normalsize}\\
\vspace{1cm}

\begin{large}
Department of Electrical and Information Engineering\\
Faculty of Engineering\\
University of Ruhuna\\
Sri Lanka
\end{large}\\

\vspace{1cm}

\begin{normalsize}in partial fulfillment of the requirements for the \end{normalsize}\\
\vspace{1cm}

\begin{large}\textbf{Degree of the Bachelor of the Science of Engineering Honours}\end{large}\\

\vspace{1cm}
\begin{normalsize}by  \end{normalsize}
\vspace{1cm}

\begin{tabular}[h]{lll}
 %%%%%%%%%%%%%%%%%%%%%%%%%%%%%%%%%%%%%%%%%%%%%%%%%%%%%%%%%%
 % Names and Registration Numbers
 %%%%%%%%%%%%%%%%%%%%%%%%%%%%%%%%%%%%%%%%%%%%%%%%%%%%%%%%%%
 M.R.M. Ashfaq	& - & 	EG/2021/4417\\
 T. Jathurshan	& - & 	EG/2021/4568\\
 M.F.A. Munsif	& - & 	EG/2021/4684\\
 M.K.M. Shamil 	& - &	EG/2021/4810
 %%%%%%%%%%%%%%%%%%%%%%%%%%%%%%%%%%%%%%%%%%%%%%%%%%%%%%%%%%
\end{tabular}\\
\vspace{1cm}

20th January 2026\\
\vspace{1cm}

%%%%%%%%%%%%%%%%%%%%%%%%%%%%%%%%%%%%%%%%%%%%%%%%%%%%%%
% if one supervisor
%%%%%%%%%%%%%%%%%%%%%%%%%%%%%%%%%%%%%%%%%%%%%%%%%%%%%%%
%  .............................................. \\
% Prof. A.B.C. Dee\\
% (Supervisor)


%%%%%%%%%%%%%%%%%%%%%%%%%%%%%%%%%%%%%%%%%%%%%%%%%%%%
% If one supervisor
%%%%%%%%%%%%%%%%%%%%%%%%%%%%%%%%%%%%%%%%%%%%%%%%%%%%
.............................................. \\
Dr. P.A.D.S.N. Wijesekara\\
(Supervisor)


\end{center}

%%%%%%%%%%%%%%%%%%%%%%%%%%%%%%%%%%%%%%%%%%%%%%%%%%%%%%%%%%%%%%%%%%%%%%%%%%%%%%%%%%%%%%%%%%%%%%%%%%
% END OF FILE
%%%%%%%%%%%%%%%%%%%%%%%%%%%%%%%%%%%%%%%%%%%%%%%%%%%%%%%%%%%%%%%%%%%%%%%%%%%%%%%%%%%%%%%%%%%%%%%%%%


\renewcommand{\thepage}{\roman{page}} % Start page numbering in roman

\chapter*{Abstract}
False accusation attacks pose a significant security challenge in Software-Defined Vehicular Networks (SDVN), where malicious entities deliberately accuse honest nodes of misbehavior to manipulate trust and reputation systems. Such attacks can severely disrupt routing decisions, degrade network performance, and undermine overall system stability. This study identifies four prominent variants of false accusation attacks: single-accuser opportunistic fabrication, Sybil-amplified consensus flooding, timing-based accusations during high-noise periods, and evidence spoofing through tampered or fabricated logs. \\ \\
However, existing SDVN security solutions largely address false accusations in isolation, lack robust mechanisms for privacy-preserving verification, and fail to intelligently model complex accusation behaviors across diverse attack variants, resulting in limited resilience and scalability.\\ \\
To address these threats, this research proposes a novel multi-layered defense framework that integrates blockchain technology, Zero-Knowledge Proofs (ZKPs), and Graph Neural Networks combined with Large Language Models (GNN-LLMs). Blockchain is employed to ensure immutable and decentralized logging of reputation and accusation records, while ZKPs enable privacy-preserving verification of accusation authenticity without revealing sensitive vehicular information. GNN-LLMs are leveraged to model SDVN topology and behavioral relationships, enabling the detection of anomalous accusation patterns and the intelligent interpretation of complex attack behaviors. \\\\
The proposed framework is evaluated through comprehensive simulation across all identified attack variants, demonstrating its effectiveness in mitigating false accusation attacks while preserving network performance and privacy. To the best of our knowledge, this work represents the first integrated use of blockchain, ZKPs, and GNN-LLMs specifically designed to counter false accusation attacks in Software-Defined Vehicular Networks.
\newpage

% Table of Contents
\tableofcontents
\newpage

% List of Figures
\listoffigures
\newpage

% List of Tables
\listoftables
\newpage

% Acronyms
\addcontentsline{toc}{chapter}{Acronyms} % Add to table of contents
\acuseall % Use all acronyms to ensure they appear in the list
\printacronyms
\newpage

\renewcommand{\thepage}{\arabic{page}} % Start page numbering in arabic 
\setcounter{page}{1} % start page numbering from 1
\setcounter{secnumdepth}{3}

% Main Content
\chapter{Introduction}
\section{Evolution of Networking Paradigms: From SDN to SDVN}
To understand the security challenges in Software-Defined Vehicular Networks, it is essential to examine the evolution from traditional Software-Defined Networking through Vehicular Ad-hoc Networks to the integrated SDVN architecture.

\subsection{Software-Defined Networking (SDN)}
Software-Defined Networking (SDN) represents a revolutionary paradigm shift by fundamentally decoupling the control plane from the data plane \cite{godanj2016simple}. In traditional networks, both intelligence for routing decisions (control plane) and packet forwarding (data plane) reside together within network devices. SDN addresses these limitations through architectural separation where the control plane is extracted and centralized into software-based SDN controllers, while the data plane remains in simplified network devices focusing solely on packet forwarding \cite{godanj2016simple}.

The SDN architecture consists of three distinct layers:
\begin{itemize}
    \item \textbf{Application Plane:} Network applications defining desired behaviors and policies, communicating through the northbound API \cite{kreutz2014software}.
    \item \textbf{Control Plane:} SDN controller maintaining global network view, making routing decisions, and translating policies into forwarding rules \cite{kreutz2014software}.
    \item \textbf{Data Plane:} Physical and virtual network devices forwarding packets according to flow tables populated by the controller \cite{kreutz2014software}.
\end{itemize}

\begin{figure}[H]
    \centering
    \includegraphics[width=0.85\textwidth]{diagrams/1_SDN.png}
    \caption{SDN Data Plane Architecture \cite{nunez2023brief}}
    \label{fig:sdn}
\end{figure}

\subsection{Vehicular Ad-hoc Networks (VANET)}
Vehicular Ad-hoc Networks (VANET) represent a specialized class of Mobile Ad-hoc Networks specifically designed for vehicle-to-vehicle (V2V) and vehicle-to-infrastructure (V2I) communication \cite{raja2010issues}. VANETs enable vehicles equipped with On-Board Units to communicate directly with each other and with Roadside Units deployed along roadways \cite{raja2010issues}.

Distinctive characteristics of VANET:
\begin{itemize}
    \item \textbf{High Mobility:} Vehicles move at varying speeds creating highly dynamic topology with frequent link disruptions \cite{raja2010issues}.
    \item \textbf{Dynamic Topology:} Network topology changes rapidly and unpredictably as vehicles enter, leave, or change lanes \cite{rehman2013vehicular}.
    \item \textbf{Variable Network Density:} Node density varies dramatically based on location, time, and events \cite{lee2021vanet}.
    \item \textbf{Distributed Decision Making:} Each vehicle makes local routing decisions without global network visibility \cite{raja2010issues}.
\end{itemize}

\begin{figure}[H]
    \centering
    \includegraphics[width=0.85\textwidth]{diagrams/2_VANET.png}
    \caption{Architecture of Vehicular Ad-Hoc Networks (VANETs) \cite{rehman2013vehicular}}
    \label{fig:vanet}
\end{figure}

\subsection{Software-Defined Vehicular Networks (SDVN)}
Software-Defined Vehicular Networks (SDVN) emerge as a convergence architecture integrating the programmability and centralized control of SDN with the mobility and distributed communication of VANET \cite{correia2017architecture}. SDVN addresses fundamental VANET limitations—particularly distributed routing in highly dynamic topologies—by introducing centralized intelligence while maintaining vehicle-to-vehicle communication \cite{li2016control}.

SDVN architectural components:
\begin{itemize}
    \item \textbf{SDN-Enabled Vehicles:} Vehicles as mobile SDN switches querying controllers for forwarding instructions \cite{li2016control}.
    \item \textbf{SDN-Enabled RSUs:} Hybrid devices for packet forwarding and controller-to-vehicle communication aggregation \cite{li2016control}.
    \item \textbf{Hierarchical Controllers:} Local controllers for region-specific routing with global controller coordination \cite{correia2017architecture}.
    \item \textbf{Hybrid Control:} Time-critical safety messages via direct V2V; non-urgent traffic via centralized SDN \cite{dhawankar2017software}.
\end{itemize}

\begin{figure}[H]
    \centering
    \includegraphics[width=0.85\textwidth]{diagrams/3_SDVN.png}
    \caption{Architecture of Software-Defined Vehicular Networks (SDVN) \cite{hama2025security}}
    \label{fig:sdvn}
\end{figure}

\subsection{Security Challenges in SDVN}
Software-Defined Vehicular Networks inherit security vulnerabilities from both Software-Defined Networking and Vehicular Ad-hoc Networks, while simultaneously introducing new attack surfaces due to their centralized control architecture and highly dynamic vehicular environment. Although SDVN improves network flexibility and routing efficiency, the tight coupling between mobile data-plane entities and centralized control logic increases the overall attack impact when security assumptions are violated \cite{arif2020sdn}.

One of the core security challenges in SDVN is maintaining reliable decision-making under highly dynamic conditions. Frequent topology changes, intermittent connectivity, and variable node density make it difficult to obtain accurate and consistent network state information \cite{arif2020sdn}. As a result, control-plane decisions may be based on incomplete, delayed, or noisy inputs, reducing the effectiveness of conventional security validation mechanisms \cite{arif2020sdn}.

The reliance on centralized or hierarchical controllers further complicates the security landscape \cite{hama2025security}. Controllers aggregate large volumes of network information and enforce global policies, making them attractive targets for compromise. Any disruption, manipulation, or failure at the control plane can propagate rapidly across the network, affecting routing stability, quality of service, and overall system reliability \cite{hama2025security}.

Additionally, SDVN must balance security enforcement with privacy preservation. Vehicles continuously exchange sensitive operational and contextual information, yet excessive disclosure of such data violates privacy requirements and regulatory constraints. Existing security solutions often struggle to simultaneously ensure data integrity, system robustness, and privacy protection in large-scale vehicular environments \cite{hama2025security}. These unresolved challenges highlight the need for advanced security mechanisms capable of supporting trustworthy operation in SDVN, thereby motivating the problem addressed in the following section.

\subsection{False Accusation Attacks in Software-Defined Vehicular Networks}
False accusation attacks in Software-Defined Vehicular Networks (SDVNs) occur when malicious vehicles, RSUs, or compromised controllers deliberately submit fabricated or misleading reports against honest nodes to manipulate trust and reputation systems. Since SDVN relies on data-plane feedback for centralized control decisions, such attacks can unjustly penalize legitimate entities, disrupt routing and service allocation, and degrade overall network stability. The highly dynamic topology, high mobility, and time-sensitive control decisions in SDVN further amplify the impact of false accusations, making timely and reliable verification difficult and allowing coordinated or opportunistic attackers to exploit trust mechanisms effectively.


\subsection{Limitations of Existing Works}
Existing security and trust management solutions for SDVN suffer from several key limitations in addressing false accusation attacks. Most approaches assume simplified or isolated attack behaviors and fail to capture coordinated, adaptive, and multi-variant accusation strategies under realistic vehicular conditions \cite{lee2012efficient}. Additionally, many solutions require extensive disclosure of behavioral evidence, compromising privacy \cite{ullah2025decentralized}, while relying on centralized or mutable reputation storage that is vulnerable to tampering and single-point failures. Furthermore, traditional rule-based or statistical detection mechanisms lack the intelligence to model complex relational and temporal patterns of accusation behavior, resulting in limited robustness and scalability in dynamic SDVN environments \cite{cardona2020software}.

\section{Problem Statement}
Existing trust and reputation systems in Software-Defined Vehicular Networks cannot reliably mitigate false accusation attacks under realistic conditions involving high mobility, coordinated adversaries, and privacy constraints, resulting in unjust penalization of honest nodes and degraded network trust.

\section{Objectives and Scope}
\subsection{Objectives}
The primary objective of this research is to develop a robust defense framework for mitigating false accusation attacks in Software-Defined Vehicular Networks (SDVNs). 

\singlespacing

The specific objectives are:
\begin{enumerate}
    \item \textbf{Analyze and model false accusation attacks in SDVNs } to understand their impact on trust and reputation systems under dynamic vehicular conditions, and to identify patterns of data-plane (vehicle/RSU) and control-plane (controller) attacks.
    \item \textbf{Design and implement a secure, tamper-resistant reputation management framework } combining blockchain-based logging, Zero-Knowledge Proofs for privacy-preserving verification, and Graph Neural Network–based anomaly detection to identify malicious accusations and coordinated attacks.
    \item \textbf{Evaluate the effectiveness of the proposed defense framework } through simulations in NS-3 integrated with a blockchain platform (e.g., Hyperledger Fabric), assessing attack detection accuracy, false positives, network performance, and overall Quality of Service.
\end{enumerate}

\subsection{Scope}
The scope of this research is defined to ensure focused investigation and practical feasibility. The study concentrates on the following aspects:
\begin{itemize}
    \item The research is limited to false accusation attacks targeting trust and reputation systems in Software-Defined Vehicular Networks.
    \item Both data-plane adversaries (malicious vehicles and RSUs) and control-plane adversaries (compromised or malicious SDN controllers) are considered within the threat model.
    \item The proposed framework focuses on the integration of blockchain, Zero-Knowledge Proofs, Graph Neural Networks, and Large Language Models as core defensive components.
    \item Evaluation is conducted using simulation-based experiments rather than real-world vehicular deployments, with performance metrics including detection accuracy, reputation stability, routing efficiency, and QoS impact.
    \item The scope excludes physical-layer attacks (e.g., jamming), traditional cryptographic key management protocols, and non-reputation-based network attacks that do not involve accusation manipulation.
\end{itemize}

This defined scope ensures that the research remains targeted toward developing a robust, privacy-preserving, and intelligent defense mechanism for false accusation attacks while maintaining relevance to real-world SDVN deployments.

\subsection{Methodological Contributions}
The key methodological contributions of this research are as follows:
\begin{itemize}
    \item \textbf{Multi-Layered Defense Framework:} A framework for Software-Defined Vehicular Networks that jointly integrates blockchain, Zero-Knowledge Proofs (ZKPs), and AI-based learning models to mitigate false accusation attacks across both data plane and control plane entities.
    \item \textbf{Decentralized Reputation Management:} A blockchain-based reputation mechanism with immutable logging and weighted endorser trust scoring, preventing unilateral or Sybil-amplified false accusations from influencing blacklist decisions.
    \item \textbf{Privacy-Preserving Accusation Verification:} A Zero-Knowledge Proof (ZKP) approach that allows vehicles and RSUs to prove accusation legitimacy without revealing sensitive identity or contextual information.
    \item \textbf{Graph-Based Anomaly Detection:} A Graph Neural Network (GNN) model to capture spatio-temporal and relational patterns of coordinated accusation behaviors in highly dynamic SDVN topologies.
    \item \textbf{LLM-Assisted Semantic Reasoning:} A semantic reasoning layer that analyzes unstructured controller logs and network events to interpret complex attack scenarios, generate explainable incident reports, and support administrator decision-making.
    \item \textbf{Smart Contract-Based Controller Trust Evaluation:} A controller trust mechanism implemented via smart contracts to detect compromised SDN controllers and automatically trigger failover, ensuring resilience against insider and control-plane attacks.
\end{itemize} 

\subsection{Implications of the Research}
The proposed framework has significant implications for the design of secure and trustworthy SDVN infrastructures. By ensuring robust mitigation of false accusation attacks while preserving privacy and network performance, the framework enhances the reliability of trust-based routing and decision-making in vehicular environments. The integration of explainable AI components further supports operational transparency and administrative control, making the solution practical for real-world deployment. Beyond SDVN, the proposed methodology can be adapted to other cyber-physical and software-defined systems that rely on distributed trust, such as smart transportation, industrial IoT, and intelligent edge networks.

\subsection{Novelty of the Proposed Work}
The novelty of this research lies in its problem formulation and attack modeling rather than in the individual techniques applied. It formulates false accusation attacks in Software-Defined Vehicular Networks as a multi-variant, cross-layer trust manipulation problem affecting both the data plane and control plane under realistic constraints such as high mobility and privacy preservation. Unlike existing approaches that treat false accusations as isolated events, the proposed model captures their temporal, relational, and behavioral dependencies, including coordinated and timing-based attack strategies.
Based on this formulation, the research proposes a unified defense framework where accusation validation and trust evolution are jointly guided by cryptographic consistency, historical trust behavior, and network-wide relationships. Blockchain, Zero-Knowledge Proofs, and GNN–LLM components are employed as enabling mechanisms aligned with the problem model, rather than as a novelty in themselves, distinguishing the work from fragmented or assumption-driven defenses in existing literature.


\subsection{Organization of the Paper}
The remainder of this paper is organized as follows. Chapter 2 presents a comprehensive review of related work on trust management and false accusation attacks in Software-Defined Vehicular Networks, highlighting existing limitations. Chapter 3 describes the system model, threat model, and the proposed multi-layered defense framework in detail. Chapter 4 outlines the project timeline and resources required for implementation and evaluation. Chapter 5 concludes the paper by summarizing the key contributions and discussing the significance of the proposed approach for secure and trustworthy SDVN deployment.

\chapter{Literature Review}
\section{Previous Works}

\subsection{Software-Defined Vehicular Networks (SDVN)}
The Software, Defined Vehicular Networks (SDVN) were proposed by applying SDN architecture concepts such as centralized control via separation of control,  plane from data,  plane,  higher programmability and traffic engineering to vehicular network scenarios where dynamic traffic between geographically distributed nodes, such as handovers, congestion, and strict latency requirements, are of utmost importance to secure network trust, routing efficiency and network stability \cite{cardona2020software}.  The architecture of SDVN is built on several fundamental components such as On,  Board Units (OBU) to support inter,  vehicular communication,  Road Side Units (RSU) acting as gateway between vehicles and infrastructure, and Trusted Authorities (TA) in charge of establishing trust relationships,  where control plane is layered in the same way as SDN with hierarchical SDN control plane where central controllers have global command of the network that can secure traffic flow control,  coordinated security policy enforcement, and access control management \cite{adnan2021towards}.  This single point of control however also presents certain vulnerabilities that can be exploited to manipulate reputation framework and trust systems in SDVN where successfully compromised control plane or RSU can influence entire network segments, with introduction of security concerns involving control plane communications, SDN controller digital identity impersonation attacks, and RSU corruption affecting the entire network,  leading to potential selection of reputation based false accusation attack by manipulating reputation management system where malicious nodes can coerce honest nodes to be falsely accused, direct all traffic towards attack nodes while blacklisting legitimate pathways,  significantly impair QoS with high packet loss while elevating communication delays, and destabilize trust system by proliferation of unstable trust scores \cite{adnan2021towards}.  Corresponding research gaps and defenses are identified with solutions combining decentralized verification with blockchain, based delivery of tamper,  proof reputation records, Zero Knowledge Proofs to enable validation without revealing additional information, and AI, enhanced anomaly detection to counter reputation based exploits such as single attacker opportunistic false accusations,  colluding crowdsourcing election,  or sustained timing manipulation during high,  rumor phases, Evidence falsification, and RSU/Sensor hardware co, opted collusions \cite{cardona2020software}.

\subsection{False Accusation Attacks in Vehicular Networks}
False accusation attacks are a critical vulnerability in vehicular networks where malicious nodes spread false alert messages accusing normal nodes as misbehaving nodes, with simulation showing accelerated throughput decrease and more severe network degradation than selfish nodes as false accusation nodes restrict legitimate nodes from routing,  thus contributing to efficiency and survivability degradation of networks \cite{lee2012efficient}.  Its core attack source involves malicious nodes on the forwarding path of the message falsely claiming that normal nodes have failed to deliver their data packets while forwarding packets,  which causes source nodes to falsely consider all normal nodes as misbehavers upon deletion of acknowledgment messages,  hence excluding legitimate nodes from the network, by reputational smearing in distributed cooperative intrusion detection system especially when nodes colluding or fake reporting of reputational values by compromised nodes, on account of diverse attack variants such as single,  accuser opportunistic spurious reports, Sybil, simulated votes to increase sequential falsity,  timed false accusations at high,  noise intervals which are indistinguishable from real error reports, evidence falsifying with forged logs and traceroute simulations,  GPS false routes with fabricated reporting, trust, disturbance attacks, and collusions with compromised gate,  kept infrastructure \cite{gyawali2020machine}, thereby resulting in malicious path removal with bad nodes being blacklisted, relocation of traffic load to attacker‘s nodes, Quality of Service degradation,  and instability of control,  plane with fluctuating trust scores \cite{che2022trust}.  Research complications still exists in devising robust countermeasures with decentralized validation via blockchain, enabled evidence, Zero,  Knowledge proof for witnesses to protect sensitive vehicular data, and machine learning empowered analysis of network topologies by Graph neural network, led anomaly detection models to find suspect clusters of lies in conjunction with Large Language Models for situational awareness and counterword generation to effectively nullify these emerging false accusation threats in Software,  Defined Vehicular Networks.

\subsection{Blockchain Technology in Vehicular Networks}
For trust management, blockchain has been considered as a promising solution for vehicular networks. Built upon distributed ledgers,  blockchain provides an immutable support for exchanging and storing reputation scores, trust evidence and behavior traces in a public and verifiable manner by using decentralized consensus,  and thus overcoming single point of failure and performance bottleneck encountered in centralized trust management approaches[1]. A distributed reputation management over consortium blockchain is realized by adopting three novel phases for global trust computation:  message verification and local trust metric calculation by vehicles, global trust computation by miners or RSUs and consensual addition of reputation throughput blocks to the tamper,  proof blockchain and dispute arbitration mechanisms for tracking malicious vehicles[2]. Existing schemes suffer from critical limitations:  trust evaluation based only on geographical neighbor that is insufficient to distinguish false accusations from malicious vehicles in the advantageous position;  lack of verification mechanisms of whether a node‘s accusation is genuine and whether the behavior of a node is malicious without yielding the underlying contents; unawareness to detect Sybil may network attack by suffusing believable fake identities to unknowing vehicles,  or collusion attack involving with compromised infrastructure components to perform consensus attacks,  or critical lack of intelligent analysis scheme to grasp malicious accusation patterns based on network topology or utter logs and behavioral evidence contexts [3]. Missing yet are essential researches on verifiable trust computing framework that employs privacy, saving Zero,  Knowledge Proof for accusation verification without revealing sensitive vehicular contents, intelligent anomaly detection scheme based on GNN for network topology graph analysis to determine malicious collusion and anomalous collection of accusation evidences,  LLM, based contextual contents analysis framework to incorporate graphs and undeniable logs for potential confrontation against single,  accuser, Timing, based, Sybil, evidence,  spoofing and collusion attacKs in SDVNRs with real, time efficiency and effective game mitigation tool for maximized trustworthy routing,  credibility and stability in verifiable malicious accusation environment.

\subsection{Zero-Knowledge Proofs (ZKP)}
Zero, Knowledge Proof (ZKP) is also a zero, knowledge proof system used in cryptography to prove of proof of the truth of knowledge statement without providing any knowledge as to the content of the statement,  other than its truth value (or interval of truth) and justifications of such, which makes it useful for privacy and security preservation applications.  As ZKP can help ascertain the validity of information without relinquishing privacy, it also helps in privacy and security preservation applications in vehicular networks. Vehicles can adopt ZKP to demonstrate attributes such as identity, privilege, or possession of some information without exposing too much sensitive data and compromising privacy and privacy preservation applications ZKP can be used in three specific applications: 1) privacy preservation authentication, 2) privacy preservation location, or information on zero, knowledge range proofs of vehicle locations, 3) privacy preservation anonymous credential verification for accountability systems without requiring any third parties, as explored in \cite{zhou2024leveraging,  kalmykov2022using}.  If ZKP is promising, the application of ZKP and trusted third, party verifiers for fairly validating accusation proofs in reputation, based trust management systems remain problematic;  ZKP based approaches that have been proposed so far, can be categorized under authentication of information and credential verification applications, but do not present the solutions for node behavior and accusation verification in privacy preserving manner without revealing sensitive vehicular data, \cite{kalmykov2022using}.  Very few papers used ZKP for presentation and argumentation that proved the accused node forwarded packets in a way that would otherwise be visible to the network,  proved the veracity of the accusation, or proved the negative impact of the accusation, proved that the accused had all the right information without revealing the entire information, proved that behavioral expectations were satisfied while addressing privacy issues, proved that reputation was calculated accurately without disclosing the value, addressed a number of false,  accusation attack signatures in context of false, message attack, integrated with compromised infrastructure, double, acuser, elite, rumor, attack, and injected attack, but not in context of defender,  stable, trues,  learning, and specific attack (falsified,  fabricated,  aggregated, flooding,  in,  time,  accurate,  silent,  audio, evidence,  spurious, attack, by colluding, traffic, injection,  attack,  using compromised infrastructure, and double,  acuser, elite,  attack). In addition, the verifiability of reputations,  trust optimization, and privacy preserved ZKP for defense against false accusation attack, in dynamic SVVNs has been rarely explored.

\subsection{Artificial Intelligence in Vehicular Networks}
GNNs have been identified as an innovative technology for unraveling intricate network configurations within Software,  Defined Vehicular Networks by means of message passing strategies that assimilate neighboring node information for deciphering relational trends, thereby facilitating the detection of anomalous accusation clusters that deviate from standard network behavior, GNN,  based solutions trained in a self, supervised manner exhibiting heightened network intrusion detection effectiveness, aiding in network topology investigation and edge feature examination to discern cohesive malicious conduct including Sybil attacks,  collusion,  tampering,  exaggeration of reputation targets,  where several nodes forge unfounded accusations, and other deception tactics such as using bogus network traces to perpetuate harmful fabrications by LLMs, by exploiting linguistic patterns in accusing messages, authenticating hypothesis truthfulness, and deploying automated response creation \cite{zoubir2024integrating, belcastro2025enhancing}.  Prominent LLMs like BERT,  based models and transformer architectures have shown exceptional performance in cybersecurity through profound contextual comprehension and natural language processing of diverse security data such as system logs, network flows, accusation messages, and threat analysis reports, with promising applications in examining accusation message semantics, claim validity verification,  detecting linguistic markers of malicious reports, and synthesizing automated reactions \cite{belcastro2025enhancing}.  An integrated approach utilizing the detection capabilities of GNNs with the contextual analysis capabilities of LLMs holds potential for effective False Accusation attack detection in vehicular networks.  GNNs can identify anomalous accusation structures,  relational patterns, and aggressive attack signatures like sudden accusation clusters or multi, node collusions, while LLMs can analyze graph,  derived data with ancillary natural language logs for contextual threat evaluation, automated response synthesis, and reasoning about complex assaults with fabricated evidence,  manipulated communication data, high,  noise, infiltration timing, and evidence,  spoofing tactics like traceroute mimicry targeted at attacking actors \cite{chattopadhyay2024gnn}.  However, still unmet challenges include evolving integrated GNN,  LLM solutions that synergize network topology insight for detecting nuanced false accusations, such as isolated,  accuser exploitation and Sybil, enhanced consensus, spamming, with capabilities for rapid real, time operation to satisfy safety needs,  secure user privacy via fusion with Zero,  Knowledge Proofs, and establish unstoppable trust with integration with blockchain for indelible evidence record, keeping in distributed Software,  Defined Vehicular Networks emphasizing route optimization and Quality of Service assurance \cite{belcastro2025enhancing}.

\section{Gaps in Literature}
\subsection{Limited Blockchain and AI Integration for SDVN }
Existing studies either investigate blockchain,  based trust management or AI,  based intrusion detection for vehicle networks.  Blockchains enable record tamper resistance for reputations and decentralized consensus \cite{gazdar2022decentralized}, while Machine Learning, based AI enables anomaly detection to identify case, specific random behaviors \cite{chattopadhyay2024gnn}.  Unfortunately,  current frameworks of either one alone are inadequate for decanting compromised vehicles for false accusation defenses.  Limited research endeavors involve at least a Graph Neural Network (GNN) for topology analysis to discover abnormal accusation group formation trend \cite{zoubir2024integrating} and at least a Large Language Model (LLM) for semantics reasoning of attack evidence and reports onto blockchain, establishing a verifiable reputation trust basis.  Notably, solutions to incorporate weighted endorser trust logic with AI, powered analysis of vehicle behavior pattern deviations to prevent malicious framing and AI, based discriminator between genuine and malevolent group‘s false accusations \cite{che2022trust} are missing. A multi, layer architecture to combine blockchain decentralized reputation record, GNN accusation structure anomaly detection, and LLM semantic attack report analysis will form a robust false accusation defense platform against multi, lateral false accusation types such as Sybil, based consensus flooding \cite{jaballah1904software} and noise, resilient causality, based false report generation in high velocity SDVN.

\subsection{Insufficient Privacy-Preserving Verification Mechanisms for Reputation Systems}
Existing reputation,  based trust management mechanisms for SDVN do not employ any privacy, preserving techniques to convince others of an honest node’s reputation score, and instead require the collection and sharing of information such as packets forwarded,  messages sent, and routing states for other nodes to verify the truth of any expressed accusations \cite{gazdar2022decentralized}. Zero,  Knowledge Proof (ZKP) applications have been developed for use in vehicular networks for verifying identities and credentials \cite{lavin2024survey} but have yet to be utilized within reputation systems to prove the accuracy of accusations without restoring the sensitive information withheld during the proof generation \cite{kalmykov2022using}.  None of the current literature describe ZKP schemes for demonstrating that a node consistently forwarded packets correctly without revealing paths,  or that an attacker’s accomplice has evidence of an honest message in their communications logs, or that reputation scores derive from proper trust evaluations without revealing sensitive details. Such schemes would help fill the security/privacy gap experienced by current SDVN reputation mechanisms where system designers are forced to sacrifice either one for the other \cite{kalmykov2022using}.  There is a pressing need to explore the use of Zero,  Knowledge Proofs with decentralized blockchain,  based reputation systems \cite{lavin2024survey} so that SDA vehicles would be able to verify nodes are honest without exchanging location,  velocity,  and destination information that adversaries could leverage for false attacks.

\subsection{Lack of Intelligent Detection for Coordinated False Accusation Attacks}
Existing misbehavior detection schemes deployed in SDVN use techniques such as statistical anomaly detection, threshold,  based filtering, and historical pattern matching that successfully detect simple, isolated packet dropping or naked false reporting (i. e., in the absence of coordinated collusion),  but are ineffective against more sophisticated multi, faceted collaboratively coordinated attacks \cite{gyawali2020machine}.  Conventional machine learning detection schemes that utilize CNNs and RNNs that analyze feature vectors derived from individual node behaviors have no means to model the highly complex relational dependencies in dynamic vehicular network topologies where trust relationships and accusation flows occur across multiple agencies to form broad interdependent clusters \cite{chang2025blockchain}.  The threats of interest require new solutions to be able to detect Sybil, boosted consensus flooding that occurs when a fleet of fake vehicles produce a massive,  conspicuous consensus and timing,  based false indications of network impairment that occur when collaborative malicious actors time their report transmissions to coincide with periods of high network congestion in order to obfuscate malicious reporting in the noise,  as well as swarms of colluding malicious RSUs and controllers producing false accusations just using conventional, standard models of anomaly detection \cite{ahmed2022privacy}.  Few detection models or frameworks have used GNNs to model a vehicular network as a graph in order to identify topological anomalies such as star, shaped accusation graphs, suspicious community clusters, or rampant community, level attack behavior against valid vehicles. There is great potential in designing graph neural network frameworks that can identify suspicious network topology patterns and detect anomalous clusters of accusations against valid vehicles to alert on colludal attacker teams,  as well as leverage attention mechanisms between message, passing iterations to filter out malicious timing,  based allegations,  during attack events occurring with high levels of network noise.

\subsection{Insufficient Immutable Evidence Logging and Historical Verification}
Current vehicular network security systems don‘t offer reliable chains, of, custody of accusation evidence,  so malicious nodes can create and alter packet logs, routing path traces and sensor measurements without notice \cite{chang2025blockchain}.  The existing vehicular security systems don‘t keep cryptographically verifiable timestamps alleging when behavioral evidence was captured and logged,  so malicious nodes can later generate fake evidence replicating the past behaviors of honest nodes to frame them \cite{ahmed2022privacy}.  Existing systems lack blockchain,  based evidence integrity protection by storing the original packet hashes in an immutable proof, of,  capture event \cite{gazdar2022decentralized},  so malicious nodes cannot forge backward, looking false accusations.  Existing systems lack forensic audit trail recording who accused who,  who provided what evidence,  which endorsers cast votes and how their reputation scores changed with cryptographically unforgeable signatures \cite{gazdar2022decentralized},  which diminishes the possibility of further uncaught frauds. These limitations can be overcome by creating a stable blockchain evidence management system maintaining immutability and traceability of evidence,  keeping unforgeable timestamps of capture events within reasonable timing windows \cite{ahmed2022privacy},  keeping a full forensic audit trail of accusation events, and providing automatic trust weight reduction of malicious nodes.

\section{Table of Literature Review Summary}

\begingroup
\renewcommand{\arraystretch}{1.15}
\setlength{\tabcolsep}{5pt}
\begin{longtable}{|p{3cm}|p{6.5cm}|p{6.5cm}|}
\caption{Comparison of prior work limitations and proposed contributions for SDVN false-accusation mitigation.}
\label{tab:false_accusation_detection} \\
\hline
\textbf{Category} &
\textbf{Limitation} &
\textbf{Proposed solution} \\
\hline
\endfirsthead

\multicolumn{3}{c}%
{{\bfseries \tablename\ \thetable{} -- continued}} \\
\hline
\textbf{Category} &
\textbf{Limitation vs. SDVN false-accusation mitigation} &
\textbf{Proposed contribution} \\
\hline
\endhead

\endfoot

\hline
\endlastfoot

False accusation / reputation poisoning &
No tamper-proof chain-of-custody; weak against Sybil-amplified flooding, timing-noise exploitation, and forged evidence; no privacy-preserving verification (\cite{lee2012efficient}) &
Blockchain-backed immutable evidence logging with ZKP-based verification and weighted endorser trust scoring to prevent unilateral false accusations \\
\hline

ML-assisted misbehavior detection &
Can be evaded by stealthy attackers below thresholds; limited integrity assurance for logs/evidence; not SDVN control-plane aware (\cite{gyawali2020machine}) &
GNN-based relational anomaly detection plus LLM-assisted log reasoning, anchored to blockchain-verified evidence and controller-aware mitigation \\
\hline

Trust distortion / Sybil-related manipulation &
Does not provide cryptographic proof of accusation authenticity; limited detection of coordinated accusation graphs (\cite{che2022trust}) &
Sybil-resistant registration, weighted endorser consensus, and GNN detection of coordinated accusation structures \\
\hline

Blockchain trust / reputation &
Immutability alone cannot stop false accusations; limited Sybil resistance; weak intelligent detection for coordinated accusation campaigns (\cite{li2020blockchain}) &
Blockchain combined with ZKP-based accusation validation and AI-driven detection of coordinated false-accusation patterns \\
\hline

Decentralized trust aggregation (blockchain) &
Heuristic/proximity-based trust exploitable; lacks privacy-preserving evidence validation; limited resilience to coordinated flooding/collusion (\cite{gazdar2022decentralized}) &
Privacy-preserving evidence validation via ZKPs, multi-endorser consensus, and anomaly detection to resist coordinated flooding/collusion \\
\hline

Blockchain + learning-assisted trust &
Not tailored to explicit false-accusation variants (timing-based accusations, forged logs, controller fabrication); potential overhead without end-to-end verification (\cite{chang2025blockchain}) &
Explicit attack-variant modeling with end-to-end verification, combining blockchain, ZKP proofs, GNN detection, and LLM reasoning \\
\hline

GNN-based anomaly detection &
Typically detection-only unless integrated with verifiable reputation evidence; no direct accusation authenticity mechanism (\cite{zoubir2024integrating}) &
GNN detections tied to blockchain-validated evidence and smart-contract enforcement for accusation authenticity \\
\hline

GNN security analytics &
Does not ensure evidence integrity/privacy; mitigation not explicitly tied to accusation workflows and trust updates (\cite{chattopadhyay2024gnn}) &
Integrated trust workflow linking GNN outputs to ZKP-based validation, reputation updates, and enforcement actions \\
\hline

Advanced DL for security analytics &
Not accusation-specific; lacks blockchain/ZKP-style verifiability for trust decisions under false accusations (\cite{belcastro2025enhancing}) &
LLM-assisted contextual analysis of logs with blockchain-trusted state and ZKP verification for explainable decisions \\
\hline
\end{longtable}
\endgroup


\chapter{Methodology}
\section{Research Design}

\begin{figure}[H]
	\centering
	\includegraphics[width=\linewidth]{diagrams/FYP_Architecture_small.png}
	\caption{Integrated High‑Level Architecture of the Proposed SDVN Framework.}
	\label{fig:arch}
\end{figure}

Figure \ref{fig:arch} illustrates the integrated high-level architecture of the proposed SDVN framework.
As shown in Figure \ref{fig:arch}, the AI analysis layer sits at the top and acts as the intelligence and reporting component.
It consumes two main categories of inputs: (i) the current network state (e.g., node status and trust values)
and (ii) continuous logs and events generated from RSUs and controllers (e.g., warnings, repeated anomalies,
and observed misbehavior). A graph-based model processes the network as connected entities (vehicles, RSUs,
and controllers) to identify suspicious patterns from relationships and traffic behavior. Its output is then
passed to a language-based reasoning component that produces a concise, human-understandable explanation of
what is happening, why it is suspicious, and what response is recommended. The final output of this layer is
an attack detection decision plus a response report that is forwarded to the enforcement path.

The blockchain layer is implemented using \textbf{Hyperledger Fabric} to maintain a shared, tamper-resistant
record of security-relevant updates across multiple controllers. This layer contains three smart contract
modules (SC1--SC3) that coordinate different types of trust decisions:
\begin{itemize}
    \item \textbf{SC1: Registration and Authentication} -- manages joining/participation status of nodes and
    stores whether a node is allowed to operate in the network.
    \item \textbf{SC2: Reputation Management} -- stores behavior scores and reputation updates derived from
    validated event evidence, preventing one controller from unfairly manipulating trust values.
    \item \textbf{SC3: Controller Trust} -- maintains trust levels for controllers themselves, supporting
    safer operation if a controller becomes unreliable or compromised.
\end{itemize}
When controllers submit updates (e.g., a reputation change or a controller trust update), Fabric validates
the update through endorsement and ordering before committing it to the ledger, ensuring that critical trust
decisions are jointly accepted and auditable.

The control plane contains multiple SDN controllers (Controller 1--3) that manage their corresponding RSU
regions. Each controller:
\begin{itemize}
    \item collects measurements and alerts from RSUs,
    \item produces structured logs/events for security evaluation,
    \item exchanges coordination messages with peer controllers through east--west communication to keep
    policies consistent,
    \item queries the Fabric-maintained state (participation, reputation, controller trust) before applying
    sensitive actions,
    \item enforces response actions (e.g., isolation, access restriction, forwarding adjustments) back toward RSUs.
\end{itemize}
In this design, controllers are both \textit{contributors} (they submit events/updates to Fabric) and
\textit{consumers} (they read the trusted state from Fabric to guide enforcement decisions).

The data plane contains vehicles and RSUs where actual wireless communication occurs.
Vehicles exchange messages directly (V2V) and also interact with nearby RSUs (V2I).
RSUs serve as the operational gateway between vehicles and controllers by:
\begin{itemize}
    \item relaying traffic and connectivity between vehicles and the control plane,
    \item collecting local observations (e.g., abnormal message rates, repeated suspicious behavior),
    \item forwarding summarized observations and alerts to the assigned controller for analysis and logging.
\end{itemize}
After the control plane decides a mitigation action (guided by the trusted Fabric state and AI reports),
the action is pushed down to RSUs, which then apply it to the local vehicular environment (e.g., restricting
a node’s access through network rules or prioritization).


\section{Threat Model}
The adversary model adopts a Zero-Trust Architecture where any network component may potentially be compromised. The threat model defines attacker capabilities, attack scenarios, and security assumptions \cite{adnan2021towards}.

\subsection{Attacker Capabilities}
\textbf{C1 - Data Plane Compromise:} Attackers can compromise up to n-f vehicular nodes and RSUs where n is the total number of nodes. Compromised nodes can participate in routing and reputation voting while executing malicious behaviors such as false accusations or packet manipulation.

\textbf{C2 - Control Plane Compromise:} Adversaries may compromise up to nc-1 SDN controllers out of nc total controllers (e.g., 3 out of 4). The blockchain ordering service tolerates up to $\lfloor(n_o-1)/3\rfloor$ Byzantine faults through BFT-SMaRt consensus, with additional resilience provided by the Controller Trust Evaluation Smart Contract \cite{yahiatene2018blockchain}.

\textbf{C3 - Wireless Communication Attacks:} Attackers can intercept, modify, replay, or flood V2V and V2I wireless transmissions over IEEE 802.11p DSRC channels.

\textbf{C4 - Computational Bounds:} Adversaries are computationally powerful but bounded by classical computing. Post-quantum cryptography (FALCON-1024, Kyber-1024) protects against future quantum threats, while frequent key rotation mitigates classical cryptanalysis \cite{bensasson2018scalable}.

\subsection{Attack Scenarios}
\textbf{Scenario 1 - Single-Accuser Fabrication:} A privileged malicious node with high reputation fabricates false evidence against honest vehicles, leveraging trusted status to cause blacklisting.

\textbf{Scenario 2 - Sybil-Amplified Flooding:} Multiple fake identities coordinate to flood the reputation system with false accusations, creating artificial consensus.

\textbf{Scenario 3 - Timing-Based Accusations:} Attackers exploit legitimate network stress events (congestion, handovers) to mask malicious accusations as genuine failures.

\textbf{Scenario 4 - Evidence-Spoofing:} Adversaries fabricate tampered packet logs, forged routing traces, or manipulated sensor data to frame honest nodes.

\section{System Assumptions}
The framework operates under the following security and infrastructure assumptions:
\begin{itemize}
    \item \textbf{A1 - Cryptographic Security:} All cryptographic primitives (FALCON-1024, Kyber-1024, RSA-2048, ECDH-256, AES-256, HMAC-SHA-256) are secure against known attacks when correctly implemented \cite{bensasson2018scalable}.
    \item \textbf{A2 - Byzantine Fault Tolerance:} BFT-SMaRt consensus tolerates up to $\lfloor(n_o-1)/3\rfloor$ Byzantine faults among orderers, maintaining safety and liveness \cite{castro1999practical}.
    \item \textbf{A3 - Bootstrap Trust:} At least one trusted controller exists during initial deployment for credential distribution and blockchain initialization.
    \item \textbf{A4 - Physical Layer Availability:} Underlying communication infrastructure (IEEE 802.11p, cellular backhaul) provides sufficient availability for safety-critical applications.
    \item \textbf{A5 - Time Synchronization:} All entities maintain loosely synchronized clocks ($\pm$1 second) via GPS or NTP for timestamp-based replay detection.
    \item \textbf{A6 - Computational Resources:} Vehicles and RSUs possess sufficient processing power (multi-core CPUs, HSMs) for cryptographic operations and ML inference.
    \item \textbf{A7 - Registration Authority:} A trusted registration authority (manufacturer, operator, or government agency) issues initial credentials tied to physical vehicle identities.
    \item \textbf{A8 - Honest Majority:} The majority of network participants are honest, with high-trust endorsers predominantly following protocol specifications.
\end{itemize}


\section{Overview of the False Accusation Attack}
\subsection{Single-Accuser Opportunistic Fabrication (Malicious Vehicle/RSU)}
\begin{figure}[H]
    \centering
    \includegraphics[width=0.7\textwidth]{diagrams/Single_Accuser/Single_Accuser_DV.png}
    \caption{Single-accuser fabrication when a vehicle/RSU is malicious (data-plane).}
    \label{fig:overview_single_accuser_dv}
\end{figure}

Figure~\ref{fig:overview_single_accuser_dv} tells the data-plane story step by step. First, in
\textbf{(step 1) Position of the Attacker}, a malicious vehicle/RSU positions itself near the honest node.
Next, in \textbf{(step 2) Fabrication of the Accusation}, it forges a false accusation packet. The attacker then
uploads this claim in \textbf{(step 3) Reporting to Brain (The Upload)} via the RSU to the controller. In
\textbf{(step 4) The Decision (Processing)}, the controller processes the report and updates the trust decision.
Finally, \textbf{(step 5) The Execution of the Consequence (The Block)} is issued, and the honest node is
blacklisted and isolated.

\subsection{Single-Accuser Opportunistic Fabrication (Malicious Controller)}
\begin{figure}[H]
    \centering
    \includegraphics[width=0.7\textwidth]{diagrams/Single_Accuser/Single_Accuser_CV.png}
    \caption{Single-accuser fabrication when the controller is malicious (control-plane).}
    \label{fig:overview_single_accuser_cv}
\end{figure}

Figure~\ref{fig:overview_single_accuser_cv} narrates the control-plane case where the controller is malicious.
In \textbf{(step 1) Target Identification}, the controller selects an honest vehicle as the victim. It then performs
\textbf{(step 2) Internal Reputation Fabrication} by creating false evidence against that node. Without external checks,
the controller issues \textbf{(step 3) Blacklist Command}. Through \textbf{(step 4) Network-Wide Propagation}, the blacklist is
sent across RSUs and neighboring nodes. The sequence ends in \textbf{(step 5) Complete Isolation}, where the victim is
fully cut off from the network.

\subsection{Sybil-Amplified Consensus Flooding (Malicious Vehicle)}
\begin{figure}[H]
    \centering
    \includegraphics[width=0.7\textwidth]{diagrams/Sybil_Attack/Sybil_Attack_DV.png}
    \caption{Sybil-amplified consensus flooding with attacker-generated Sybils (data-plane).}
    \label{fig:overview_sybil_dv}
\end{figure}

Figure~\ref{fig:overview_sybil_dv} shows a data-plane Sybil attack as a staged narrative. In
\textbf{(step 1) Sybil Generation}, the attacker creates multiple fake identities. Those Sybils then begin
\textbf{(step 2) The Flooding (The Attack)} by sending coordinated false accusation packets toward the RSU.
These accusations are forwarded in \textbf{(step 3) Reporting to Brain (The Upload)} to the controller, which
perceives them as coming from many distinct nodes. The controller performs \textbf{(step 4) The Decision (Processing)},
interpreting the volume as consensus. The story ends at \textbf{(step 5) The Consequence (The Block)}, where the
victim is blacklisted via revocation commands.

\subsection{Sybil-Amplified Consensus Flooding (Malicious Controller)}
\begin{figure}[H]
    \centering
    \includegraphics[width=0.7\textwidth]{diagrams/Sybil_Attack/Sybil_Attack_CV.png}
    \caption{Sybil-amplified consensus flooding when the controller fabricates Sybils (control-plane).}
    \label{fig:overview_sybil_cv}
\end{figure}

Figure~\ref{fig:overview_sybil_cv} presents the control-plane Sybil story where the controller itself
manufactures consensus. In \textbf{(step 1) Create Phantom Sybil IDs}, the controller generates fictitious identities
(S1*, S2*, S3*). It follows with \textbf{(step 2) Insert Fabricated Reports}, injecting false accusations tied to those IDs.
To legitimize the appearance of agreement, the controller performs \textbf{(step 3) Data Retrieval for Trust Evaluation},
then applies \textbf{(step 4) Trust Evaluation (Sybil-amplified consensus)} to justify a downgrade. The sequence ends with
\textbf{(step 5) The Consequence (The Block)} as the victim is blacklisted through controller-issued revocation commands.

\subsection{Timing-Based Accusations During High-Noise Periods (Malicious Controller)}
\begin{figure}[H]
    \centering
    \includegraphics[width=0.7\textwidth]{diagrams/Timing_Based_Accusations/Timing_Based_Accusations_CV.png}
    \caption{Timing-based accusations when the controller is malicious (control-plane).}
    \label{fig:overview_timing_cv}
\end{figure}

Figure~\ref{fig:overview_timing_cv} shows a control-plane timing attack as a sequence. In
\textbf{(step 1) Continuously Monitoring the Logs}, the malicious controller watches for packet-loss events.
When \textbf{(step 2) Legitimate Packet Loss (due to high noise)} occurs, the controller seizes the moment and
performs \textbf{(step 3) Opportunistic Fabrication}, framing the noise-induced loss as misbehavior. It then
executes \textbf{(step 4) Blacklist with Proof to All Nodes and RSU}, broadcasting revocation commands that isolate
the victim.

\subsection{Timing-Based Accusations During High-Noise Periods (Malicious Vehicle, Honest Controller)}
\begin{figure}[H]
    \centering
    \includegraphics[width=0.7\textwidth]{diagrams/Timing_Based_Accusations/Timing_Based_Accusations_DV.png}
    \caption{Timing-based accusations when the attacker is in the data plane (honest controller).}
    \label{fig:overview_timing_dv}
\end{figure}

Figure~\ref{fig:overview_timing_dv} narrates the data-plane timing attack under an honest controller. The attacker
starts by \textbf{(step 1) Sending too many packets during high-noise periods} to induce congestion. This leads to
\textbf{(step 2) Legitimate Packet Loss (due to high noise)} across the links. While loss persists, the attacker keeps
\textbf{(step 3) Checking the Logs} to pinpoint the moment of maximum degradation, then issues a \textbf{(step 4) Fabricated
Accusation (timed to align with high noise)}. The honest controller proceeds with \textbf{(step 5) Controller Validates
Accusation}, and if accepted, it completes \textbf{(step 6) Sending Blacklist with Proof to all Nodes and RSU}, resulting
in the victim’s isolation.

\subsection{Evidence-Spoofing Attack (Malicious Vehicle Collaboration)}
\begin{figure}[H]
    \centering
    \includegraphics[width=0.7\textwidth]{diagrams/Malicious Node Attack - Evidence Spoofing.png}
    \caption{Evidence-spoofing when malicious vehicles collude.}
    \label{fig:overview_evidence_node}
\end{figure}

Figure~\ref{fig:overview_evidence_node} shows how colluding vehicles frame an honest node using tampered evidence.
The story begins with \textbf{(step 1) Vehicle Send Legitimate Traffic}, where normal packets flow toward the honest RSU.
Next, in \textbf{(step 2) Multiple Vehicles Coordinates and Collaboratively Fabricate the Evidence}, the attackers in the
collusion zone craft malicious packets and forward the falsified evidence toward the controller. Meanwhile, the RSU
performs \textbf{(step 3) RSU Sends the Legitimate Traffic to the Controller}, so the controller sees a mix of legitimate
and tampered records. The sequence ends at \textbf{(step 4) Controller Trusting the Seemingly Corroborated Tampered Evidence},
leading to blacklisting of the target victim.

\subsection{Evidence-Spoofing Attack (Malicious RSU)}
\begin{figure}[H]
    \centering
    \includegraphics[width=0.7\textwidth]{diagrams/Malicious RSU Attack - Evidence Spoofing.png}
    \caption{Evidence-spoofing when the RSU is malicious.}
    \label{fig:overview_evidence_rsu}
\end{figure}

Figure~\ref{fig:overview_evidence_rsu} describes the RSU-side spoofing scenario. First, in \textbf{(step 1) Nodes Send
Legitimate Traffic to RSU}, honest vehicles forward normal packets to the RSU. The RSU then executes
\textbf{(step 2) RSU Intercepts, Tampers the Packet, and Tampered Packet Sent to the Controller}, replacing the evidence
before it reaches the controller. Finally, the process concludes with \textbf{(step 3) Victim BlackListed}, as the
controller trusts the spoofed packet trail and isolates the honest node.

\subsection{Evidence-Spoofing Attack (Malicious Controller)}
\begin{figure}[H]
    \centering
    \includegraphics[width=0.7\textwidth]{diagrams/Malicious Controller Attack - Evidence Spoofing.png}
    \caption{Evidence-spoofing when the controller is malicious.}
    \label{fig:overview_evidence_controller}
\end{figure}

Figure~\ref{fig:overview_evidence_controller} shows evidence spoofing from a compromised controller. In
\textbf{(step 1) Both Vehicles Send Legitimate Traffic}, normal packets move through the network. The honest RSU
performs \textbf{(step 2) RSU Forwards Legitimate Packet to Controller}, delivering unmodified evidence. The attack
occurs at \textbf{(step 3) Malicious Controller Itself Tampers The Packet \& Targets Vehicle B Unfairly}, where the
controller alters the evidence and fabricates blame, leading to the victim’s blacklist action.


\section{Proposed mitigation framework}
\subsection{Proposed system architecture}
\begin{figure}[H]
	\centering
	\includegraphics[width=1\textwidth]{diagrams/simple_blockdiagram.png}
	\caption{Simple Block Diagram of the Proposed Mitigation Framework.}
	\label{fig:simple_block}
\end{figure}

As shown in Figure~\ref{fig:simple_block}, the system operates as a closed-loop workflow.
The \textit{Registration and Authentication Module} controls which entities are allowed to participate,
while the \textit{Location Verification and Behavioral Monitoring} component continuously observes node
activity and produces \textit{logs and events}. These logs are forwarded to the \textit{Trust Update Core},
which processes security-relevant evidence and updates three trust functions: \textit{participation status}
(SC1), \textit{reputation updates} (SC2), and \textit{controller trust} (SC3). The validated outputs are
recorded in the \textit{shared trusted state/ledger}, which represents the latest trusted network state.

In parallel, the \textit{AI Analysis and Report} block reads both the logs/events and the trusted state
from the ledger to detect abnormal behavior and generate a recommended response. Finally, the
\textit{Mitigation and Enforcement} block applies the response actions through the control/monitoring path,
and the outcomes are fed back into the trust update process to keep future decisions consistent with the
recorded history.


\subsubsection{Blockchain-Based Decentralized Trust Management}
\label{subsec:blockchain-trust}


The core of our approach utilizes blockchain technology to eliminate single points of failure
inherent in centralized SDN architectures \cite{sharma2017distblocknet}. A permissioned blockchain serves as
a distributed ledger that records authentication events, routing-related security decisions,
and reputation changes immutably. This prevents attackers from erasing evidence of malicious
behavior or manipulating historical records. The blockchain operates across multiple SDN
controllers, forming a consortium where critical decisions require consensus among trusted
nodes \cite{yahiatene2018blockchain}. Figure~\ref{fig:functional_diagram} summarizes how these blockchain
capabilities are used during node admission (registration and authentication) and how trust
decisions are enforced.

Three specialized smart contracts govern different security aspects. First, the Registration
and Authentication Smart Contract manages node identity verification using Zero-Knowledge
Proofs (ZKPs), allowing vehicles/RSUs to prove authenticity without revealing sensitive
identifiers \cite{bensasson2018scalable}. As shown in Figure~\ref{fig:functional_diagram}, when a node attempts
to join the network, the workflow branches depending on whether it is a \emph{first-time access}.
For first-time access, a Registration Authority performs out-of-band verification and checks
physical identity attributes (e.g., VIN/serial number). After successful verification, cryptographic
credentials are issued and the node creates its key materials and ZKP-related artifacts before
submitting an initial registration request (left branch of Figure~\ref{fig:functional_diagram}).
For returning nodes (right branch), the node generates a session request, constructs a ZKP of
identity, and sends an authentication request to the controller, which then retrieves the node
record from the blockchain (``Blockchain: Retrieve Node Record'') as illustrated in
Figure~\ref{fig:functional_diagram}. The ZKP validity decision (``ZKP Valid?'') ensures that only
registered entities can proceed without exposing private identity information \cite{bensasson2018scalable}.

Second, the Reputation Management Smart Contract implements weighted endorser trust
scoring to prevent false accusations from influencing blacklist decisions \cite{alshaibani2023blockchain}. In the
workflow of Figure~\ref{fig:functional_diagram}, this corresponds to the trust decision stage
(``Trust Score Sufficient?'') after ZKP verification. Rather than trusting a single report, reputation
updates and punitive actions require endorsement from multiple parties, where each endorser’s
vote is weighted by historical accuracy. New nodes and previously malicious nodes carry minimal
weight, preventing Sybil attackers from immediately impacting reputation scores. Reputation
changes require consensus from multiple high-trust endorsers, typically requiring agreement
from at least two-thirds of weighted trust scores \cite{bessani2014state}. If trust is insufficient, the system
follows the ``Blacklisted?'' decision path: blacklisted nodes are rejected, while borderline cases
trigger enhanced verification (e.g., additional evidence checks) before any access is granted, as
captured in Figure~\ref{fig:functional_diagram}. This design ensures that a node cannot progress
from authentication into active participation unless both cryptographic proof and trust policy
conditions are satisfied.

Third, the Controller Trust Evaluation Smart Contract addresses the critical threat of controller
compromise through continuous behavioral monitoring \cite{khan2017topology}. In Figure~\ref{fig:functional_diagram},
the controller is responsible for validating incoming authentication attempts (``Controller Validates'')
and initiating blockchain lookups and policy checks. To prevent a malicious/compromised controller
from abusing this role, random subsets of nodes validate whether the controller’s reported network
topology matches actual packet flows. Discrepancies trigger trust score reductions, and if controller
trust falls below a threshold, the system automatically switches to a pre-designated backup controller,
ensuring network continuity even under insider attacks \cite{raja2020energy}.

Finally, upon successful verification and trust approval in Figure~\ref{fig:functional_diagram}, the system
issues session keys (``Issue Session Keys Encrypt with Kyber-1024'') and commits an immutable record
update (``Blockchain: Update Authentication Timestamp''). These events provide an auditable trail of
who joined, when they joined, and under what trust conditions. The same authenticated admission
event is also forwarded to the learning layer (``GNN: Add Node to Network Graph''), ensuring that the
AI components operate on blockchain-consistent, verified participation states.


\begin{figure}[H]
	\centering
	\includegraphics[width=1\linewidth]{diagrams/Functional_Diagram.png}
	\caption{Functional workflow for node registration and authentication with blockchain-backed trust evaluation.}
	\label{fig:functional_diagram}
\end{figure}

\subsubsection{AI-Driven Anomaly Detection}
While blockchain provides immutability and consensus, artificial intelligence adds intelligent pattern recognition to detect coordinated attacks that might evade rule-based systems. The vehicular network is modeled as a temporal graph where vehicles and roadside units form nodes, and communication links form edges \cite{zhou2020automating}. Graph Neural Networks (GNN) analyze this dynamic topology to identify anomalous patterns indicative of coordinated attacks.

Specifically, the GNN detects Sybil attack clusters by identifying groups of nodes with suspiciously similar behavioral patterns, identical registration timestamps, or star-topology accusation patterns where multiple accusers simultaneously target a single victim \cite{balaram2023highly}. Temporal attention mechanisms track how reputation scores change over time, flagging sudden drops that correlate with accusation floods. Recent research demonstrates that heterogeneous graph attention networks achieve over 99\% accuracy in identifying Sybil nodes in vehicular networks \cite{chen2025sybil}.

Large Language Models (LLM) complement the GNN by processing unstructured data sources such as controller logs, error messages, and system alerts \cite{zhang2024large}. The LLM performs semantic reasoning to understand causal relationships between events, for example, recognizing that a controller failover occurring minutes before a reputation spike suggests a timing-based attack. The LLM generates human-readable incident reports that explain detected attacks, identify suspected malicious nodes, assess confidence levels, and recommend mitigation actions. This interpretability is crucial for network administrators who must make final decisions on high-stakes actions like controller failover.

\subsubsection{Cryptographic Defense Mechanisms}
Cryptographic protocols form the foundation of attack prevention. Digital signatures ensure that every packet and accusation can be attributed to a specific authenticated node, preventing evidence fabrication \cite{pournaghi2020necppa}. Post-quantum cryptography algorithms protect against future quantum computing threats, particularly important for long-term infrastructure like vehicular networks. Time-stamped authentication tokens with limited validity windows prevent replay attacks where adversaries re-transmit old legitimate packets to disrupt routing \cite{benjaballah2021security}.

Location verification smart contracts validate that nodes claiming specific geographic positions are present there, preventing wormhole attacks where distant colluding nodes pretend to be neighbors \cite{quevedo2020intelligent}. Cross-validation with multiple roadside units ensures location claims are accurate, and kinematic constraints (maximum velocity and acceleration limits) detect physically impossible location changes that indicate GPS spoofing.

\subsubsection{Integrated Defense Workflow}
As illustrated in Figure~\ref{fig:simple_block}, the complete system operates as a continuous closed-loop
cycle across monitoring, validation, analysis, and response. Vehicles periodically authenticate in the
\textit{Registration and Authentication Module}, where legitimacy is verified without exposing sensitive
identity information. During each cycle, nodes exchange routing and status information through RSUs and
controllers, and any suspicious behavior observed in the field is captured by the
\textit{Location Verification and Behavioral Monitoring} component, producing the \textit{Logs and Events}
stream shown in the diagram.

When an accusation is generated, it is submitted through the controller path to the trust update process.
As indicated by the \textit{Trust Update Core} and the Fabric blocks in Figure~\ref{fig:simple_block}, each
accusation is accompanied by verifiable evidence (e.g., signatures, timestamps, and location-related proofs),
which is checked before it contributes to any trust update. Validated updates are then processed by the
Fabric trust functions: participation handling (SC1), reputation updates (SC2), and controller reliability
tracking (SC3). Accepted outcomes are committed to the \textit{Shared Ledger/World State}, ensuring that all
controllers and analysis components query a consistent trusted state.

In parallel, the \textit{AI Analysis Layer} in Figure~\ref{fig:simple_block} continuously consumes both
(i) the \textit{Logs and Events} and (ii) the Fabric-maintained \textit{Shared Ledger/World State}. The graph
analysis component evaluates the evolving network graph to identify suspicious structures and behavior patterns.
When anomalies are detected, the language-based reasoning component combines these findings with system logs to
produce threat assessments and recommended actions.

If the AI assessment and the decentralized validation outcome support the presence of an attack, the response
is applied through the \textit{Mitigation and Feedback} block (Figure~\ref{fig:simple_block}). The controllers
enforce the selected actions back to the field (e.g., restricting or isolating malicious nodes), while the final
decision and outcomes are recorded back into Fabric so that subsequent cycles build on verified history. In the
same loop, SC3 continuously tracks controller reliability to detect insider threats and support safe controller
switching when required.

This integrated design forces an adversary to overcome multiple independent protections---evidence-based validation,
decentralized agreement, trust scoring, and AI-driven anomaly detection---thereby making reputation manipulation and
coordinated attacks significantly harder to succeed while preserving operational requirements \cite{huo2023trustgnn}.


\subsection{Attack-Specific Mitigation Strategies}
The proposed methodology addresses each false accusation attack variant through targeted defense mechanisms:

\subsection{Single-Accuser Opportunistic Fabrication}

This attack occurs when a single privileged malicious node fabricates convincing evidence against legitimate vehicles. The weighted endorser trust mechanism in the Reputation Management Smart Contract prevents this by requiring accusations to be validated by multiple high-trust endorsers \cite{alshaibani2023blockchain}. A single accuser, regardless of how convincing their evidence appears, contributes only their weighted trust score to the reputation change calculation. Since blacklisting requires $\geq$2/3 weighted consensus, a lone attacker cannot unilaterally blacklist a victim unless multiple independent high-trust nodes corroborate the accusation.

Additionally, cryptographic evidence validation ensures that accusations lacking proper digital signatures, valid timestamps, or verifiable location proofs are automatically rejected. The blockchain's immutable audit logs all accusations with timestamps, allowing forensic analysis to identify nodes that frequently make false accusations, gradually reducing their trust weights to near-zero \cite{bessani2014state}.

\subsection{Sybil-Amplified Consensus Flooding}

Sybil attacks create multiple fake identities to flood the network with coordinated accusations, attempting to achieve artificial consensus. The methodology employs three defensive layers. First, the Registration Smart Contract with Zero-Knowledge Proofs ensures that each node possesses unique cryptographic credentials tied to physical vehicle identities \cite{bensasson2018scalable}. Creating Sybil identities requires either stealing legitimate credentials or registering through out-of-band channels, both of which are computationally expensive and detectable.

Second, new nodes start with minimal trust weights. Even if an attacker successfully registers multiple Sybil identities, they initially carry near-zero influence in reputation decisions. Trust accumulates gradually only through consistent honest behavior over extended periods, making instant consensus flooding ineffective \cite{balaram2023highly}.

Third, the Graph Neural Network detects coordinated attack patterns by analyzing the network topology \cite{chen2025sybil}. When multiple low-trust nodes simultaneously target a single victim, the GNN identifies this star-topology accusation pattern as anomalous. Features such as identical registration timestamps, similar behavioral patterns, and temporal clustering of accusations trigger high-confidence Sybil detection, prompting immediate investigation and potential preemptive blacklisting of the attacker cluster.

\subsection{Timing-Based Accusations During High-Noise Periods}


Sophisticated attackers exploit periods of legitimate network stress—congestion, controller handovers, or channel degradation—to mask malicious accusations as genuine failures. The temporal attention mechanism in the GNN specifically addresses this by analyzing accusation timing relative to network conditions \cite{zhou2020automating}. The system maintains historical baselines of legitimate packet loss rates during various network states (normal operation, moderate congestion, severe congestion, handover periods).

When accusations occur during high-noise periods, the LLM performs semantic reasoning to distinguish between expected failures and coordinated attacks \cite{zhang2024large}. For instance, if multiple vehicles near a congestion zone report packet loss, this correlates with legitimate network stress. However, if accusations target only specific nodes while other nearby vehicles report normal performance, this discrepancy flags a timing-based attack.

Furthermore, the weighted endorser mechanism inherently provides resilience: during genuine high-noise periods, multiple independent high-trust endorsers will naturally corroborate packet loss reports. Malicious timing-based accusations lack this broad corroboration pattern, revealing them as outliers despite the noisy environment.

\subsection{Evidence-Spoofing with Tampered Logs and Collusion}


The most sophisticated attack involves fabricating cryptographically convincing but falsified evidence—tampered packet logs, forged signatures, or collusion among multiple compromised nodes. The methodology employs multi-layer cryptographic validation as the primary defense \cite{pournaghi2020necppa}. Every accusation must include: (1) valid digital signatures from authenticated nodes, (2) unfalsifiable timestamps within acceptable time windows, (3) Zero-Knowledge Proofs demonstrating the accuser's legitimate network presence, and (4) location proofs validated by multiple independent roadside units \cite{quevedo2020intelligent}.

The blockchain's immutable chain-of-custody provides additional protection. Evidence must include traceable hashes of original packets stored on the blockchain at capture time. Retrospective evidence fabrication is impossible because the blockchain timestamp proves when data was recorded. Attackers cannot backdate evidence to match historical events \cite{sharma2017distblocknet}.

When facing collusion among multiple compromised nodes, the Controller Trust Evaluation Smart Contract becomes critical \cite{raja2020energy}. If multiple nodes collude to provide fabricated corroborating evidence, but their reports conflict with the controller's observed packet flows, the cross-endorsement validation mechanism detects this discrepancy. The system then treats the entire colluding group as suspect. Moreover, post-quantum cryptography ensures that even computationally powerful adversaries cannot forge signatures or break encryption to manufacture fake evidence \cite{bensasson2018scalable}.

In scenarios where the SDN controller itself is compromised and colluding with malicious nodes, the Controller Trust Evaluation Smart Contract continuously monitors controller behavior through random cross-endorsement validation \cite{khan2017topology}. If the controller manipulates reputation scores or routing tables to favor colluding attackers, its trust score decreases. Once trust falls below threshold, the blockchain automatically triggers failover to a pre-designated backup controller from a different administrative domain, neutralizing the insider threat.

\section{Experiments/Results. }
The proposed framework will be validated through simulation-based evaluation where all five false accusation attack variants are implemented at varying intensities from 0\% to 100\% network penetration. Performance will be measured using standard vehicular network metrics: Packet Delivery Ratio (percentage of packets successfully delivered), Packet Interception Ratio (percentage compromised by attackers), and Matthews Correlation Coefficient for attack detection accuracy \cite{nayak2021tbddosa}.

Comparative analysis against existing approaches, including traditional trust-based systems, pure cryptographic methods, and basic blockchain implementations—will demonstrate the framework's superiority in maintaining network performance under attack while correctly identifying and isolating malicious nodes. Statistical significance will be assessed through repeated trials with different network topologies and mobility patterns, ensuring results are not artifacts of specific scenarios \cite{adnan2021towards}.

\chapter{Timeline and Resource Required}

\section{Timeline}
\renewcommand{\arraystretch}{1.2}
\setlength{\tabcolsep}{3pt}

\begin{sidewaystable}[htbp]
\centering
\scriptsize

\caption{Project Timeline}
\resizebox{\textheight}{!}{%
\begin{tabular}{|p{6cm}|*{40}{c|}}
\hline

% ---------------- Year row ----------------
\multirow{2}{*}{\textbf{Task}}
& \multicolumn{8}{c|}{\textbf{2025}}
& \multicolumn{32}{c|}{\textbf{2026}} \\ \cline{2-41}

% ---------------- Month row ----------------
& \multicolumn{4}{c|}{November}
& \multicolumn{4}{c|}{December}
& \multicolumn{4}{c|}{January}
& \multicolumn{4}{c|}{February}
& \multicolumn{4}{c|}{March}
& \multicolumn{4}{c|}{April}
& \multicolumn{4}{c|}{May}
& \multicolumn{4}{c|}{June}
& \multicolumn{4}{c|}{July}
& \multicolumn{4}{c|}{August} \\ \cline{2-41}

% ---------------- Week numbers ----------------
& 1 & 2 & 3 & 4
& 1 & 2 & 3 & 4
& 1 & 2 & 3 & 4
& 1 & 2 & 3 & 4
& 1 & 2 & 3 & 4
& 1 & 2 & 3 & 4
& 1 & 2 & 3 & 4
& 1 & 2 & 3 & 4
& 1 & 2 & 3 & 4
& 1 & 2 & 3 & 4 \\ \hline

% ---------------- Tasks ----------------
Select the project topic
& \cellcolor{red} & & & 
& & & &
& & & &
& & & &
& & & &
& & & &
& & & &
& & & &
& & & &
& & & & \\ \hline

Discuss with supervisors and co-supervisors
& \cellcolor{red} & & & 
& & & &
& & & &
& & & &
& & & &
& & & &
& & & &
& & & &
& & & &
& & & & \\ \hline

Study the related work and implementation
& & \cellcolor{red} & \cellcolor{red} & \cellcolor{red}
& & &  & 
& & & &
& & & &
& & & &
& & & &
& & & &
& & & &
& & & &
& & & & \\ \hline

Prepare the project proposal and presentation
& & & & 
& \cellcolor{red} & \cellcolor{red} & & 
& & & &
& & & &
& & & &
& & & &
& & & &
& & & &
& & & &
& & & & \\ \hline

Study related technologies
& & & & 
& & & \cellcolor{red} & \cellcolor{red}
& \cellcolor{red} & & &
& & & &
& & & &
& & & &
& & & &
& & & &
& & & &
& & & & \\ \hline

Model sample LV network
& & & & 
& & & &
& & \cellcolor{red} & \cellcolor{red} & 
& & & &
& & & &
& & & &
& & & &
& & & &
& & & &
& & & & \\ \hline

Impact assessment for sample network
& & & & 
& & & &
& & & & \cellcolor{red}
& \cellcolor{red} & & &
& & & &
& & & &
& & & &
& & & &
& & & &
& & & & \\ \hline

Collect required actual data from LECO
& & & & 
& & & &
& & & &
& & \cellcolor{red} & & 
& & & &
& & & &
& & & &
& & & &
& & & &
& & & & \\ \hline

Model actual LV network using OpenDSS software
& & & & 
& & & &
& & & &
& & & \cellcolor{red} & \cellcolor{red}
& & & &
& & & & 
& & & &
& & & &
& & & &
& & & & \\ \hline

Model the demand profile
& & & & 
& & & &
& & & &
& & & & 
& \cellcolor{red} & & &
& & & &
& & & &
& & & &
& & & &
& & & & \\ \hline

Impact assessment for increasing PV penetration 
& & & & 
& & & &
& & & &
& & & &
& & \cellcolor{red} & \cellcolor{red} &
& & & &
& & & &
& & & &
& & & &
& & & & \\ \hline

Impact assessment for increasing EVCS penetration
& & & & 
& & & &
& & & &
& & & &
& & & & \cellcolor{red}
& \cellcolor{red} & & &
& & & &
& & & &
& & & &
& & & & \\ \hline

Identifying the optimization techniques
& & & & 
& & & &
& & & &
& & & &
& & & &
& & \cellcolor{red} & \cellcolor{red} &
& & & &
& & & &
& & & &
& & & & \\ \hline

Developing optimum power flow algorithm for day head market
& & & & 
& & & &
& & & &
& & & &
& & & &
& & & & \cellcolor{red}
& \cellcolor{red} & \cellcolor{red} & \cellcolor{red} &
& & & &
& & & &
& & & & \\ \hline

Developing optimum power flow algorithm for intraday market
& & & & 
& & & &
& & & &
& & & &
& & & &
& & & &
& & & & \cellcolor{red}
& \cellcolor{red} & \cellcolor{red} & \cellcolor{red} &
& & & &
& & & & \\ \hline

Integrating Blockchain technology with smart contract
& & & & 
& & & &
& & & &
& & & &
& & & &
& & & &
& & & & 
& & & & \cellcolor{red}
& \cellcolor{red} & \cellcolor{red} & \cellcolor{red} &
& & & & \\ \hline

Testing and validation of developed 
& & & & 
& & & &
& & & &
& & & &
& & & &
& & & &
& & & &
& & & &
& & & & \cellcolor{red}
& \cellcolor{red} & & & \\ \hline

Prepare final report and presentation
& & & & 
& & & &
& & & &
& & & &
& & & &
& & & &
& & & &
& & & &
& & & &
& & \cellcolor{red} & \cellcolor{red} & \cellcolor{red} \cellcolor{red} \\ \hline


\end{tabular}}
\end{sidewaystable}



\section{Resource Required}
For the implementation of the proposed blockchain, ZKP, and GNN-LLM-based defense framework for mitigating false accusation attacks in SDVN, the following tools and technologies are needed:

\begin{itemize}
    \item \textbf{Network Simulator 3 (NS-3):} A discrete-event network simulator for modeling SDVN, simulating V2V and V2I communications, and evaluating the framework under various attack scenarios.
    
    \item \textbf{Hyperledger Fabric:} A permissioned blockchain platform for implementing the distributed ledger, deploying smart contracts for registration, reputation management, and controller trust evaluation with BFT-SMaRt consensus.
    
    \item \textbf{PyTorch or TensorFlow:} Deep learning frameworks for implementing Graph Neural Networks with temporal attention to detect Sybil attacks and coordinated false accusation patterns.
    
    \item \textbf{Hugging Face Transformers:} A library for integrating Large Language Models to process controller logs and system alerts, perform semantic reasoning, and generate automated incident reports.
    
    \item \textbf{libsodium or PyCryptodome:} Cryptographic libraries for implementing post-quantum algorithms (FALCON-1024, Kyber-1024), digital signatures, key exchange, and encryption for securing communications.
    
    \item \textbf{zkSNARK Libraries (libsnark or circom):} Zero-Knowledge Proof frameworks for privacy-preserving authentication, enabling identity verification without revealing sensitive vehicle information.
    
    \item \textbf{Python with NetworkX and Pandas:} Programming environment for graph construction, feature extraction, statistical analysis, and visualization of experimental results.
\end{itemize}

These tools will be employed for the design, implementation, simulation, and evaluation of the proposed defense framework against false accusation attacks in Software-Defined Vehicular Networks.

\chapter{Conclusion}
False accusation attacks represent a critical security threat in Software-Defined Vehicular Networks, where malicious entities exploit reputation-based trust management systems to falsely accuse legitimate nodes, leading to route elimination, network instability, and degraded Quality of Service. This research addresses sophisticated attack variants including single-accuser fabrication, Sybil-amplified flooding, timing-based accusations, and evidence-spoofing through a novel multi-layered defense framework.

The proposed solution uniquely integrates blockchain technology for tamper-proof decentralized reputation logging, Zero-Knowledge Proofs for privacy-preserving verification without revealing sensitive vehicular information, and Graph Neural Networks combined with Large Language Models for intelligent anomaly detection through network topology analysis and contextual threat interpretation. The framework operates under a Zero-Trust Architecture with specialized smart contracts for registration authentication, weighted endorser trust scoring, and continuous controller trust evaluation.

Comprehensive simulation-based validation using NS-3 will evaluate the framework across varying attack intensities, measuring Packet Delivery Ratio, Packet Interception Ratio, and Matthews Correlation Coefficient. Comparative analysis will demonstrate superiority over traditional trust-based systems and basic blockchain implementations.

This research represents the first integrated application of blockchain, ZKP, and GNN-LLM technologies specifically designed to counter false accusation attacks in SDVN, enhancing the reliability and trustworthiness of intelligent transportation systems essential for future autonomous driving and smart city applications.

% References
\renewcommand{\bibname}{References}
\bibliographystyle{ieeetr}
\addcontentsline{toc}{chapter}{References} % Add to table of contents
\bibliography{bibliography} 
%\printbibliography

\end{document}
