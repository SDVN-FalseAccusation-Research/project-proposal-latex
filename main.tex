\documentclass[12pt, a4paper]{report}
\usepackage{graphicx} % For including images
\usepackage{titlesec} % For customizing section titles
\usepackage{tocloft} % For customizing table of contents
\usepackage{acro} % For acronyms
\usepackage{rotating}
\usepackage{multirow}
\usepackage[table]{xcolor}
\usepackage{array}
\usepackage{float}

%\usepackage{hyperref} % For clickable links in the document
%% ____Bibliography____%%
\usepackage[numbers,sort&compress]{natbib}
\usepackage{chapterbib}
\usepackage[breaklinks]{hyperref}
%\hypersetup{colorlinks=true,citecolor=blue,linkcolor=blue,urlcolor=blue}
% Page margins
\usepackage[left=1in, right=1in, top=1in, bottom=1in]{geometry}

% Remove page number from the first page
\thispagestyle{empty}

% Customize table of contents, list of figures, and list of tables
\renewcommand{\cfttoctitlefont}{\hfill\Large\bfseries}
\renewcommand{\cftaftertoctitle}{\hfill}
\renewcommand{\cftloftitlefont}{\hfill\Large\bfseries}
\renewcommand{\cftafterloftitle}{\hfill}
\renewcommand{\cftlottitlefont}{\hfill\Large\bfseries}
\renewcommand{\cftafterlottitle}{\hfill}

\DeclareAcronym{SDVN}{
  short = SDVN,
  long  = Software-Defined Vehicular Networks
}
\DeclareAcronym{SDN}{
  short = SDN,
  long  = Software-Defined Networking
}
\DeclareAcronym{VANET}{
  short = VANET,
  long  = Vehicular Ad Hoc Networks
}
\DeclareAcronym{V2V}{
  short = V2V,
  long  = Vehicle-to-Vehicle
}
\DeclareAcronym{V2I}{
  short = V2I,
  long  = Vehicle-to-Infrastructure
}
\DeclareAcronym{V2P}{
  short = V2P,
  long  = Vehicle-to-Pedestrian
}
\DeclareAcronym{V2N}{
  short = V2N,
  long  = Vehicle-to-Network
}
\DeclareAcronym{RSU}{
  short = RSU,
  long  = Roadside Units
}
\DeclareAcronym{OBU}{
  short = OBU,
  long  = On-Board Units
}
\DeclareAcronym{GNN}{
  short = GNN,
  long  = Graph Neural Networks
}
\DeclareAcronym{LSTM}{
  short = LSTM,
  long  = Long Short-Term Memory
}
\DeclareAcronym{ZKP}{
  short = ZKP,
  long  = Zero-Knowledge Proofs
}
\DeclareAcronym{ZKRP}{
  short = ZKRP,
  long  = Zero-Knowledge Range Proofs
}
\DeclareAcronym{BFT}{
  short = BFT,
  long  = Byzantine Fault Tolerance
}
\DeclareAcronym{API}{
  short = API,
  long  = Application Programming Interface
}
\DeclareAcronym{DSRC}{
  short = DSRC,
  long  = Dedicated Short-Range Communications
}
\DeclareAcronym{GPS}{
  short = GPS,
  long  = Global Positioning System
}
\DeclareAcronym{NTP}{
  short = NTP,
  long  = Network Time Protocol
}
\DeclareAcronym{RSA}{
  short = RSA,
  long  = Rivest-Shamir-Adleman
}
\DeclareAcronym{ECDH}{
  short = ECDH,
  long  = Elliptic Curve Diffie-Hellman
}
\DeclareAcronym{AES}{
  short = AES,
  long  = Advanced Encryption Standard
}
\DeclareAcronym{HMAC}{
  short = HMAC,
  long  = Hash-based Message Authentication Code
}
\DeclareAcronym{SHA}{
  short = SHA,
  long  = Secure Hash Algorithm
}
\DeclareAcronym{BERT}{
  short = BERT,
  long  = Bidirectional Encoder Representations from Transformers
}
\DeclareAcronym{QoS}{
  short = QoS,
  long  = Quality of Service
}


\begin{document}


\thispagestyle{empty}

\begin{center}

\begin{center}
 \includegraphics[width=2cm,keepaspectratio=true]{uor_logo.jpg}
 % uor_logo.jpg: 236x331 pixel, 72dpi, 8.33x11.68 cm, bb=0 0 236 331
\end{center}

\vspace{1.5cm}
\begin{huge}
%%%%%%%%%%%%%%%%%%%%%%%%%%%%%%%%%%%%%%%%%%%%%%%%%%%%
% Project title
%%%%%%%%%%%%%%%%%%%%%%%%%%%%%%%%%%%%%%%%%%%%%%%%%%%%
A Blockchain, ZKP, and GNN-LLM-based Combined Defense for False Accusation Attack Mitigation in Software Defined Vehicular Networks
%%%%%%%%%%%%%%%%%%%%%%%%%%%%%%%%%%%%%%%%%%%%%%%%%%%%
\end{huge} \\
\vspace{1cm}

\begin{normalsize}
An undergraduate project proposal report submitted to the
\end{normalsize}\\
\vspace{1cm}

\begin{large}
Department of Electrical and Information Engineering\\
Faculty of Engineering\\
University of Ruhuna\\
Sri Lanka
\end{large}\\

\vspace{1cm}

\begin{normalsize}in partial fulfillment of the requirements for the \end{normalsize}\\
\vspace{1cm}

\begin{large}\textbf{Degree of the Bachelor of the Science of Engineering Honours}\end{large}\\

\vspace{1cm}
\begin{normalsize}by  \end{normalsize}
\vspace{1cm}

\begin{tabular}[h]{lll}
 %%%%%%%%%%%%%%%%%%%%%%%%%%%%%%%%%%%%%%%%%%%%%%%%%%%%%%%%%%
 % Names and Registration Numbers
 %%%%%%%%%%%%%%%%%%%%%%%%%%%%%%%%%%%%%%%%%%%%%%%%%%%%%%%%%%
 M.R.M. Ashfaq	& - & 	EG/2021/4417\\
 T. Jathurshan	& - & 	EG/2021/4568\\
 M.F.A. Munsif	& - & 	EG/2021/4684\\
 M.K.M. Shamil 	& - &	EG/2021/4810
 %%%%%%%%%%%%%%%%%%%%%%%%%%%%%%%%%%%%%%%%%%%%%%%%%%%%%%%%%%
\end{tabular}\\
\vspace{1cm}

20th January 2026\\
\vspace{1cm}

%%%%%%%%%%%%%%%%%%%%%%%%%%%%%%%%%%%%%%%%%%%%%%%%%%%%%%
% if one supervisor
%%%%%%%%%%%%%%%%%%%%%%%%%%%%%%%%%%%%%%%%%%%%%%%%%%%%%%%
%  .............................................. \\
% Prof. A.B.C. Dee\\
% (Supervisor)


%%%%%%%%%%%%%%%%%%%%%%%%%%%%%%%%%%%%%%%%%%%%%%%%%%%%
% If one supervisor
%%%%%%%%%%%%%%%%%%%%%%%%%%%%%%%%%%%%%%%%%%%%%%%%%%%%
.............................................. \\
Dr. P.A.D.S.N. Wijesekara\\
(Supervisor)


\end{center}

%%%%%%%%%%%%%%%%%%%%%%%%%%%%%%%%%%%%%%%%%%%%%%%%%%%%%%%%%%%%%%%%%%%%%%%%%%%%%%%%%%%%%%%%%%%%%%%%%%
% END OF FILE
%%%%%%%%%%%%%%%%%%%%%%%%%%%%%%%%%%%%%%%%%%%%%%%%%%%%%%%%%%%%%%%%%%%%%%%%%%%%%%%%%%%%%%%%%%%%%%%%%%


\renewcommand{\thepage}{\roman{page}} % Start page numbering in roman

\chapter*{Abstract}
False accusation attacks pose a significant security challenge in Software-Defined Vehicular Networks (SDVN), where malicious entities deliberately accuse honest nodes of misbehavior to manipulate trust and reputation systems. Such attacks can severely disrupt routing decisions, degrade network performance, and undermine overall system stability. This study identifies four prominent variants of false accusation attacks: single-accuser opportunistic fabrication, Sybil-amplified consensus flooding, timing-based accusations during high-noise periods, and evidence spoofing through tampered or fabricated logs. \\ \\
To address these threats, this research proposes a novel multi-layered defense framework that integrates blockchain technology, Zero-Knowledge Proofs (ZKPs), and Graph Neural Networks combined with Large Language Models (GNN-LLMs). Blockchain is employed to ensure immutable and decentralized logging of reputation and accusation records, while ZKPs enable privacy-preserving verification of accusation authenticity without revealing sensitive vehicular information. GNN-LLMs are leveraged to model SDVN topology and behavioral relationships, enabling the detection of anomalous accusation patterns and the intelligent interpretation of complex attack behaviors. \\\\
The proposed framework is evaluated through comprehensive simulation across all identified attack variants, demonstrating its effectiveness in mitigating false accusation attacks while preserving network performance and privacy. To the best of our knowledge, this work represents the first integrated use of blockchain, ZKPs, and GNN-LLMs specifically designed to counter false accusation attacks in Software-Defined Vehicular Networks.
\newpage

% Table of Contents
\tableofcontents
\newpage

% List of Figures
\listoffigures
\newpage

% List of Tables
\listoftables
\newpage

% Acronyms
\addcontentsline{toc}{chapter}{Acronyms} % Add to table of contents
\acuseall % Use all acronyms to ensure they appear in the list
\printacronyms
\newpage

\renewcommand{\thepage}{\arabic{page}} % Start page numbering in arabic 
\setcounter{page}{1} % start page numbering from 1
\setcounter{secnumdepth}{3}

% Main Content
\chapter{Introduction}
\section{Evolution of Networking Paradigms: From SDN to SDVN}
To understand the security challenges in Software-Defined Vehicular Networks, it is essential to examine the evolution from traditional Software-Defined Networking through Vehicular Ad-hoc Networks to the integrated SDVN architecture.

\subsection{Software-Defined Networking (SDN)}
Software-Defined Networking (SDN) represents a revolutionary paradigm shift by fundamentally decoupling the control plane from the data plane \cite{godanj2016simple}. In traditional networks, both intelligence for routing decisions (control plane) and packet forwarding (data plane) reside together within network devices. SDN addresses these limitations through architectural separation where the control plane is extracted and centralized into software-based SDN controllers, while the data plane remains in simplified network devices focusing solely on packet forwarding \cite{godanj2016simple}.

The SDN architecture consists of three distinct layers:
\begin{itemize}
    \item \textbf{Application Plane:} Network applications defining desired behaviors and policies, communicating through the northbound API \cite{kreutz2014software}.
    \item \textbf{Control Plane:} SDN controller maintaining global network view, making routing decisions, and translating policies into forwarding rules \cite{kreutz2014software}.
    \item \textbf{Data Plane:} Physical and virtual network devices forwarding packets according to flow tables populated by the controller \cite{kreutz2014software}.
\end{itemize}

\begin{figure}[h!]
    \centering
    \includegraphics[width=0.85\textwidth]{diagrams/1_SDN.png}
    \caption{SDN Data Plane Architecture \cite{nunez2023brief}}
    \label{fig:sdn}
\end{figure}

\subsection{Vehicular Ad-hoc Networks (VANET)}
Vehicular Ad-hoc Networks (VANET) represent a specialized class of Mobile Ad-hoc Networks specifically designed for vehicle-to-vehicle (V2V) and vehicle-to-infrastructure (V2I) communication \cite{raja2010issues}. VANETs enable vehicles equipped with On-Board Units to communicate directly with each other and with Roadside Units deployed along roadways \cite{raja2010issues}.

Distinctive characteristics of VANET:
\begin{itemize}
    \item \textbf{High Mobility:} Vehicles move at varying speeds creating highly dynamic topology with frequent link disruptions \cite{raja2010issues}.
    \item \textbf{Dynamic Topology:} Network topology changes rapidly and unpredictably as vehicles enter, leave, or change lanes \cite{rehman2013vehicular}.
    \item \textbf{Variable Network Density:} Node density varies dramatically based on location, time, and events \cite{lee2021vanet}.
    \item \textbf{Distributed Decision Making:} Each vehicle makes local routing decisions without global network visibility \cite{raja2010issues}.
\end{itemize}

\begin{figure}[h!]
    \centering
    \includegraphics[width=0.85\textwidth]{diagrams/2_VANET.png}
    \caption{Architecture of Vehicular Ad-Hoc Networks (VANETs) \cite{rehman2013vehicular}}
    \label{fig:vanet}
\end{figure}

\subsection{Software-Defined Vehicular Networks (SDVN)}
Software-Defined Vehicular Networks (SDVN) emerge as a convergence architecture integrating the programmability and centralized control of SDN with the mobility and distributed communication of VANET \cite{correia2017architecture}. SDVN addresses fundamental VANET limitations—particularly distributed routing in highly dynamic topologies—by introducing centralized intelligence while maintaining vehicle-to-vehicle communication \cite{li2016control}.

SDVN architectural components:
\begin{itemize}
    \item \textbf{SDN-Enabled Vehicles:} Vehicles as mobile SDN switches querying controllers for forwarding instructions \cite{li2016control}.
    \item \textbf{SDN-Enabled RSUs:} Hybrid devices for packet forwarding and controller-to-vehicle communication aggregation \cite{li2016control}.
    \item \textbf{Hierarchical Controllers:} Local controllers for region-specific routing with global controller coordination \cite{correia2017architecture}.
    \item \textbf{Hybrid Control:} Time-critical safety messages via direct V2V; non-urgent traffic via centralized SDN \cite{dhawankar2017software}.
\end{itemize}

\begin{figure}[h!]
    \centering
    \includegraphics[width=0.85\textwidth]{diagrams/3_SDVN.png}
    \caption{Architecture of Software-Defined Vehicular Networks (SDVN) \cite{hama2025security}}
    \label{fig:sdvn}
\end{figure}

\subsection{Security Challenges in SDVN}
Software-Defined Vehicular Networks inherit security vulnerabilities from both Software-Defined Networking and Vehicular Ad-hoc Networks, while simultaneously introducing new attack surfaces due to their centralized control architecture and highly dynamic vehicular environment. Although SDVN improves network flexibility and routing efficiency, the tight coupling between mobile data-plane entities and centralized control logic increases the overall attack impact when security assumptions are violated \cite{arif2020sdn}.

One of the core security challenges in SDVN is maintaining reliable decision-making under highly dynamic conditions. Frequent topology changes, intermittent connectivity, and variable node density make it difficult to obtain accurate and consistent network state information \cite{arif2020sdn}. As a result, control-plane decisions may be based on incomplete, delayed, or noisy inputs, reducing the effectiveness of conventional security validation mechanisms \cite{arif2020sdn}.

The reliance on centralized or hierarchical controllers further complicates the security landscape \cite{hama2025security}. Controllers aggregate large volumes of network information and enforce global policies, making them attractive targets for compromise. Any disruption, manipulation, or failure at the control plane can propagate rapidly across the network, affecting routing stability, quality of service, and overall system reliability \cite{hama2025security}.

Additionally, SDVN must balance security enforcement with privacy preservation. Vehicles continuously exchange sensitive operational and contextual information, yet excessive disclosure of such data violates privacy requirements and regulatory constraints. Existing security solutions often struggle to simultaneously ensure data integrity, system robustness, and privacy protection in large-scale vehicular environments \cite{hama2025security}. These unresolved challenges highlight the need for advanced security mechanisms capable of supporting trustworthy operation in SDVN, thereby motivating the problem addressed in the following section.

\section{Problem Statement: False Accusation Attacks in SDVN}
Trust and reputation systems are essential for secure routing and decision-making in Software-Defined Vehicular Networks. These systems rely on reports and behavioral evidence submitted by vehicles and RSUs to identify malicious participants. However, this reliance creates a critical vulnerability: malicious entities can fabricate or manipulate accusations to falsely label honest nodes as attackers.

False accusation attacks undermine the fundamental assumption that reports submitted to the controller are truthful. When exploited, these attacks cause legitimate vehicles to be blacklisted, trusted routes to be eliminated, and network traffic to be redirected through adversarial paths. Over time, this results in degraded network performance, biased routing decisions, and erosion of trust in the SDVN control framework.

This research addresses false accusation attacks under two distinct attacker models:
\begin{itemize}
    \item \textbf{Data-plane attacker model:} Malicious vehicles or RSUs generate false accusations while the SDN controller remains honest but vulnerable to deception.
    \item \textbf{Control-plane attacker model:} The SDN controller itself is malicious or compromised, enabling it to fabricate, amplify, or manipulate accusation records internally without relying on genuine vehicle reports.
\end{itemize}

Under these assumptions, four major variants of false accusation attacks are identified and addressed:

\subsection{Single-Accuser Opportunistic Fabrication}
In this attack, a single malicious vehicle falsely accuses a nearby honest node of misbehavior during transient network conditions such as packet loss or congestion.
\begin{itemize}
    \item \textbf{Data-plane version:} A malicious vehicle exploits momentary failures to submit fabricated reports against an honest vehicle.
    \item \textbf{Control-plane version:} A malicious controller directly injects fabricated accusation records into the reputation system, attributing them to legitimate vehicles.
\end{itemize}

\subsection{Sybil-Amplified Consensus Flooding}
This attack amplifies false accusations by using multiple forged identities to create artificial consensus.
\begin{itemize}
    \item \textbf{Data-plane version:} A malicious vehicle creates or controls multiple Sybil identities, each submitting coordinated accusations against a target node.
    \item \textbf{Control-plane version:} The controller internally generates phantom vehicle identities and uses them to simulate widespread agreement, forcing the blacklisting of honest nodes.
\end{itemize}

\subsection{Timing-Based Accusations During High-Noise Periods}
Here, attackers exploit periods of high mobility, interference, or network congestion when verification is difficult.
\begin{itemize}
    \item \textbf{Data-plane version:} Malicious vehicles submit accusations during peak traffic or handover events, masking falsified claims within legitimate noise.
    \item \textbf{Control-plane version:} A malicious controller selectively issues accusations during known unstable periods to minimize detection.
\end{itemize}

\subsection{Evidence-Spoofing with Tampered Logs and Collusion}
This attack targets the integrity of behavioral evidence.
\begin{itemize}
    \item \textbf{Data-plane version:} Colluding vehicles submit manipulated logs or selectively omit information to support false accusations.
    \item \textbf{Control-plane version:} The controller alters or fabricates historical logs and metrics, presenting them as authentic evidence of misbehavior.
\end{itemize}

These attack variants demonstrate that false accusation attacks are not limited to distributed adversaries but remain effective even when centralized control components are compromised. Therefore, a defense mechanism must be tamper-resistant, privacy-preserving, and capable of detecting structural and behavioral anomalies across both planes.

\section{Objectives and Scope}
\subsection{Objectives}
The primary objective of this research is to design, implement, and evaluate a comprehensive defense framework for mitigating false accusation attacks in Software-Defined Vehicular Networks. To achieve this overarching goal, the specific objectives of the study are as follows:
\begin{enumerate}
    \item To analyze and model false accusation attacks in SDVN, focusing on their impact on trust and reputation management systems under dynamic vehicular conditions.
    \item To design a blockchain-based reputation management mechanism that ensures tamper-proof logging, distributed validation, and resilience against manipulation of accusation records.
    \item To develop Zero-Knowledge Proof–based verification methods that enable the validation of accusation authenticity and node behavior without revealing sensitive vehicular or contextual information.
    \item To construct Graph Neural Network models for representing SDVN topology and relational interactions, enabling detection of anomalous accusation patterns and coordinated attacks.
    \item To integrate Large Language Models with graph-based insights to interpret complex attack behaviors, analyze contextual evidence, and support intelligent mitigation decisions.
    \item To evaluate the effectiveness of the proposed multi-layered defense framework in terms of attack detection accuracy, false positive reduction, network performance, and Quality of Service preservation.
    \item To validate the proposed solution through simulation in a Software-Defined Vehicular Network environment using NS-3 and a blockchain platform such as Hyperledger Fabric.
\end{enumerate}

\subsection{Scope}
The scope of this research is defined to ensure focused investigation and practical feasibility. The study concentrates on the following aspects:
\begin{itemize}
    \item The research is limited to false accusation attacks targeting trust and reputation systems in Software-Defined Vehicular Networks.
    \item Both data-plane adversaries (malicious vehicles and RSUs) and control-plane adversaries (compromised or malicious SDN controllers) are considered within the threat model.
    \item The proposed framework focuses on the integration of blockchain, Zero-Knowledge Proofs, Graph Neural Networks, and Large Language Models as core defensive components.
    \item Evaluation is conducted using simulation-based experiments rather than real-world vehicular deployments, with performance metrics including detection accuracy, reputation stability, routing efficiency, and QoS impact.
    \item The scope excludes physical-layer attacks (e.g., jamming), traditional cryptographic key management protocols, and non-reputation-based network attacks that do not involve accusation manipulation.
\end{itemize}

This defined scope ensures that the research remains targeted toward developing a robust, privacy-preserving, and intelligent defense mechanism for false accusation attacks while maintaining relevance to real-world SDVN deployments.

\chapter{Literature Review}
\section{Previous Works}

\subsection{Software-Defined Vehicular Networks (SDVN)}
Vehicular Ad Hoc Networks (VANETs) represent a specialized class of Mobile Ad Hoc Networks (MANETs) characterized by high node mobility, rapidly changing network topology, and highly unstable communication environments, supporting various communication patterns including Vehicle-to-Vehicle (V2V), Vehicle-to-Infrastructure (V2I), Vehicle-to-Pedestrian (V2P), and Vehicle-to-Network (V2N) \cite{adnan2021towards}, \cite{cardona2020software}. Traditional VANETs face significant challenges including limited scalability, insufficient flexibility, lack of programmability, and inefficient traffic management, often resulting in network congestion, reduced throughput, and compromised routing efficiency \cite{adnan2021towards}. To address these limitations, Software-Defined Vehicular Networks (SDVN) emerged by integrating SDN principles—specifically the separation of the control plane from the data plane—into vehicular networking environments, enabling centralized control, enhanced programmability, and improved traffic management capabilities critical for maintaining network trust and stability \cite{adnan2021towards}, \cite{cardona2020software}. This architectural transformation is particularly significant for implementing reputation-based systems and trust mechanisms, as the centralized SDN controller can maintain global network views and coordinate security policies across distributed vehicular nodes \cite{cardona2020software}.

SDVN architectures comprise critical components including On-Board Units (OBUs) enabling vehicle communications, Road Side Units (RSUs) functioning as infrastructure gateways, Tamper Proof Devices (TPDs) providing secure storage for vehicle credentials and keys, and Trusted Authorities managing authentication and trust relationships \cite{adnan2021towards}. The literature identifies hierarchical SDVN architectures where control planes are divided into upper and lower tiers, with main SDN controllers maintaining global oversight and sub-controllers managing local network segments, facilitating secure traffic flow management between V2V, V2R, and V2I communications \cite{adnan2021towards}, \cite{cardona2020software}. However, this centralized architecture introduces specific vulnerabilities that can be exploited to manipulate network trust and routing decisions, as compromised controllers or RSUs can affect entire network segments under their control \cite{adnan2021towards}. Research demonstrates that while SDVN implementations achieve significant performance improvements—up to 84\% reduction in controller overhead and 60\% reduction in network bandwidth consumption—the centralized control plane becomes a critical point for security attacks affecting routing efficiency, Quality of Service (QoS), and overall network stability \cite{cardona2020software}.

Despite its advantages, SDVN faces critical security challenges across multiple threat vectors: man-in-the-middle attacks on communication channels, data plane vulnerabilities enabling injection of fake traffic flows, RSU vulnerabilities that allow attackers to cause network disorder through Denial of Service attacks, control plane communication attacks including Distributed Denial of Service that generate excessive controller queries causing service delays, and SDN controller vulnerabilities such as identity spoofing enabling malicious actors to impersonate legitimate controllers and manipulate network operations \cite{adnan2021towards}. These vulnerabilities create opportunities for sophisticated attacks targeting reputation systems and trust mechanisms, where malicious nodes can exploit centralized trust management to falsely accuse legitimate nodes, manipulate routing decisions, and compromise network integrity \cite{adnan2021towards}. Critical research gaps exist in comprehensive defense mechanisms that integrate decentralized verification with advanced technologies to mitigate reputation-based attacks, particularly those involving coordinated malicious behavior, fabricated evidence, and collusion with compromised infrastructure components \cite{adnan2021towards}, \cite{cardona2020software}. While existing research has proposed various security mechanisms for V2V and V2I communications, there remains a significant gap in integrated approaches that combine tamper-proof reputation logging, privacy-preserving verification, and intelligent anomaly detection to robustly address false accusation attacks that manipulate trust systems, eliminate legitimate routes, degrade QoS, and destabilize control-plane operations in dynamic SDVN environments.

\subsection{False Accusation Attacks in Vehicular Networks}
False accusation attacks represent a critical threat in vehicular networks where malicious nodes fabricate false alert messages accusing normal nodes as misbehaving nodes, causing rapid network performance degradation \cite{lee2012efficient}. Simulation results demonstrate that false accusation attacks produce faster throughput reduction over time compared to selfish nodes, affecting network performance more dramatically as false accusation nodes can exclude legitimate participants from routing processes, thereby degrading overall network efficiency and survivability \cite{lee2012efficient}. In reputation-based intrusion detection systems operating in distributed and cooperative network structures, approaches that detect abnormal activities based on individual node-created reputations are inherently vulnerable to false accusations, particularly when nodes conspire together to signify normal nodes as intruders or when compromised nodes report wrong reputation scores to neighboring nodes \cite{lee2012efficient}. The fundamental attack mechanism involves malicious nodes on communication paths falsely reporting that normal nodes failed to deliver data to destination nodes, although these accused nodes actually forwarded packets correctly; when source nodes fail to receive acknowledgment messages within certain timeframes due to malicious nodes discarding them, they mistakenly identify normal nodes as abnormal based on false reports, leading to legitimate node exclusion from the network \cite{lee2012efficient}. Existing research identifies critical limitations where most studies do not consider comprehensive response methods for false accusations, and intelligent attackers performing abnormal activities while maintaining misbehaviors below predefined thresholds evade detection, allowing attacks to continue undetected \cite{lee2012efficient}, \cite{gyawali2020machine}. Vehicular networks face diverse attack variants including false alert generation attacks, position falsification attacks using GPS spoofing techniques or Sybil attacks, and trust-distortion attacks that manipulate trust management mechanisms tricking nodes into accepting inaccurate reliability estimates \cite{che2022trust}, \cite{gyawali2020machine}.

The consequences of false accusation attacks severely impact network functionality through multiple mechanisms including false path elimination where legitimate nodes are blacklisted removing valid routes, route concentration toward attacker-controlled nodes compromising routing integrity, Quality of Service degradation with increased packet loss and delays, and control-plane instability from frequent trust score fluctuations \cite{lee2012efficient}. Experimental evidence confirms that while cryptographic methods provide protection against external attacks, they remain vulnerable to internal attacks where authenticated vehicles execute malicious behavior; protocols like CONFIDANT that detect abnormal activities without considering false accusations show initial performance improvements but experience rapid performance degradation as normal nodes are excluded due to malicious false accusations over time, demonstrating fatal vulnerabilities despite capability in detecting simple abnormal activities \cite{lee2012efficient}, \cite{gyawali2020machine}. Critical research gaps persist in comprehensive defense mechanisms integrating decentralized verification, tamper-proof evidence logging, privacy-preserving accusation verification, and intelligent anomaly detection to robustly mitigate sophisticated false accusation attack patterns including single-accuser opportunistic fabrications, Sybil-amplified consensus flooding, timing-based accusations during high-noise periods, evidence-spoofing with tampered logs, and collusion with compromised infrastructure components in dynamic vehicular network environments that compromise network trust, routing efficiency, and overall stability \cite{lee2012efficient}.

\subsection{Blockchain Technology in Vehicular Networks}
Blockchain technology has emerged as a promising solution for addressing trust management challenges in vehicular networks due to its significant characteristics including consistency, fault tolerance, decentralization, and tamper-proof properties \cite{chang2025blockchain}. Blockchain serves as an immutable distributed ledger where vehicular communication data, trust evaluations, and reputation information can be verified and traced publicly, with the consensus mechanism contributing to establishing trust among vehicles while overcoming performance bottlenecks associated with centralized trust management systems \cite{li2020blockchain}. The decentralized nature of blockchain addresses critical limitations in traditional vehicular networks, particularly the difficulty of establishing trusted central servers to save and update historical trust information of all vehicles in real-time across wide geographic scopes with high mobility, while ensuring recorded trust information maintains data consistency and tamper-resistant properties \cite{li2020blockchain}. Various blockchain architectures for vehicular networks have been proposed, including systems maintaining multiple blockchains such as certificate blockchains (CerBC) for vehicle identity verification and request blockchains (ReqBC) for recording location-based service queries and cooperation requests, enabling Registration Authorities to manage vehicle certificates, track malicious vehicles, and initiate dispute arbitration when behaviors cannot be determined as malicious \cite{gazdar2022decentralized}, \cite{li2020blockchain}.

Research demonstrates that blockchain-based trust management schemes for vehicular networks adopt different architectural approaches addressing specific security and performance challenges \cite{gazdar2022decentralized}, \cite{chang2025blockchain}. Decentralized blockchain-based frameworks compute global trust metrics for each vehicle stored in immutable ledgers, typically operating through three phases: trust metrics evaluation where vehicles assess authenticity of received messages and calculate local trust metrics for originator vehicles, trust metrics aggregation where miners collect and aggregate received trust metrics to compute aggregated values, and blocks generation and validation where aggregated trust metrics are packed into blocks and added to the blockchain after solving consensus mechanisms like Proof-of-Work \cite{gazdar2022decentralized}. However, existing blockchain schemes exhibit limitations including trust evaluation relying solely on geographic position proximity to events which is insufficient for determining credibility as malicious vehicles near events can generate false messages, underutilization of blockchain capabilities with excessive network structure complexity, and dependence on centralized authorities like Law Enforcement Agencies for recording public key-identity pairs leading to scalability challenges \cite{gazdar2022decentralized}. Some approaches assign trust metric assessment to RSUs causing delays in global trust metrics convergence and sharing, while others lack clear explanation of trust metrics calculation processes and factors considered for assessing vehicle trustworthiness \cite{gazdar2022decentralized}.

Blockchain integration with vehicular networks enables enhanced security capabilities including distributed reputation management, anonymous authentication, and collaborative intrusion detection \cite{gazdar2022decentralized}, \cite{chang2025blockchain}. The tamper-proof nature of blockchain allows secure storage of reputation scores, certificate management including registration, updating, and revocation, and provides proof of certificate existence with capabilities for tracking malicious vehicles through dispute arbitration mechanisms \cite{li2020blockchain}. Recent developments propose deep learning-driven blockchain-based trust management systems combining physical unclonable function-based identity authentication, Bayesian-based message reliability computation, and optimized consensus mechanisms adapted for vehicular mobility, offering multi-faceted trust management architectures that enhance trust evaluation fidelity and detection of nuanced attack types \cite{chang2025blockchain}. However, critical research gaps remain in developing comprehensive blockchain frameworks that effectively integrate tamper-proof reputation logging with privacy-preserving verification mechanisms, intelligent anomaly detection for identifying coordinated malicious behaviors including false accusation attacks, and efficient consensus protocols suitable for highly dynamic vehicular environments with frequent topology changes, particularly for mitigating sophisticated attacks involving fabricated evidence, Sybil-amplified consensus flooding, and collusion with compromised infrastructure components while maintaining real-time performance requirements for safety-critical applications \cite{gazdar2022decentralized}, \cite{chang2025blockchain}, \cite{li2020blockchain}.

\subsection{Zero-Knowledge Proofs (ZKP)}
Zero-Knowledge Proof (ZKP) is a cryptographic technique enabling one party (the prover) to demonstrate the validity of a statement to another party (the verifier) without revealing any additional information beyond the proof's validity \cite{zhou2024leveraging}. ZKPs play a crucial role in enhancing privacy and security in vehicular networks by allowing vehicles to prove attributes such as identity credentials, authorization, or specific information possession without disclosing sensitive data that could compromise privacy \cite{zhou2024leveraging}. Two main categories of ZKP technology exist: Interactive ZKP (Inter-ZKP) where the prover and verifier engage in iterative exchange protocols to establish statement validity without disclosing sensitive information, and Non-Interactive ZKP where the prover can authenticate information without requiring back-and-forth communication with the verifier \cite{zhou2024leveraging}. Advanced ZKP implementations including Zero-Knowledge Succinct Non-Interactive Arguments of Knowledge (zk-SNARKs) and Zero-Knowledge Scalable Transparent Arguments of Knowledge (zk-STARKs) have emerged offering compact proofs independent of computation size, strong privacy guarantees, post-quantum security resistance, and efficient verification capabilities particularly valuable for blockchain and cryptocurrency applications \cite{lavin2024survey}, \cite{zhou2024leveraging}. In vehicular data sharing systems, zk-SNARKs protect privacy during authentication procedures by enabling vehicles to prove identity attributes or credential ownership without revealing underlying sensitive information, with research demonstrating their effectiveness in concealing registration and attribute information while maintaining secure authentication processes \cite{zhou2024leveraging}, \cite{kalmykov2022using}. Applications of ZKP in vehicular networks address critical privacy challenges including identity authentication, location privacy preservation, and secure credential verification without trusted intermediaries, with zero-knowledge range proofs (ZKRP) enabling validation of vehicle locations without revealing precise coordinates and self-blindable signature schemes enabling anonymous key exchange protocols where vehicles compute anonymous shared keys based on zero-knowledge proof of knowledge, achieving secure authentication, forward unlinkability, and accountability without requiring interference from Roadside Units or Certificate Authorities \cite{zhou2024leveraging}, \cite{kalmykov2022using}, \cite{almarshoud2022location}.

Despite significant advantages, ZKP-enabled systems face challenges including scalability concerns as blockchain networks grow, computational overhead for proof generation particularly in resource-constrained vehicular environments, complexity requiring cryptographic expertise for implementation and validation, trusted setup requirements in some implementations that could be compromised, and performance overhead that may not be justified for small computations \cite{lavin2024survey}, \cite{zhou2024leveraging}. Adaptive zero-knowledge authentication protocols in VANETs provide high imitation resistance without using symmetric and asymmetric ciphers, reducing time spent on authentication by decreasing the number of modular exponentiation operations and reducing the number of responses while calculating true and falsified digests of the prover and verifying response correctness \cite{kalmykov2022using}. Critical research gaps persist in developing efficient ZKP mechanisms specifically optimized for highly dynamic vehicular environments with frequent topology changes, integrating ZKP with decentralized verification systems for validating accusation authenticity in reputation-based trust management, combining ZKP with blockchain for tamper-proof evidence logging while maintaining real-time performance requirements for safety-critical applications, and addressing emerging threats and attack vectors as ZKP-enabled systems become more prevalent in vehicular networks \cite{lavin2024survey}, \cite{zhou2024leveraging}. Furthermore, limited research exists on utilizing ZKP for privacy-preserving verification of node behavior and accusation authenticity in defense mechanisms against sophisticated false accusation attack patterns including evidence-spoofing with tampered logs, Sybil-amplified consensus flooding, and collusion with compromised infrastructure components, particularly in scenarios requiring verifiable proofs without revealing sensitive vehicle operational data or communication patterns that could be exploited by adversaries in dynamic SDVN environments \cite{zhou2024leveraging}, \cite{almarshoud2022location}.

\subsection{Artificial Intelligence in Vehicular Networks}
Machine learning and deep learning techniques have emerged as critical tools for detecting sophisticated attacks in vehicular networks, particularly for identifying anomalous patterns in reputation systems and trust management schemes vulnerable to false accusation attacks \cite{zoubir2024integrating}, \cite{chattopadhyay2024gnn}, \cite{belcastro2025enhancing}. Graph Neural Networks (GNNs) represent a breakthrough technology for analyzing complex network structures in Software-Defined Vehicular Networks (SDVN), leveraging message passing mechanisms that aggregate information from neighboring nodes to understand relational patterns and identify anomalous accusation clusters that deviate from normal network behavior \cite{zoubir2024integrating}, \cite{chattopadhyay2024gnn}. Self-supervised GNN-based approaches such as GraphSAGE and Anomal-E have demonstrated superior performance in network intrusion detection by analyzing network topology and edge features, with GraphSAGE introducing inductive learning capabilities through neighbor sampling mechanisms that enable efficient processing of dynamic vehicular network scenarios where topology changes frequently due to vehicle mobility \cite{zoubir2024integrating}, \cite{chattopadhyay2024gnn}. Research demonstrates GNN-based models achieving macro F1 scores exceeding 93\% in anomaly detection tasks by enriching network flow data with relevant features and representing traffic as graphs, allowing GNNs to learn complex relationships between nodes and detect coordinated malicious behaviors including Sybil attacks, collusive tampering, and reputation manipulation attempts where multiple nodes conspire to fabricate false accusations against legitimate participants \cite{zoubir2024integrating}, \cite{chattopadhyay2024gnn}, \cite{belcastro2025enhancing}. Deep learning approaches incorporating attention mechanisms and adversarial training have shown effectiveness in adapting to sophisticated attack patterns, with federated learning frameworks enabling privacy-preserving model training across distributed vehicular nodes without requiring centralized data collection, addressing scalability concerns in large-scale SDVN deployments \cite{belcastro2025enhancing}.

Large Language Models (LLMs) including GPT-4, BERT-based architectures, and transformer models have demonstrated remarkable capabilities in cybersecurity applications through advanced contextual understanding and natural language processing of heterogeneous security data including system logs, network flows, and threat reports. The integration of GNN-LLM frameworks represents a promising approach for comprehensive defense against false accusation attacks in vehicular networks, where GNNs analyze network topology to identify anomalous accusation patterns and structural relationships while LLMs process graph-derived insights combined with natural language logs for contextual threat assessment, automated response generation, and interpretation of complex attack scenarios involving fabricated evidence and tampered communication records. Transformer-based architectures capture temporal and structural relationships in dynamic network environments, enabling detection of timing-based accusations during high-noise periods and evidence-spoofing attacks where attackers fabricate traceroute-like evidence to appear authentic. BERT-based models fine-tuned for security applications have shown effectiveness in parsing and classifying security-relevant textual data, with potential applications in analyzing accusation messages, verifying claim authenticity, and identifying linguistic patterns associated with malicious false reporting in reputation systems. However, critical challenges persist including scalability concerns for real-time processing in resource-constrained vehicular environments, computational overhead of combined GNN-LLM architectures, bias in training data leading to under-representation of novel false accusation attack variants, and requirements for extensive validation in dynamic SDVN scenarios with frequent topology changes and high mobility. Critical research gaps exist in developing integrated GNN-LLM frameworks that combine network topology analysis with contextual threat interpretation for detecting sophisticated false accusation attacks involving single-accuser opportunistic fabrications, Sybil-amplified consensus flooding, collusion with compromised infrastructure components, and coordinated reputation manipulation, while maintaining real-time performance requirements, privacy preservation, and ensuring verifiable trust mechanisms in distributed vehicular network architectures prioritizing routing efficiency and QoS stability.

\section{Gaps in Literature}
\subsection{Limited Integration of Advanced Technologies}
Existing research in vehicular network security predominantly focuses on single-technology solutions that address isolated aspects of false accusation attack mitigation. Blockchain-based approaches have demonstrated effectiveness in providing tamper-proof reputation logging and distributed consensus mechanisms for trust management, with several studies implementing decentralized trust evaluation systems where RSUs or miners aggregate trust metrics received from vehicles and store them in immutable ledgers \cite{gazdar2022decentralized}, \cite{chang2025blockchain}, \cite{li2020blockchain}. However, these blockchain-centric solutions often lack sophisticated attack detection mechanisms beyond basic reputation score tracking and threshold-based anomaly identification. Similarly, reputation-based intrusion detection systems have been extensively studied, with approaches utilizing node weight management, compensation algorithms, and ElGamal cryptography to address false accusations through detection algorithms that identify malicious reporting patterns \cite{lee2012efficient}. Machine learning-based misbehavior detection systems, including those employing Dempster-Shafer theory for collaborative detection and beta distribution-based reputation updates, have shown improved accuracy in identifying malicious vehicles but operate independently without integration with decentralized verification mechanisms \cite{gyawali2020machine}. Zero-knowledge proof applications in vehicular networks have primarily focused on privacy-preserving authentication and anonymous credential systems, with implementations using zk-SNARKs for identity protection during authentication procedures, yet these remain disconnected from comprehensive reputation management and attack detection frameworks \cite{zhou2024leveraging}, \cite{kalmykov2022using}, \cite{almarshoud2022location}.

The critical gap lies in the absence of comprehensive frameworks that synergistically combine blockchain for tamper-proof evidence logging and distributed validation, Zero-Knowledge Proofs for privacy-preserving verification of node behavior and accusation authenticity, and advanced AI techniques including Graph Neural Networks and Large Language Models for intelligent anomaly detection and contextual threat interpretation. While individual technologies have matured independently, no existing work presents a unified multi-layered security architecture that leverages the complementary strengths of blockchain's immutability, ZKP's privacy preservation, GNN's topology analysis capabilities, and LLM's contextual understanding to create a holistic defense against sophisticated false accusation attacks in Software-Defined Vehicular Networks. This fragmentation prevents the development of robust systems capable of simultaneously ensuring verifiable trust through cryptographic proofs, maintaining privacy of sensitive vehicular data, detecting complex coordinated attack patterns through network topology analysis, and interpreting multi-source evidence through natural language processing, ultimately leaving SDVN environments vulnerable to advanced false accusation attack variants that exploit the limitations of isolated single-technology defenses.

\subsection{Privacy vs. Verification Trade-off}
Current reputation-based trust management systems in vehicular networks face a fundamental tension between the need for transparent verification of accusations and the imperative to protect sensitive vehicular data from exposure. Existing approaches typically require nodes to reveal detailed behavioral information including packet forwarding records, communication logs, routing decisions, and network interaction histories to enable verification of misbehavior reports and calculation of trust scores \cite{lee2012efficient}, \cite{gyawali2020machine}, \cite{gazdar2022decentralized}. Blockchain-based trust evaluation frameworks aggregate trust metrics by having vehicles share local trust assessments with RSUs or miners, who then compute global trust values and store them in distributed ledgers, inherently exposing vehicle behavior patterns and reputation judgments to infrastructure components and potentially to other network participants \cite{gazdar2022decentralized}, \cite{chang2025blockchain}, \cite{li2020blockchain}. Collaborative misbehavior detection systems employing techniques like Dempster-Shafer theory require vehicles to exchange reputation scores and feedback about suspected malicious nodes, creating privacy risks where vehicular communication patterns, mobility behaviors, and social trust relationships become visible to observers who could infer sensitive information about vehicle routes, destinations, and driver identities \cite{gyawali2020machine}. Traditional cryptographic authentication schemes using Public Key Infrastructure enable identity verification but do not address the privacy concerns inherent in reputation systems where the content of accusations, evidence supporting misbehavior claims, and historical behavior records must be evaluated for authenticity, forcing a choice between transparency for verification and confidentiality of operational data \cite{kalmykov2022using}, \cite{almarshoud2022location}.

The research gap centers on insufficient integration of privacy-preserving cryptographic techniques, particularly Zero-Knowledge Proofs, into reputation management and false accusation detection systems for vehicular networks. While ZKP applications in VANETs have demonstrated effectiveness in privacy-preserving authentication, anonymous credential verification, and location privacy protection through techniques like zero-knowledge range proofs and self-blindable signatures \cite{zhou2024leveraging}, \cite{kalmykov2022using}, \cite{almarshoud2022location}, these implementations have not been extended to enable verification of accusation authenticity and node behavior validation without revealing the underlying sensitive data. Existing systems lack mechanisms to generate verifiable proofs that demonstrate: (1) an accused node actually forwarded packets correctly without disclosing packet contents or routing paths, (2) an accuser possesses legitimate evidence of misbehavior without exposing complete communication logs or network topology information, (3) behavioral patterns match expected legitimate operation without revealing precise locations, velocities, or destinations, and (4) reputation scores were calculated correctly without exposing individual trust assessments or social network relationships. This gap leaves SDVN environments unable to resolve false accusation attacks while simultaneously preserving privacy of vehicular operational data including location traces, communication patterns, routing decisions, and trust relationships, ultimately forcing system designers to compromise either security through reduced verification rigor or privacy through excessive data exposure, neither of which is acceptable for safety-critical intelligent transportation applications where both verifiable trust and data confidentiality are paramount requirements.

\subsection{Incomplete Detection of Sophisticated Attack Variants}
Existing defense mechanisms for reputation-based trust management in vehicular networks demonstrate capability in detecting basic misbehavior patterns such as simple packet dropping, straightforward false reporting by individual malicious nodes, and rudimentary reputation manipulation where single attackers provide consistently incorrect trust assessments \cite{lee2012efficient}, \cite{gyawali2020machine}. Reputation-based intrusion detection techniques employing node weight management and suspicious node lists can identify nodes that repeatedly make false accusations, using mechanisms like NWMS (Node Weight Management Server) to track accusation accuracy and exclude nodes whose accusations are frequently proven false through cross-verification with destination nodes \cite{lee2012efficient}. Collaborative misbehavior detection systems applying machine learning classifiers and Dempster-Shafer theory achieve improved detection accuracy by combining multiple observers' assessments and using reputation scores as belief values for feedback aggregation, enabling identification of vehicles broadcasting false alerts or position falsification attacks \cite{gyawali2020machine}. However, these approaches rely primarily on statistical anomaly detection, threshold-based filtering, and historical pattern matching that assume malicious behaviors exhibit consistent detectable signatures across time, making them effective against unsophisticated attacks but vulnerable to intelligent adversaries who carefully craft attack strategies to evade detection mechanisms.

Critical gaps exist in detecting and mitigating sophisticated false accusation attack variants that exploit weaknesses in current defense systems. Sybil-amplified consensus flooding attacks, where adversaries create multiple fake identities to generate artificial consensus supporting false accusations and overwhelm reputation systems with coordinated malicious reports, remain largely unaddressed as existing blockchain-based trust management systems lack robust Sybil resistance mechanisms and cannot effectively distinguish between legitimate consensus from multiple independent observers versus fabricated agreement from colluding Sybil nodes \cite{che2022trust}, \cite{lee2012efficient}, \cite{gyawali2020machine}, \cite{gazdar2022decentralized}. Timing-based accusations during high-noise periods exploit temporal vulnerabilities by launching false accusation campaigns when network conditions naturally produce high packet loss, communication failures, or routing disruptions, allowing malicious accusations to blend with legitimate error reports and evade detection systems that cannot differentiate between actual misbehavior and network-induced failures under challenging conditions \cite{lee2012efficient}. Evidence-spoofing attacks involving tampered logs, fabricated traceroute-like data, and manipulated packet forwarding records present verification challenges that current systems cannot address since they lack cryptographic integrity protection for behavioral evidence and cannot validate that submitted proof of misbehavior is authentic rather than artificially constructed by sophisticated attackers \cite{lee2012efficient}, \cite{gyawali2020machine}. Collusion attacks involving compromised Roadside Units, controllers, or groups of malicious vehicles coordinating their accusations represent the most dangerous variant, as existing distributed trust evaluation systems assume infrastructure honesty and cannot detect scenarios where RSUs deliberately validate false accusations, controllers manipulate routing decisions based on fabricated reputation data, or coordinated vehicle groups strategically time their false reports to maximize damage against targeted legitimate participants \cite{lee2012efficient}, \cite{gazdar2022decentralized}, \cite{chang2025blockchain}. These sophisticated attack variants remain insufficiently addressed in current literature, leaving SDVN environments vulnerable to coordinated, intelligent false accusation campaigns that can systematically eliminate legitimate nodes, concentrate routing toward attacker-controlled paths, degrade Quality of Service through strategic blacklisting, and destabilize control-plane trust management through carefully orchestrated reputation manipulation.

\subsection{Lack of Context-Aware Threat Intelligence}
Traditional machine learning approaches applied to misbehavior detection and intrusion prevention in vehicular networks predominantly utilize classifiers such as Support Vector Machines, Random Forests, and deep learning architectures including Convolutional Neural Networks and Recurrent Neural Networks that process feature vectors extracted from individual node behaviors, packet characteristics, and temporal sequences \cite{gyawali2020machine}, \cite{belcastro2025enhancing}. These methods excel at identifying anomalous patterns in isolated data instances by learning statistical distributions of normal versus malicious behaviors, achieving high accuracy in binary or multi-class classification tasks that distinguish between attack types based on pre-defined feature sets \cite{gyawali2020machine}, \cite{belcastro2025enhancing}. However, Convolutional Neural Networks designed for grid-structured data like images and time-series cannot effectively capture the complex relational dependencies and graph-structured interactions inherent in vehicular network topologies where trust relationships, communication patterns, and accusation flows form intricate networks of interdependencies between vehicles, RSUs, and controllers \cite{zoubir2024integrating}, \cite{chattopadhyay2024gnn}. Recurrent Neural Networks and LSTM architectures, while capable of modeling temporal sequences, similarly fail to represent the spatial graph structure of vehicular networks and cannot leverage topological features such as node centrality, community structures, accusation propagation paths, and structural anomalies that indicate coordinated false accusation campaigns \cite{zoubir2024integrating}, \cite{chattopadhyay2024gnn}, \cite{belcastro2025enhancing}. Existing blockchain-based trust management systems aggregate reputation scores arithmetically without considering the network context of who is accusing whom, the structural positions of accusers and accused within the vehicular network graph, or the patterns of accusation flows that could reveal coordinated attacks \cite{gazdar2022decentralized}, \cite{chang2025blockchain}, \cite{li2020blockchain}.

The research gap manifests in two critical dimensions requiring fundamentally different analytical approaches. First, limited application of Graph Neural Networks to model vehicular network topologies as graphs where nodes represent vehicles and edges represent communication links, trust relationships, or accusation flows, preventing systems from detecting structural anomalies such as sudden accusation clustering around specific nodes, unusual propagation patterns of false reports through the network, or coordinated attack signatures where multiple accusers with similar network positions simultaneously target legitimate vehicles \cite{zoubir2024integrating}, \cite{chattopadhyay2024gnn}. GNNs' message-passing mechanisms that aggregate neighborhood information could identify anomalous accusation clusters by analyzing whether accusation patterns deviate from expected network behavior, detecting Sybil attacks through structural analysis of whether multiple accusers exhibit suspiciously similar connectivity patterns, and recognizing collusion by identifying communities of nodes that consistently coordinate their accusations, yet these capabilities remain unexploited in vehicular false accusation detection research \cite{zoubir2024integrating}, \cite{chattopadhyay2024gnn}. Second, complete absence of Large Language Model-based contextual analysis for interpreting complex multi-source evidence including natural language logs, structured accusation messages, behavioral descriptions, and heterogeneous data from network monitoring, preventing systems from performing sophisticated reasoning about accusation authenticity, evidence consistency, and threat context \cite{belcastro2025enhancing}. LLMs' advanced natural language understanding could enable analysis of accusation message semantics to detect linguistically suspicious patterns in false reports, cross-referencing of textual evidence with behavioral data to identify inconsistencies suggesting fabrication, interpretation of system logs to contextualize whether reported misbehavior aligns with observed network conditions, and generation of human-interpretable explanations for detection decisions to support incident response, yet no existing work leverages these capabilities for false accusation defense \cite{belcastro2025enhancing}. This dual gap—absence of GNN-based network topology analysis combined with lack of LLM-driven contextual evidence interpretation—leaves SDVN environments without intelligent threat assessment capabilities necessary to detect sophisticated coordinated false accusation attacks that manifest through both structural network patterns and subtle contextual inconsistencies in accusation evidence.


\chapter{Methodology}
\section{Research Design}
This research proposes a blockchain-based security framework integrated with artificial intelligence to mitigate False Accusation Attacks in Software-Defined Vehicular Networks (SDVN). The framework operates on a Zero-Trust Architecture principle, where no network entity—vehicles, roadside units, or controllers—is trusted by default \cite{rose2020zero}. Instead, continuous verification and validation mechanisms ensure that only legitimate nodes participate in routing decisions.

The methodology addresses four distinct attack variants that exploit reputation management systems in SDVN: (1) single-accuser fabrications where privileged nodes falsely accuse legitimate vehicles, (2) Sybil-amplified attacks where multiple fake identities create artificial consensus, (3) timing-based accusations that exploit network congestion periods, and (4) evidence-spoofing through tampered logs \cite{khan2017topology}, \cite{arif2020sdn}. Each variant requires specialized detection and mitigation strategies integrated into a cohesive defense framework.

\section{Threat Model}
The adversary model adopts a Zero-Trust Architecture where any network component may potentially be compromised. The threat model defines attacker capabilities, attack scenarios, and security assumptions \cite{adnan2021towards}.

\subsection{Attacker Capabilities}
\textbf{C1 - Data Plane Compromise:} Attackers can compromise up to n-f vehicular nodes and RSUs where n is the total number of nodes. Compromised nodes can participate in routing and reputation voting while executing malicious behaviors such as false accusations or packet manipulation.

\textbf{C2 - Control Plane Compromise:} Adversaries may compromise up to nc-1 SDN controllers out of nc total controllers (e.g., 3 out of 4). The blockchain ordering service tolerates up to $\lfloor(n_o-1)/3\rfloor$ Byzantine faults through BFT-SMaRt consensus, with additional resilience provided by the Controller Trust Evaluation Smart Contract \cite{yahiatene2018blockchain}.

\textbf{C3 - Wireless Communication Attacks:} Attackers can intercept, modify, replay, or flood V2V and V2I wireless transmissions over IEEE 802.11p DSRC channels.

\textbf{C4 - Computational Bounds:} Adversaries are computationally powerful but bounded by classical computing. Post-quantum cryptography (FALCON-1024, Kyber-1024) protects against future quantum threats, while frequent key rotation mitigates classical cryptanalysis \cite{bensasson2018scalable}.

\subsection{Attack Scenarios}
\textbf{Scenario 1 - Single-Accuser Fabrication:} A privileged malicious node with high reputation fabricates false evidence against honest vehicles, leveraging trusted status to cause blacklisting.

\textbf{Scenario 2 - Sybil-Amplified Flooding:} Multiple fake identities coordinate to flood the reputation system with false accusations, creating artificial consensus.

\textbf{Scenario 3 - Timing-Based Accusations:} Attackers exploit legitimate network stress events (congestion, handovers) to mask malicious accusations as genuine failures.

\textbf{Scenario 4 - Evidence-Spoofing:} Adversaries fabricate tampered packet logs, forged routing traces, or manipulated sensor data to frame honest nodes.

\section{System Assumptions}
The framework operates under the following security and infrastructure assumptions:
\begin{itemize}
    \item \textbf{A1 - Cryptographic Security:} All cryptographic primitives (FALCON-1024, Kyber-1024, RSA-2048, ECDH-256, AES-256, HMAC-SHA-256) are secure against known attacks when correctly implemented \cite{bensasson2018scalable}.
    \item \textbf{A2 - Byzantine Fault Tolerance:} BFT-SMaRt consensus tolerates up to $\lfloor(n_o-1)/3\rfloor$ Byzantine faults among orderers, maintaining safety and liveness \cite{castro1999practical}.
    \item \textbf{A3 - Bootstrap Trust:} At least one trusted controller exists during initial deployment for credential distribution and blockchain initialization.
    \item \textbf{A4 - Physical Layer Availability:} Underlying communication infrastructure (IEEE 802.11p, cellular backhaul) provides sufficient availability for safety-critical applications.
    \item \textbf{A5 - Time Synchronization:} All entities maintain loosely synchronized clocks ($\pm$1 second) via GPS or NTP for timestamp-based replay detection.
    \item \textbf{A6 - Computational Resources:} Vehicles and RSUs possess sufficient processing power (multi-core CPUs, HSMs) for cryptographic operations and ML inference.
    \item \textbf{A7 - Registration Authority:} A trusted registration authority (manufacturer, operator, or government agency) issues initial credentials tied to physical vehicle identities.
    \item \textbf{A8 - Honest Majority:} The majority of network participants are honest, with high-trust endorsers predominantly following protocol specifications.
\end{itemize}

\section{Blockchain-Based Decentralized Trust Management}
The core of our approach utilizes blockchain technology to eliminate single points of failure inherent in centralized SDN architectures \cite{sharma2017distblocknet}. A permissioned blockchain serves as a distributed ledger that records all network events, routing decisions, and reputation changes immutably. This prevents attackers from erasing evidence of malicious behavior or manipulating historical records. The blockchain operates across multiple SDN controllers, forming a consortium where critical decisions require consensus among trusted nodes \cite{yahiatene2018blockchain}.

Three specialized smart contracts govern different security aspects. The Registration and Authentication Smart Contract manage node identity verification using Zero-Knowledge Proofs, allowing vehicles to prove their authenticity without revealing sensitive information such as vehicle identification numbers or driver identities \cite{bensasson2018scalable}. This privacy-preserving mechanism prevents attackers from tracking specific vehicles while ensuring only registered nodes participate in the network.

The Reputation Management Smart Contract implements weighted endorser trust scoring to prevent false accusations from influencing blacklist decisions \cite{alshaibani2023blockchain}. When a node accuses another of malicious behavior, the accusation undergoes multi-endorser validation where each endorser's vote is weighted by their historical accuracy. New nodes and previously malicious nodes carry minimal weight, preventing Sybil attackers from immediately impacting reputation scores. Reputation changes require consensus from multiple high-trust endorsers, typically requiring agreement from at least two-thirds of weighted trust scores \cite{bessani2014state}.

The Controller Trust Evaluation Smart Contract addresses the critical threat of controller compromise through continuous behavioral monitoring \cite{khan2017topology}. Random subsets of nodes validate whether the controller's reported network topology matches actual packet flows. Discrepancies trigger trust score reductions, and if controller trust falls below a threshold, the system automatically switches to a pre-designated backup controller, ensuring network continuity even under insider attacks \cite{raja2020energy}.

\section{AI-Driven Anomaly Detection}
While blockchain provides immutability and consensus, artificial intelligence adds intelligent pattern recognition to detect coordinated attacks that might evade rule-based systems. The vehicular network is modeled as a temporal graph where vehicles and roadside units form nodes, and communication links form edges \cite{zhou2020automating}. Graph Neural Networks (GNN) analyze this dynamic topology to identify anomalous patterns indicative of coordinated attacks.

Specifically, the GNN detects Sybil attack clusters by identifying groups of nodes with suspiciously similar behavioral patterns, identical registration timestamps, or star-topology accusation patterns where multiple accusers simultaneously target a single victim \cite{balaram2023highly}. Temporal attention mechanisms track how reputation scores change over time, flagging sudden drops that correlate with accusation floods. Recent research demonstrates that heterogeneous graph attention networks achieve over 99\% accuracy in identifying Sybil nodes in vehicular networks \cite{chen2025sybil}.

Large Language Models (LLM) complement the GNN by processing unstructured data sources such as controller logs, error messages, and system alerts \cite{zhang2024large}. The LLM performs semantic reasoning to understand causal relationships between events, for example, recognizing that a controller failover occurring minutes before a reputation spike suggests a timing-based attack. The LLM generates human-readable incident reports that explain detected attacks, identify suspected malicious nodes, assess confidence levels, and recommend mitigation actions. This interpretability is crucial for network administrators who must make final decisions on high-stakes actions like controller failover.

\section{Cryptographic Defense Mechanisms}
Cryptographic protocols form the foundation of attack prevention. Digital signatures ensure that every packet and accusation can be attributed to a specific authenticated node, preventing evidence fabrication \cite{pournaghi2020necppa}. Post-quantum cryptography algorithms protect against future quantum computing threats, particularly important for long-term infrastructure like vehicular networks. Time-stamped authentication tokens with limited validity windows prevent replay attacks where adversaries re-transmit old legitimate packets to disrupt routing \cite{benjaballah2021security}.

Location verification smart contracts validate that nodes claiming specific geographic positions are present there, preventing wormhole attacks where distant colluding nodes pretend to be neighbors \cite{quevedo2020intelligent}. Cross-validation with multiple roadside units ensures location claims are accurate, and kinematic constraints (maximum velocity and acceleration limits) detect physically impossible location changes that indicate GPS spoofing.

\section{Integrated Defense Workflow}
The complete system operates through continuous cycles. Vehicles authenticate periodically using zero-knowledge proofs, proving their legitimacy without revealing identities. During each authentication cycle, nodes exchange routing information and may submit accusations if they observe suspicious behavior. All accusations include cryptographic evidence—digital signatures, timestamps, and location proofs—which smart contracts validate before accepting.

Simultaneously, the GNN analyzes the network graph updated every few seconds, identifying clusters and patterns. When anomalies are detected, the LLM processes this alongside system logs to generate threat assessments. If both AI systems and weighted endorser consensus agree that an attack is occurring, the reputation management contract updates scores and potentially blacklists malicious nodes. Throughout this process, the controller trusts evaluation contract monitors for insider threats, ready to trigger failover if needed.

This multi-layered approach ensures that attackers must simultaneously defeat multiple independent security mechanisms—cryptographic verification, blockchain consensus, weighted trust scoring, and AI anomaly detection—making successful attacks computationally infeasible \cite{huo2023trustgnn}. The framework maintains vehicular network performance requirements while providing comprehensive security against reputation manipulation attacks.

\section{Attack-Specific Mitigation Strategies}
The proposed methodology addresses each false accusation attack variant through targeted defense mechanisms:

\subsection{Single-Accuser Opportunistic Fabrication}
This attack occurs when a single privileged malicious node fabricates convincing evidence against legitimate vehicles. The weighted endorser trust mechanism in the Reputation Management Smart Contract prevents this by requiring accusations to be validated by multiple high-trust endorsers \cite{alshaibani2023blockchain}. A single accuser, regardless of how convincing their evidence appears, contributes only their weighted trust score to the reputation change calculation. Since blacklisting requires $\geq$2/3 weighted consensus, a lone attacker cannot unilaterally blacklist a victim unless multiple independent high-trust nodes corroborate the accusation.

Additionally, cryptographic evidence validation ensures that accusations lacking proper digital signatures, valid timestamps, or verifiable location proofs are automatically rejected. The blockchain's immutable audit logs all accusations with timestamps, allowing forensic analysis to identify nodes that frequently make false accusations, gradually reducing their trust weights to near-zero \cite{bessani2014state}.

\subsection{Sybil-Amplified Consensus Flooding}
Sybil attacks create multiple fake identities to flood the network with coordinated accusations, attempting to achieve artificial consensus. The methodology employs three defensive layers. First, the Registration Smart Contract with Zero-Knowledge Proofs ensures that each node possesses unique cryptographic credentials tied to physical vehicle identities \cite{bensasson2018scalable}. Creating Sybil identities requires either stealing legitimate credentials or registering through out-of-band channels, both of which are computationally expensive and detectable.

Second, new nodes start with minimal trust weights. Even if an attacker successfully registers multiple Sybil identities, they initially carry near-zero influence in reputation decisions. Trust accumulates gradually only through consistent honest behavior over extended periods, making instant consensus flooding ineffective \cite{balaram2023highly}.

Third, the Graph Neural Network detects coordinated attack patterns by analyzing the network topology \cite{chen2025sybil}. When multiple low-trust nodes simultaneously target a single victim, the GNN identifies this star-topology accusation pattern as anomalous. Features such as identical registration timestamps, similar behavioral patterns, and temporal clustering of accusations trigger high-confidence Sybil detection, prompting immediate investigation and potential preemptive blacklisting of the attacker cluster.

\subsection{Timing-Based Accusations During High-Noise Periods}
Sophisticated attackers exploit periods of legitimate network stress—congestion, controller handovers, or channel degradation—to mask malicious accusations as genuine failures. The temporal attention mechanism in the GNN specifically addresses this by analyzing accusation timing relative to network conditions \cite{zhou2020automating}. The system maintains historical baselines of legitimate packet loss rates during various network states (normal operation, moderate congestion, severe congestion, handover periods).

When accusations occur during high-noise periods, the LLM performs semantic reasoning to distinguish between expected failures and coordinated attacks \cite{zhang2024large}. For instance, if multiple vehicles near a congestion zone report packet loss, this correlates with legitimate network stress. However, if accusations target only specific nodes while other nearby vehicles report normal performance, this discrepancy flags a timing-based attack.

Furthermore, the weighted endorser mechanism inherently provides resilience: during genuine high-noise periods, multiple independent high-trust endorsers will naturally corroborate packet loss reports. Malicious timing-based accusations lack this broad corroboration pattern, revealing them as outliers despite the noisy environment.

\subsection{Evidence-Spoofing with Tampered Logs and Collusion}
The most sophisticated attack involves fabricating cryptographically convincing but falsified evidence—tampered packet logs, forged signatures, or collusion among multiple compromised nodes. The methodology employs multi-layer cryptographic validation as the primary defense \cite{pournaghi2020necppa}. Every accusation must include: (1) valid digital signatures from authenticated nodes, (2) unfalsifiable timestamps within acceptable time windows, (3) Zero-Knowledge Proofs demonstrating the accuser's legitimate network presence, and (4) location proofs validated by multiple independent roadside units \cite{quevedo2020intelligent}.

The blockchain's immutable chain-of-custody provides additional protection. Evidence must include traceable hashes of original packets stored on the blockchain at capture time. Retrospective evidence fabrication is impossible because the blockchain timestamp proves when data was recorded. Attackers cannot backdate evidence to match historical events \cite{sharma2017distblocknet}.

When facing collusion among multiple compromised nodes, the Controller Trust Evaluation Smart Contract becomes critical \cite{raja2020energy}. If multiple nodes collude to provide fabricated corroborating evidence, but their reports conflict with the controller's observed packet flows, the cross-endorsement validation mechanism detects this discrepancy. The system then treats the entire colluding group as suspect. Moreover, post-quantum cryptography ensures that even computationally powerful adversaries cannot forge signatures or break encryption to manufacture fake evidence \cite{bensasson2018scalable}.

In scenarios where the SDN controller itself is compromised and colluding with malicious nodes, the Controller Trust Evaluation Smart Contract continuously monitors controller behavior through random cross-endorsement validation \cite{khan2017topology}. If the controller manipulates reputation scores or routing tables to favor colluding attackers, its trust score decreases. Once trust falls below threshold, the blockchain automatically triggers failover to a pre-designated backup controller from a different administrative domain, neutralizing the insider threat.

\section{Validation Approach}
The proposed framework will be validated through simulation-based evaluation where all five false accusation attack variants are implemented at varying intensities from 0\% to 100\% network penetration. Performance will be measured using standard vehicular network metrics: Packet Delivery Ratio (percentage of packets successfully delivered), Packet Interception Ratio (percentage compromised by attackers), and Matthews Correlation Coefficient for attack detection accuracy \cite{nayak2021tbddosa}.

Comparative analysis against existing approaches, including traditional trust-based systems, pure cryptographic methods, and basic blockchain implementations—will demonstrate the framework's superiority in maintaining network performance under attack while correctly identifying and isolating malicious nodes. Statistical significance will be assessed through repeated trials with different network topologies and mobility patterns, ensuring results are not artifacts of specific scenarios \cite{adnan2021towards}.

\chapter{Timeline and Resource Required}

\section{Timeline}
Provide a realistic timeline for completing the project. Use a Gantt chart or table.
\renewcommand{\arraystretch}{1.2}
\setlength{\tabcolsep}{3pt}

\begin{sidewaystable}[htbp]
\centering
\scriptsize

\caption{Project Timeline}
\resizebox{\textheight}{!}{%
\begin{tabular}{|p{6cm}|*{40}{c|}}
\hline

% ---------------- Year row ----------------
\multirow{2}{*}{\textbf{Task}}
& \multicolumn{8}{c|}{\textbf{2025}}
& \multicolumn{32}{c|}{\textbf{2026}} \\ \cline{2-41}

% ---------------- Month row ----------------
& \multicolumn{4}{c|}{November}
& \multicolumn{4}{c|}{December}
& \multicolumn{4}{c|}{January}
& \multicolumn{4}{c|}{February}
& \multicolumn{4}{c|}{March}
& \multicolumn{4}{c|}{April}
& \multicolumn{4}{c|}{May}
& \multicolumn{4}{c|}{June}
& \multicolumn{4}{c|}{July}
& \multicolumn{4}{c|}{August} \\ \cline{2-41}

% ---------------- Week numbers ----------------
& 1 & 2 & 3 & 4
& 1 & 2 & 3 & 4
& 1 & 2 & 3 & 4
& 1 & 2 & 3 & 4
& 1 & 2 & 3 & 4
& 1 & 2 & 3 & 4
& 1 & 2 & 3 & 4
& 1 & 2 & 3 & 4
& 1 & 2 & 3 & 4
& 1 & 2 & 3 & 4 \\ \hline

% ---------------- Tasks ----------------
Select the project topic
& \cellcolor{red} & & & 
& & & &
& & & &
& & & &
& & & &
& & & &
& & & &
& & & &
& & & &
& & & & \\ \hline

Discuss with supervisors and co-supervisors
& \cellcolor{red} & & & 
& & & &
& & & &
& & & &
& & & &
& & & &
& & & &
& & & &
& & & &
& & & & \\ \hline

Study the related work and implementation
& & \cellcolor{red} & \cellcolor{red} & \cellcolor{red}
& & &  & 
& & & &
& & & &
& & & &
& & & &
& & & &
& & & &
& & & &
& & & & \\ \hline

Prepare the project proposal and presentation
& & & & 
& \cellcolor{red} & \cellcolor{red} & & 
& & & &
& & & &
& & & &
& & & &
& & & &
& & & &
& & & &
& & & & \\ \hline

Study related technologies
& & & & 
& & & \cellcolor{red} & \cellcolor{red}
& \cellcolor{red} & & &
& & & &
& & & &
& & & &
& & & &
& & & &
& & & &
& & & & \\ \hline

Model sample LV network
& & & & 
& & & &
& & \cellcolor{red} & \cellcolor{red} & 
& & & &
& & & &
& & & &
& & & &
& & & &
& & & &
& & & & \\ \hline

Impact assessment for sample network
& & & & 
& & & &
& & & & \cellcolor{red}
& \cellcolor{red} & & &
& & & &
& & & &
& & & &
& & & &
& & & &
& & & & \\ \hline

Collect required actual data from LECO
& & & & 
& & & &
& & & &
& & \cellcolor{red} & & 
& & & &
& & & &
& & & &
& & & &
& & & &
& & & & \\ \hline

Model actual LV network using OpenDSS software
& & & & 
& & & &
& & & &
& & & \cellcolor{red} & \cellcolor{red}
& & & &
& & & & 
& & & &
& & & &
& & & &
& & & & \\ \hline

Model the demand profile
& & & & 
& & & &
& & & &
& & & & 
& \cellcolor{red} & & &
& & & &
& & & &
& & & &
& & & &
& & & & \\ \hline

Impact assessment for increasing PV penetration 
& & & & 
& & & &
& & & &
& & & &
& & \cellcolor{red} & \cellcolor{red} &
& & & &
& & & &
& & & &
& & & &
& & & & \\ \hline

Impact assessment for increasing EVCS penetration
& & & & 
& & & &
& & & &
& & & &
& & & & \cellcolor{red}
& \cellcolor{red} & & &
& & & &
& & & &
& & & &
& & & & \\ \hline

Identifying the optimization techniques
& & & & 
& & & &
& & & &
& & & &
& & & &
& & \cellcolor{red} & \cellcolor{red} &
& & & &
& & & &
& & & &
& & & & \\ \hline

Developing optimum power flow algorithm for day head market
& & & & 
& & & &
& & & &
& & & &
& & & &
& & & & \cellcolor{red}
& \cellcolor{red} & \cellcolor{red} & \cellcolor{red} &
& & & &
& & & &
& & & & \\ \hline

Developing optimum power flow algorithm for intraday market
& & & & 
& & & &
& & & &
& & & &
& & & &
& & & &
& & & & \cellcolor{red}
& \cellcolor{red} & \cellcolor{red} & \cellcolor{red} &
& & & &
& & & & \\ \hline

Integrating Blockchain technology with smart contract
& & & & 
& & & &
& & & &
& & & &
& & & &
& & & &
& & & & 
& & & & \cellcolor{red}
& \cellcolor{red} & \cellcolor{red} & \cellcolor{red} &
& & & & \\ \hline

Testing and validation of developed 
& & & & 
& & & &
& & & &
& & & &
& & & &
& & & &
& & & &
& & & &
& & & & \cellcolor{red}
& \cellcolor{red} & & & \\ \hline

Prepare final report and presentation
& & & & 
& & & &
& & & &
& & & &
& & & &
& & & &
& & & &
& & & &
& & & &
& & \cellcolor{red} & \cellcolor{red} & \cellcolor{red} \cellcolor{red} \\ \hline


\end{tabular}}
\end{sidewaystable}



\section{Resource Required}
State the resource required for the project and estimate the budget for the resources required.

\chapter{Conclusion}
Summarize the key points of your proposal and reiterate the importance of the project.

% References
\renewcommand{\bibname}{References}
\bibliographystyle{ieeetr}
\addcontentsline{toc}{chapter}{References} % Add to table of contents
\bibliography{bibliography} 
%\printbibliography

\end{document}